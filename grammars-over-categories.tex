\documentclass[12pt]{article}
%AMS-TeX packages
\usepackage{tikz}
\usepackage[utf8]{inputenc}
\usepackage{amssymb,amsmath,amsthm,stmaryrd, wasysym}
\usepackage{mathtools}
%geometry (sets margin) and other useful packages
\usepackage[margin=1.25in]{geometry}
\usepackage{graphicx,ctable,booktabs}
\usepackage{tikz-cd}
\usepackage{mathpartir}

\usepackage[sort&compress,square,comma,authoryear]{natbib}
\bibliographystyle{plainnat}

\newcommand{\leftmultimap}{\mathop{\rotatebox[origin=c]{180}{$\multimap$}}}
\newcommand{\Set}{\textrm{Set}}
\newcommand{\FinSet}{\mathbb F}
\newcommand{\Cat}{\textrm{Cat}}
\newcommand{\Lang}{\textrm{Lang}}
\newcommand{\Grammar}{\textrm{Grammar}}
\newcommand{\Action}{\textrm{Action}}
\newcommand{\sem}[1]{\llbracket{#1}\rrbracket}
\newtheorem{definition}{Definition}
\newtheorem{theorem}{Theorem}

\newcommand\rsepimp{\mathbin{-\mkern-6mu*}}
\newcommand\lsepimp{\mathbin{\rotatebox[origin=c]{180}{$-\mkern-6mu*$}}}
\newcommand\sepconj{\mathbin{*}}
\newcommand\bunch{\mathcal W}

\begin{document}
\title{Grammars over Free Categories}
\author{Max S. New}
\maketitle

We consider the semantics of \emph{formal grammars} over free
categories as a generalization of the notion of a formal grammar over
a free monoid. Whereas free monoids are a mathematical formalization
of completely \emph{unstructured} data, free categories are a very
simple class of \emph{structured} data, and so formal grammars over
them provide a simple setting for describing grammars over
\emph{typed} expression languages: those whose syntax is entirely
sequential.

As an example, we might consider a simple Hoare logic over a typed
sequential programming language. \ldots TODO examples

A signature for a free category $\Sigma$ is a \emph{quiver}, also
known as a directed multigraph. Then this is completed to the free
category $F\Sigma$. If the quiver $\Sigma$ has exactly one object,
then the free category is the delooping of the free monoid on the set
of edges $\Sigma_e^*$. In this way, formal grammars over free monoids,
i.e., strings, will be a special case.

\section{A Bicategory of Grammars}

A formal grammar over a free monoid $\Sigma^*$ denotes a set of
\emph{parses} for each string. This is fruitfully modeled as an object
of the presheaf category $\Set^{\Sigma^*}$, where $\Sigma^*$ here is
viewed as the \emph{discrete} category on the elements of
$\Sigma^*$. In plain English: a grammar\footnote{Here we are taking a
maximally broad view of what a grammar can be, abstracted from any
syntactic notion of formal grammar. A Lawverian terminology would call
this an \emph{objective} grammar as opposed to the \emph{subjective}
grammars from specific grammar formalisms such as regular expressions,
CFGs etc.} is a function from strings to the set of parses of the
string.

Several important structures immediately follow from this definition:
\begin{enumerate}
\item Since $\Set^{\Sigma^*}$ is a category, we get a notion of
  \emph{morphism} of grammars: a function that translates parses of a
  string $w$ in the domain grammar to parses of the same string in the
  codomain grammar.
\item Since $\Set^{\Sigma^*}$ is a presheaf category, we have the
  \emph{Yoneda embedding} $\Sigma^* \to \Set^{\Sigma^*}$. This says
  for each word $w$ in the language there is a corresponding grammar
  $w$ that has a unique parse for $w$ and no parses for any other
  string.
\item Since $\Sigma^*$ is a monoid, $\Set^{\Sigma^*}$ is monoidal
  closed with the \emph{Day Convolution} monoidal closed
  structure. The monoidal product is the sequencing of grammars and
  the monoidal unit is the grammar that has a unique parse for only
  the empty string.
\item Additionally any presheaf category is a bi-cartesian closed
  category, which gives us conjunction, disjunction and implication of
  grammars.
\end{enumerate}
Note that because we are working with a free monoid, we only ever need
to apply the Yoneda embedding to individual generators as the Yoneda
embedding is a strong monoidal functor.

To generalize this to free categories, note that the elements of the
monoid $\Sigma^*$ are in the free category view \emph{morphisms} of
the free category: the endomorphisms of the single object
$F\Sigma(*,*)$. This suggests that a \emph{grammar} over a free
category be paramaterized by a source and target object, that is, we
have for each pair of objects $a,b$, a grammar over the morphisms
$F\Sigma(a,b)$, which we can define as $\Set^{F\Sigma(a,b)}$. We will
call this category $\Grammar(a,b)$.

\begin{enumerate}
\item First, a morphism between grammars is the same as before: a
  morphism from $\alpha: G \to H : \Grammar(a,b)$ is a family of
  functions $\alpha_f : G(f) \to H(f)$ for every $f : a \to b$.
\item Now for each morphism $f : a \to b$ in $F\Sigma(a,b)$ we have
  the Yoneda embedding as a grammar $Yf : \Grammar(a,b)$ that has a
  unique parse of only $f$, i.e., $(Yf)(g) = \{ * \,|\, f = g \}$. As
  above we only need to use this when $f$ is a generator.
\item We generalize the monoidal structure on grammars over strings to
  a \emph{biclosed bicategory} structure on grammars over typed
  strings. That is for each object $a$ we get an ``empty string''
  grammar $\epsilon_a : \Grammar(a,a)$ that matches only the identity
  morphism. Next, given grammars $G : \Grammar(a,b)$ and $H :
  \Grammar(b, c)$ we can define a \emph{composition} grammar $G
  \odot_b H : \Grammar(a,c)$ by
  \[ (G \odot_b H)f = \sum_{g: a \to b}\sum_{h: b \to c} Gg \times Hh \]
  Additionally, since the composition is (inherently) non-symmetric we get a covariant and contravariant hom structure:
  \[ \inferrule{G : \Grammar(c,b) \and H : \Grammar(a,b)}{G \triangleright_{b} H : \Grammar(a,c)} \]
  defined as
  \[ (G \triangleright_{b} H)(f : a \to c) = \prod_{g : c \to b} G(g) \to H(gf) \]
  and a dual
  \[ \inferrule{H : \Grammar(b,c)\and G : \Grammar(b,a) }{H \mathop{\prescript{}{b}\triangleleft} G : \Grammar(a,c)} \]
  defined as
  \[ (H \mathop{\prescript{}{b}\triangleleft} G)(f : a \to c) = \prod_{g: b \to a} G(g) \to H(fg) \]
\item Additionally, each hom-category is a bicartesian closed
  category, giving us analogous disjunction, conjunction and
  implication of grammars.
\end{enumerate}

\end{document}
