\documentclass[acmsmall,anonymous,review,screen]{acmart}
\usepackage{mathpartir}
\usepackage{quiver}
\usepackage{enumitem}
\usepackage{wrapfig}
\usepackage{fancyvrb}
\usepackage{amsmath}
\usepackage{stmaryrd}

\DeclareFontFamily{U}{dmjhira}{}
\DeclareFontShape{U}{dmjhira}{m}{n}{ <-> dmjhira }{}

\DeclareRobustCommand{\yo}{\text{\usefont{U}{dmjhira}{m}{n}\symbol{"48}}}

\usepackage{xstring} % Need it for strlen
\newcommand{\cat}[1]{
  \relax
  \StrLen{#1}[\arglength]
  \ifnum\arglength=1
  \mathcal{#1}
  \else
  \mathbf{#1}
  \fi
}

\newcommand{\FKC}{\textrm{F}_{\textrm{KC}}}
\newcommand{\sem}[1]{\llbracket{#1}\rrbracket}
\newcommand{\Set}{\cat{Set}}
\newcommand{\lto}{\multimap}
\newcommand{\grammar}{\Set^{\cat{\Sigma^*}}}
\newcommand{\id}{\textrm{id}}
\newtheorem*{construction}{Construction}

%% Rights management information.  This information is sent to you
%% when you complete the rights form.  These commands have SAMPLE
%% values in them; it is your responsibility as an author to replace
%% the commands and values with those provided to you when you
%% complete the rights form.
\setcopyright{acmcopyright}
\copyrightyear{2018}
\acmYear{2018}
\acmDOI{XXXXXXX.XXXXXXX}

%% These commands are for a PROCEEDINGS abstract or paper.
\acmConference[Woodstock '18]{Woodstock '18: ACM Symposium on Neural
  Gaze Detection}{June 03--05, 2018}{Woodstock, NY}
\acmBooktitle{Woodstock '18: ACM Symposium on Neural Gaze Detection,
  June 03--05, 2018, Woodstock, NY}
\acmPrice{15.00}
\acmISBN{978-1-4503-XXXX-X/18/06}


\begin{document}
\title{Kleene Categories}
\author{Steven Schaefer}
\affiliation{
  \department{Electrical Engineering and Computer Science}
  \institution{University of Michigan}
  \country{USA}
}
\email{stschaef@umich.edu}

\author{Max S. New}
\affiliation{
  \department{Electrical Engineering and Computer Science}
  \institution{University of Michigan}
  \country{USA}
}
\email{maxsnew@umich.edu}

\author{Pedro H. Azevedo de Amorim}
\affiliation{
  \department{Department of Computer Science}
  \institution{University of Oxford}
  \country{UK}
}
\email{pedro.azevedo.de.amorim@cs.ox.ac.uk}

\begin{abstract}
  We propose a categorification of Kleene algebras and study some of its basic properties.
\end{abstract}

\maketitle


\section{Kleene Algebra and Kleene Categories}

Kleene algebras are an important tool in the theory of regular
languages. More broadly, they serve as a theoretical substrate to
studying various kinds of formal languages. Formally, a Kleene algebra is a tuple
$(A, +, \cdot, (-)^*, 1, 0)$, where $A$ is a set, $+$ and $\cdot$
are binary operations over $A$, $(-)^*$ is a function over $A$, and
$1$ and $0$ are constants. These structures satisfy the axoims depicted
in Figure~\ref{fig:axioms}.

\begin{figure}
  \begin{align*}
    x + (y + z) &= (x + y) + z & x + y &= y + x\\
    x + 0 &= x & x + x &= x\\
    x(yz) &= (xy)z & x1 &= 1x = x\\
    x(y + z) &= xy + xz & (x + y)z &= xz + yz\\
    x0 &= 0x = x & & \\
    1 + aa^* &\leq a^* & 1 + a^*a &\leq a^*\\
     b + ax \leq x &\implies a^*b \leq x &  b + xa \leq x &\implies ba^* \leq x
  \end{align*}
  \label{fig:axioms}
  \caption{Kleene algebra axioms}
\end{figure}

The addition operation induces a partial order structure where $a \leq
b$ if $a + b = b$. In the theory of formal languages this order
structure models language containment.

\begin{example}
  The free Kleene algebra over a set $\Sigma$ is given by regular
  expressions with alphabet $\Sigma$, quotiented by the Kleene algebra
  axioms.
\end{example}

Kleene algebra provides an algebraic foundation for the implementation
of langauge \emph{recognizers}, i.e., programs that determine whether
or not a string is a member of a formal language. On the other hand,
the fact that $+$ is idempotent means that a Kleene algebra cannot
distinguish between multiple parses of the same string, and so Kleene
algebra is insufficient for the discussion of \emph{parsers} which may
select one of several different possible parses in the case of
ambiguous grammars.

Our goal is to provide a generalization of Kleene algebra that is
``proof-relevant'', and so can provide a foundation for parsing
regular expressions in the same way that Kleene algebra does for
recognition. To do this we need to move beyond the view of a Kleene
algebra as algebraic structure over sets, and instead as a kind of
structured \emph{poset}. We then generalize this to a Kleene
\emph{category}, where the morphisms of the category correspond to not
just language inclusion but \emph{formal reductions} of grammars, and
isomorphism is not language equivalence but instead \emph{strong
equivalence} of grammars.

The most complex part of this generalization is the appropriate
generalization of the Kleene star. For that we make the following
definitions
\begin{definition}
  Let $\cat C$ be a monoidal category. Given an object $A \in \cat C$,
  a \textbf{parameterized cons-list object} consists of
  \begin{enumerate}
  \item an object $L \in \cat C$
  \item with morphisms $\textrm{nil} : I \to L$ and $\textrm{cons} : A \otimes L \to L$
  \item such that for every $P,B \in \cat C$, and morphisms $n : P \to
    B$ and $c : A \otimes B \to B$ there is a unique morphism $\textrm{foldr}(n,c) : L
    \otimes P \to B$ satisfying

    % https://q.uiver.app/#q=WzAsNixbMSwwLCJMIFxcb3RpbWVzIFAiXSxbMCwwLCJJIFxcb3RpbWVzIFAiXSxbMCwxLCJQIl0sWzIsMCwiQSBcXG90aW1lcyBMIFxcb3RpbWVzIFAiXSxbMiwxLCJBIFxcb3RpbWVzIEIiXSxbMSwxLCJCIl0sWzEsMiwiIiwwLHsibGV2ZWwiOjIsInN0eWxlIjp7ImhlYWQiOnsibmFtZSI6Im5vbmUifX19XSxbMiw1LCJuIl0sWzQsNSwiYyIsMl0sWzMsNCwiQSBcXG90aW1lcyBcXHRleHRybXtmb2xkcn0obixjKSJdLFszLDAsIlxcdGV4dHJte2NvbnN9IFxcb3RpbWVzIFAiLDJdLFsxLDAsIlxcdGV4dHJte25pbH0gXFxvdGltZXMgUCJdLFswLDUsIlxcdGV4dHJte2ZvbGRyfShuLGMpIl1d
\[\begin{tikzcd}[ampersand replacement=\&]
	{I \otimes P} \& {L \otimes P} \& {A \otimes L \otimes P} \\
	P \& B \& {A \otimes B}
	\arrow["{\textrm{nil} \otimes P}", from=1-1, to=1-2]
	\arrow[equals, from=1-1, to=2-1]
	\arrow["{\textrm{foldr}(n,c)}", from=1-2, to=2-2]
	\arrow["{\textrm{cons} \otimes P}"', from=1-3, to=1-2]
	\arrow["{A \otimes \textrm{foldr}(n,c)}", from=1-3, to=2-3]
	\arrow["n", from=2-1, to=2-2]
	\arrow["c"', from=2-3, to=2-2]
\end{tikzcd}\]
  \end{enumerate}

  A \textbf{cons-list object} is like a parameterized cons-list object but where $P$ can only be $I$.

  A \textbf{parameterized snoc-list object} is a cons-list object in
  the dual monoidal structure $\cat C^{co}$. In this case we call the
  $\textrm{cons}$ instead $\textrm{snoc}$ and the $\textrm{foldr}$
  morphism $\textrm{foldl}$. A non-parameterized snoc-list object is
  defined similarly.
\end{definition}

\begin{construction}
  Given a cons-list object for $A$, we can define a morphism $\textrm{snoc} : L \otimes A \to L$ as $\textrm{foldr}(\eta, \textrm{cons})$
  where $\eta : A \to L$ is the composition
  % https://q.uiver.app/#q=WzAsNCxbMCwwLCJBIl0sWzEsMCwiQSBcXG90aW1lcyBJIl0sWzIsMCwiQSBcXG90aW1lcyBMIl0sWzMsMCwiTCJdLFswLDEsIiIsMCx7ImxldmVsIjoyLCJzdHlsZSI6eyJoZWFkIjp7Im5hbWUiOiJub25lIn19fV0sWzEsMiwiQSBcXG90aW1lcyBcXHRleHRybXtuaWx9Il0sWzIsMywiXFx0ZXh0cm17Y29uc30iXV0=
\[\begin{tikzcd}[ampersand replacement=\&]
	A \& {A \otimes I} \& {A \otimes L} \& L
	\arrow[equals, from=1-1, to=1-2]
	\arrow["{A \otimes \textrm{nil}}", from=1-2, to=1-3]
	\arrow["{\textrm{cons}}", from=1-3, to=1-4]
\end{tikzcd}\]
\end{construction}

\begin{definition}
  A Kleene category is a distributive monoidal category $\cat{K}$ such
  that for every object $A$, there is a cons-list object
  $(A^*,\textrm{nil},\textrm{cons})$ such that $(A^*, \textrm{nil},
  \textrm{snoc})$ is a snoc-list object.
\end{definition}

As a sanity check, note that Kleene algebras are indeed examples of
Kleene categories.

\begin{lemma}
  A Kleene algebra is equivalent to a posetal Kleene category.
\end{lemma}

% This isn't an example of a Kleene category, it's an example of a model of conjunctive ordered logic
An unexpected example comes from the theory of substructural logics.

\begin{example}
  The opposite category of every Kleene category is a model of a variant of
  conjunctive ordered logic, where the Kleene star plays the role of the ``of
  course'' modality from substructural logics.
\end{example}

This proposed definition is sensible. After all, the proposed axioms
are a direct translation of the Kleene algebra axioms to a categorical
setting. Its most convoluted aspect is the axiomatization of the
Kleene star as a parameterized list and snoc object. If the Kleene
category $\cat{K}$ has more structure, then these isomorphisms hold
``for free''.

\begin{definition}
  A pre-Kleene category is a distributive monoidal category with a (non-parameterized) cons-list object.
\end{definition}

\begin{theorem}
  If $\cat K$ is a pre-Kleene category that is also biclosed, then it is a Kleene category.
\end{theorem}
\begin{proof}
  TODO: closed implies parameterized for free and that cons-list and snoc-list objects are equivalent.
\end{proof}

%% \begin{theorem}
%%   \label{th:kleeneclosed}
%%   Let $\cat{K}$ be a Kleene category such that it is also monoidal
%%   closed.  Then, the initial algebras isomorphisms hold automatically.
%% \end{theorem}
%% \begin{proof}
%%   We prove this by the unicity (up-to isomorphism) of initial
%%   algebras. Let $[hd, tl]: 1 + (\mu X.\, F_{A, 1}(X)) \otimes A \to
%%   (\mu X.\, F_{A, 1}(X))$ be the initial algebra structure of $(\mu
%%   X.\, F_{A, 1}(X))$ and consider the map $[hd, tl] : B + B \otimes
%%   (\mu X.\, F_{A, 1}(X)) \otimes A \to B\otimes (\mu X.\, F_{A,
%%     1}(X))$.

%%   Now, let $[f,g] : B + A \otimes Y \to Y$ be an $F_{A,B}$-algebra and
%%   we want to show that there is a unique algebra morphism $h : B
%%   \otimes \mu X.\, F_{A,1} \to Y$. We can show existence and uniqueness
%%   by showing that the diagram on the left commute if, and only if,
%%   the diagram on the right commutes:

%%   This equivalence follows by using the adjunction structure given
%%   by the monoidal closed structure of $\cat{K}$. A completely analogous
%%   argument for $G_{A,B}$ also holds.
%% \end{proof}

This result feels similar in spirit to the definition of action
algebras, which are algebras where the product also has adjoint
operations which results in the Kleene star being more easily
axiomatized \cite{kozen1994}. We are now ready to prove that our
concept of formal grammars fit nicely within our categorical
framework. We start by presenting a well-known construction
from presheaf categories.

\begin{definition}
  Let $\cat{C}$ be a locally small monoidal category and $F$, $G$ be
  two functors $\cat{C} \to \Set$. Their Day convolution tensor
  product is defined as the following coend formula:
  \[
  (F \otimes_{Day} G)(x) = \int^{(y,z) \in \cat{C}\times\cat{C}}\cat{C}(y\otimes z, x) \times F(y) \times G(z) 
  \]
  Dually, its internal hom is given by the following end formula:
  \[
  (F \lto_{Day} G)(x) = \int_{y} \Set(F(y), G(x \otimes y))
  \]
\end{definition}
Note that in the case the category $\cat C$ is the discrete category
of a monoid, the Day convolution admits a simpler description where
the coend degenerates to a coproduct and the end ot a product.

\begin{lemma}[\cite{day1970}]
  Under the assumptions above, the presheaf category $\Set^{\cat{C}}$ is
  monoidal closed.
\end{lemma}

%% \begin{theorem}
%%   Let $\cat{K}$ be a Kleene category and $A$ a discrete category.
%%   The functor category $[A, \cat{K}]$.
%%   (HOW GENERAL SHOULD THIS THEOREM BE? BY ASSUMING ENOUGH STRUCTURE,
%%   E.G. K = Set, THIS THEOREM BECOMES SIMPLE TO PROVE)
%% \end{theorem}
\begin{theorem}
  If $\cat{C}$ is a locally small monoidal category, then
  $\Set^{\cat{C}}$ is a Kleene category.
\end{theorem}
\begin{proof}
  By the lemma above, $\Set^{\cat{C}}$ is monoidal closed, and since it
  is a presheaf category, it has coproducts, and since the tensor
  is a left adjoint, it preserves colimits and, therefore, it is
  a distributive category.

  As for the Kleene star, since presheaf categories admit small colimits,
  the initial algebra of the functors $F_{A,B}$ and $G_{A,B}$ can be
  defined as the filtered colimit of the diagrams:

  From Theorem~\ref{th:kleeneclosed} it follows that these initial
  algebras satisfy the required isomorphisms and this concludes the
  proof.
\end{proof}

\begin{corollary}
  For every alphabet $\Sigma$, the presheaf category $\grammar$
  is a Kleene category.
\end{corollary}
\begin{proof}
  Note that string concatenation and the empty string make the
  discrete category $\Sigma^*$ a strict monoidal category.
\end{proof}

\begin{definition}
  Let $K$ be a Kleene category over alphabet $\Sigma$. This
  induces a strong monoidal functor $\Sigma^* \to K$. The
  nerve functor $w \mapsto K(w, -)$ is the grammar interpretation
  functor of $K$.
\end{definition}

\section{Proof-Relevant Deterministic Automata}

Now, we will consider a categorification of deterministic automata.

\begin{definition}
  A proof-relevant DFA is an $F$-coalgebra, for the functor
  $F(\cat{C}) = \Set \times (\Sigma \Rightarrow \cat{C})$.
\end{definition}

Assuming that the underlying category $\cat{C}$ is not discrete,
arrows $f : X \to Y$ in $\cat{C}$ should be thought of as a proof that
the state $X$ is related to the $Y$. This can be seen as a
generalization of simulation relations. After all, functoriality of $F
: \cat{C} \times \Sigma \to \cat{C}$ implies that $F(f, id_c): F(X, c)
\to F(Y, c)$, i.e. after one transition, the end states will also be
related.

\begin{definition}
  We can equip $\grammar$ with the $F$-coalgebra structure $\langle \alpha_1, \alpha_2\rangle$,
  where $\alpha_1(F) = F(\varepsilon)$ and $\alpha_2(F, c) = \yo(c) \lto F$.
\end{definition}

\begin{theorem}
  The morphism $\alpha : \grammar \to F(\grammar)$ is the final
  $F$-coalgebra.
\end{theorem}
\begin{proof}
  Given an $F$-coalgebra $\langle \alpha, \delta\rangle : \cat{C} \to \Set \times \cat{C}^\Sigma$, we can define the functor $\cat{C} \times \Sigma^* \to \Set$ much like in the DFA case:
  by induction on $\Sigma^*$.
\end{proof}


\begin{comment}
\subsection{DFAs with simulation relation}

Let $\langle \tau, \delta\rangle : X \to 2 \times X^\Sigma$ be a DFA
over an alphabet $\Sigma$.

\begin{definition}
  A binary relation $\mathcal{R} \subseteq X \times X$ is a simulation if
  \begin{itemize}
  \item It is a preorder.
  \item If $x_1 \mathcal{R} x_2$ and $x_1$ is an accept state, so is $x_2$.
  \item If $x_1 \mathcal{R} x_2$ then for every $a : \Sigma$,
    $\delta(x_1, a) \mathcal{R} \delta(x_2, a)$.
  \end{itemize}
\end{definition}

We refine the notion of DFA morphism so that it also preserves the
simulation relation.

\begin{definition}
A morphism between DFAs with simulation $f : (X, \tau, \delta,
\mathcal{R}) \to (Y, \tau', \delta', \mathcal{S})$ is a function $f :
X \to Y$ such that it is a DFA morphism and if $x_1 \mathcal{R} x_2$,
then $f(x_1) \mathcal{S} f(x_2)$.
\end{definition}

These can be naturally organized as a category, which we denote
$\cat{SimDFA}$.

\begin{theorem}
  The forgetful functor $U : \cat{SimDFA} \to \cat{DFA}$ is a fibration.
\end{theorem}
\begin{proof}
  Since this functor is faithful, proving the universal property of
  fibrations is simpler. Consider the following lifting problem:

  We define the DFA with relation $(X, \mathcal{S}^*)$, where
  $\mathcal{S} = \{ (x_1, x_2) \mid f(x_1) \mathcal{S} f(x_2)\}$.
  We now prove that $\mathcal{S}^*$ is a simulation relation.
  \begin{description}
  \item[Preorder:] It follows directly from the fact that $\mathcal{S}$
    is a preorder.
  \item[Preserves acceptance:] Assume that $f(x_1) \mathcal{S} f(x_2)$
    and that $x_1$ is an accepting state. Since $f$ is a DFA morphism,
    it follows that $f(x)$ is an accepting state if, and only if, $x$
    is accepting. Since by assumption $\mathcal{S}$ is a simulation
    relation, we can conclude that $f(x_2)$ is an accepting state and
    so is $x_2$.
  \item[Stable under transition:] Assume that $f(x_1) \mathcal{S} f(x_2)$,
    $\delta_a(x_1) = x'_1$ and $\delta_a(x_2) = x'_2$. We want to
    show that $f(x'_1) \mathcal{S} f(x'_2)$. By assumption that $f$ is
    a DFA morphism, we have $f \circ \delta_a = \delta'_a \circ f$,
    for every character $a : \Sigma$. Therefore, we have to prove that
    $\delta'_a(f(x_1)) \mathcal{S} \delta'_a(f(x_2))$. This follows
    by the assumption that $\mathcal{S}$ is a simulation relation.
  \end{description}

  Therefore $(X, \mathcal{S}^*)$ is indeed a lifting of $X$.
  Showing that it is the Cartesian lifting follows by unfolding
  the definitions and using the fact that $U$ is faithful.
\end{proof}
\end{comment}

\section{Proposed construction for free Kleene categories}

For now, we ignore the subtleties behind pseudomonads, 2-monads,
lax algebras, etc.

Let $\cat{C}$ be a category. It is more or less straightforward to
define the free Kleene category $\FKC(\cat{C})$. We can do that by
using the internal language of Kleene categories. Concretely, it
will be a type theory with (non-commutative) tensors, sum types and
a primitive type of lists $L(A)$ for every $A$. By type theoretic reasons, this
structure forces the coproduct to distribute over the tensor, where we
can write the morphism $A \otimes (B + C) \to (A \otimes B) + (A
\otimes C) $ as the term
\begin{align*}
  &z \vdash \mathsf{let}\, x \otimes y = z\, \mathsf{in}\\
  &\mathsf{case}\, y \, \mathsf{with}\\
  &| \mathsf{in}_1\, y_1 \Rightarrow \mathsf{in}_1\, (x \otimes y_1)\\
  &| \mathsf{in}_2\,y_2\Rightarrow \mathsf{in}_2 \,(x \otimes y_2)\\
  &\mathsf{end}
\end{align*}
%
The list type includes primitive terms for $\textrm{cons}$,
$\textrm{nil}$, $\textrm{foldr}$ and $\textrm{foldl}$ where
$\textrm{foldr}$ and $\textrm{foldl}$ allow for a context to be used
in the construction of the nil case, but no free variables can be used
in the cons case:
\begin{mathpar}
  \inferrule
  {\Gamma \vdash L : A^* \and \Delta \vdash N : B \and a:A,b:B \vdash M : B
  }
  {\Gamma,\Delta \vdash \textrm{foldr}(L,N,a.b.M) : B}
\end{mathpar}

We can prove directly by the equational theory of this type theory
that this is an inverse to the canonical map going in the other
direction. The inductive nature of type theories should make the
following theorem pretty straightforward to prove:
%
\begin{theorem}
  The forgetful functor $\cat{KC} \to \cat{Cat}$ has a left adjoint
  in $\cat{2Cat}$.
\end{theorem}

\section{Revisiting Brzozowski's Distributive Law}

It has been observed that Brzozowski's derivative can be more conceptually
explained as a distributive law between the Kleene algebra monad and the
DFA functor $2 \times (\Sigma \Rightarrow -)$.

\begin{definition}
  We can equip the two element set $2 = \{0, 1 \}$ with the following KA structure
  \begin{itemize}
    \item $b_1 + b_2 = b_1 \lor b_2$
    \item $0 = 0$
    \item $b_1 b_2 = b_1 \land b_2$
    \item $1 = 1$
    \item $b^* = 1$
  \end{itemize}
\end{definition}

Since the axiomatization of Kleene algebras over an alphabet $\Sigma$
is a Horn theory, the category of Kleene algebras and Kleene algebra
morphisms is monadic over $\Set$. Let $E_\Sigma : \Set \to \Set$. be the 
Kleene algebra monad. It is possible to show the following theorem.

\begin{theorem}
  There is a distributive law $E_\Sigma(2 \times (\Sigma \Rightarrow X)) \to 2 \times (\Sigma \Rightarrow E_\Sigma(X))$.
\end{theorem}
\begin{proof}
  The component $E_\Sigma(2 \times (\Sigma \Rightarrow X)) \to 2$ is
  defined using the universal property of free algebraic structures,
  i.e. it suffices to define a morphism $2 \times (\Sigma \Rightarrow
  X)) \to 2$, which we choose it to be the first projection
  $\pi_1$. The component $E_\Sigma(2 \times (\Sigma \Rightarrow X))
  \to (\Sigma \Rightarrow E_\Sigma(X))$ requires us to be more
  clever. There is a ``non-standard'' Kleene algebra structure on
  $(\Sigma \Rightarrow E_\Sigma(X)) \times E_\Sigma(X)$ that makes
  this possible.
\end{proof}

\section{Generalization}

We now want to prove that the distributive law above generalizes to
the Kleene category case. We begin by showing that the ``non-standard''
structure defined in the previous section can be extended to the
proof-relevant setting.

\begin{lemma}
  Let $\cat{K}$ and $\cat{L}$ be two Kleene categories, $F : \cat{L}
  \to \cat{K}$ and $G : \cat{L} \to \cat{K}$ two Kleene category
  morphisms. We can equip $\cat{K} \times \cat{L}$ with the following
  Kleene category structure:
  \begin{itemize}
  \item $(A, X) + (B, Y) = (A + B, X + Y)$
  \item $0 = (0, 0)$
  \item $(A, X) \otimes (B, Y) = (A \otimes F(Y) + G(X) \otimes B, X \otimes Y)$
  \item $(A, X)^* = (A \otimes F(X^*), X^*)$
  \end{itemize}
\end{lemma}
%% \begin{proof}
%%   Since the morphism structure is defined componentwise, it is
%%   immediate to show that the proposed cocartesian structure is
%%   valid. For the monoidal structure, the bifunctoriality of $\otimes$
%%   follows from the bifunctoriality of the monoidal structures
%%   $\otimes$ and $+$ in $\cat{K}$. Next, we define the coherence
%%   isomorphisms. As suggested by the action on the $\cat{L}$ component,
%%   such isomorphisms are the same as in $\cat{L}$. The $\cat{K}$
%%   components of these morphisms are more interesting:
%% \end{proof}

With this lemma and the 2-monadicity of Kleene categories over $\cat{Cat}$,
we can define the distributive law between the proof-relevant
DFA functor and the Kleene category 2-monad.

\paragraph{Possible motivations}

People have considered derivative-based parsing before, this would provide
a semantic counterpart to those implementations.

This should imply that the algebraic and coalgebraic semantics of regular grammars coincides?

\section{Canonicity for Free Kleene Categories}

Let $\Sigma$ be a set. Then the ``standard semantics'' of the free
Kleene category $\FKC(\Sigma)$ is the unique functor $\sem{\cdot} :
\FKC(\Sigma) \to \Set^{\Sigma^*}$ preserving the (Day convolution)
Kleene category structure.

We first address whether we can generalize Kozen's completeness
theorems for Kleene algebras. Firstly, the fact that the Free Kleene
category is complete for \emph{language} inclusion and \emph{weak}
equivalence follows immediately from the fact that the free Kleene
algebra is equivalent to the posetal reflection of the free Kleene
category and the completeness theorem for Kleene algebras. That is,
for any objects $A, B \in \FKC(\Sigma)$, if $|\sem{A}| \subseteq
|\sem{B}|$ then there is a morphism $A \to B$ in $\FKC(\Sigma)$.

The immediate generalization to strong equivalence is false. For
example, we might ask that $\sem{\cdot}$ is a \emph{full} functor,
that any semantic reduction of the formal grammars is the denotation
of a morphism in the free Kleene category. But consider that the
regular grammar $1^*$ denotes a discrete grammar of natural numbers
and so morphisms $1^* \otimes \cdots \to 1^*$ are mapped into a set of
multi-argument functions on natural numbers. So completeness would
imply that all functions on natural numbers can be defined in the
syntax of the free Kleene category, which fails as all such functions
should be primitive recursive\footnote{flesh out the
details?}.

Failing completeness for definability of reductions, we could ask that
the \emph{equational theory} of free Kleene categories is
complete. This means that $\sem{\cdot}$ is a \emph{faithful} functor,
i.e., if two syntactic morphisms denote the same reduction $\sem{M} =
\sem{N}$, then they are provably equal $M = N$. I don't know if this
is true yet, but it seems unlikely to me.

However there is a (very) restricted form of completeness that is
true. The denotation $\sem{\cdot}$ is fully faithful when restricting
to hom sets of the form $\FKC(\Sigma)(\lceil w \rceil, A)$. The
significance of this is that the denotation of such a morphism is
equivalent to $\sem{A}\,w$, the set of ``parse trees'' for the grammar
$A$. Then fullness says that the term syntax is at least complete
enough to describe all parse trees, and faithfulness says that the
equational theory is at least complete enough to prove that terms that
represent the same parse tree are provably equal. In other words, for
every parse tree, there is a unique equivalence class of terms that
map to it. This theorem is called canonicity because it says that all
terms of a certain type are provably equal to a term in a
``canonical'' form.

\begin{theorem}{Canonicity}
  The denotation $\sem{\cdot}$ is bijective when restricting to hom
  sets of the form $\FKC(\Sigma)(\lceil w \rceil, A)$.
\end{theorem}

To prove this theorem, we first change perspective. The function
$\lceil \cdot \rceil : \Sigma^* \to \FKC(\Sigma)$ induces a restricted
Yoneda embedding called the ``nerve'' functor $N : \FKC(\Sigma) \to
\Set^{\Sigma^*}$ defined by
\[ N(A)w = \FKC(\Sigma)(\lceil w \rceil, A) \]
The action of $\sem{\cdot}$ on the restricted hom sets then induces a
natural transformation
\[ u : N \Rightarrow \sem{\cdot} \]
defined as
\[ u_w(f : \FKC(\Sigma)(\lceil w \rceil , A)) = \sem{f}(\iota_w) : A w \]
where $\iota_w : \sem{\lceil w \rceil} w$ is the unique parse tree
witnessing that the grammar $Y c_1 \otimes \cdots \otimes \varepsilon$
matches the string $c_1\cdots$. This $\iota_w$ is a universal element
of $\sem{\lceil w \rceil}$, meaning that evaluating at $\iota_w$ is a
bijection between $\Set^{\Sigma^*}(\sem{\lceil w \rceil}, A)$ and $A w$. So we
have that $\sem{\cdot}$ is fully faithful on these hom sets if and
only if $u$ is a natural isomorphism.

To do this we will explicitly construct an inverse natural
transformation $q : \sem{\cdot} \to N$ which we call ``quote'' because it quotes
a semantic parse tree back into the syntax. $q$ can be defined
explicitly by recursion on syntactic types and naturality proven by
induction over syntactic terms but we will develop a more structured
proof using the universal property of the free Kleene category.  To do
this we need to develop a Kleene category for which constructing a
functor from $\FKC(\Sigma)$ into it is equivalent to defining such a
natural transformation. We can do this by using the following
properties of \emph{comma categories}.

\begin{lemma}
  Let $F : \cat C \to \cat E$ and $G : \cat D \to \cat E$ and let $F
  \downarrow G$ be the comma category (i.e., whose objects are
  morphisms $F c \to G d$) with projections $p_C,p_D$.
  \begin{enumerate}
  \item If $\cat C, \cat D, \cat E$ have a monoidal structure for
    which $F$ is oplax and $G$ is lax monoidal, then $F \downarrow G$
    can be equipped with a monoidal structure such that $p_C,p_D$ are
    strict monoidal.
  \item If $\cat C, \cat D$ have (specified) colimits of shape
    $\mathcal I$ and $F$ preserves colimits of shape $\mathcal I$ then
    $F \downarrow G$ has colimits of shape $\mathcal I$ and the
    projections (strictly) preserve them.
  \end{enumerate}
\end{lemma}

Then we can prove the following
\begin{theorem}{Gluing for Kleene categories}
  Let $\cat K, \cat L, \cat M$ be Kleene categories and $\Phi : \cat K
  \to \cat M$ be a Kleene functor and $G : \cat L \to \cat M$ be a lax
  monoidal functor. Then $\Phi \downarrow G$ is a Kleene category, and
  the projections are (strict) Kleene functors.
\end{theorem}
\begin{proof}
  The monoidal and coproduct aspects of Kleene categories are handled
  by the previous lemma. We need to show that the monoidal structure
  is distributive and construct the Kleene star.

  For the Kleene star object, given $f :
  \Phi(k) \Rightarrow G l$ we define the carrier as
  \[ \Phi(k^*) \cong \Phi(k)^* \to (G l)^* \to G l^* \]
  where $f^* : \Phi(k)^* \to (G l)^*$ and $g : (G l)^* \to G l^*$ is
  defined by the induced cons-structure $G l \otimes G l^* \to G (l
  \otimes l^*) \to G l^*$ and $I \to G I \to G l^*$. Then it is
  routine to check that the nil, cons and fold all lift to commutative
  squares.

  TODO: address the snoc-part as well, and distributivity
\end{proof}

Then $\sem {\cdot} : \FKC(\Sigma) \to \Set^{\Sigma^*}$ is a strict
Kleene functor and $N : \FKC(\Sigma) \to \Set^{\Sigma^*}$ is a lax
monoidal functor, and so $\sem{\cdot}\downarrow N$ is a Kleene
category and the projection to $\FKC(\Sigma)^2$ is a strict Kleene
functor. Then by the universal property of $\FKC(\Sigma)$, there is a
unique strict Kleene functor $q$ from $\FKC(\Sigma)$ to
$\sem{\cdot}\downarrow N$ and further the composition with the
projection to $\FKC(\Sigma)^2$ is strict Kleene and so $q$ is
equivalent to a natural transformation $\sem{\cdot} \Rightarrow N$.

\begin{lemma}
  $q \circ u$ is the identity
\end{lemma}
\begin{proof}
  Let $f : \FKC(\Sigma)(\lceil w \rceil, A)$, then by naturality,
  $q(\sem{f}(\iota)) = f \circ q(\sem{\id}(\iota))$ so it suffices to
  show that $q(\iota) = \id$. This can be confirmed by calculation by
  induction on the length of $w$.
\end{proof}

\begin{lemma}
  $u \circ q$ is the identity
\end{lemma}
\begin{proof}
  By the universal property of $\FKC(\Sigma)$, it suffices to show
  that $u \circ q$ and $\id$ correspond to (strict?) Kleene functors
  from $\FKC(\Sigma)$ to $\sem{\cdot} \downarrow \sem{\cdot}$. The
  identity part should be a general theorem, and verifying the Kleene
  category properties is done by cases.
\end{proof}

\end{document}
