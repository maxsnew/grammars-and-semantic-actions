\documentclass[acmsmall,anonymous,review,screen]{acmart}
\usepackage{mathpartir}
\usepackage{quiver}
\usepackage{enumitem}
\usepackage{wrapfig}
\usepackage{fancyvrb}
\usepackage{amsmath}

\DeclareFontFamily{U}{dmjhira}{}
\DeclareFontShape{U}{dmjhira}{m}{n}{ <-> dmjhira }{}

\DeclareRobustCommand{\yo}{\text{\usefont{U}{dmjhira}{m}{n}\symbol{"48}}}

\usepackage{xstring} % Need it for strlen
\newcommand{\cat}[1]{
  \relax
  \StrLen{#1}[\arglength]
  \ifnum\arglength=1
  \mathcal{#1}
  \else
  \mathbf{#1}
  \fi
}

\newcommand{\Set}{\cat{Set}}
\newcommand{\lto}{\multimap}
\newcommand{\grammar}{\Set^{\cat{\Sigma^*}}}

%% Rights management information.  This information is sent to you
%% when you complete the rights form.  These commands have SAMPLE
%% values in them; it is your responsibility as an author to replace
%% the commands and values with those provided to you when you
%% complete the rights form.
\setcopyright{acmcopyright}
\copyrightyear{2018}
\acmYear{2018}
\acmDOI{XXXXXXX.XXXXXXX}

%% These commands are for a PROCEEDINGS abstract or paper.
\acmConference[Woodstock '18]{Woodstock '18: ACM Symposium on Neural
  Gaze Detection}{June 03--05, 2018}{Woodstock, NY}
\acmBooktitle{Woodstock '18: ACM Symposium on Neural Gaze Detection,
  June 03--05, 2018, Woodstock, NY}
\acmPrice{15.00}
\acmISBN{978-1-4503-XXXX-X/18/06}


\begin{document}
\title{Kleene Categories}
\author{Steven Schaefer}
\affiliation{
  \department{Electrical Engineering and Computer Science}
  \institution{University of Michigan}
  \country{USA}
}
\email{stschaef@umich.edu}

\author{Max S. New}
\affiliation{
  \department{Electrical Engineering and Computer Science}
  \institution{University of Michigan}
  \country{USA}
}
\email{maxsnew@umich.edu}

\author{Pedro H. Azevedo de Amorim}
\affiliation{
  \department{Department of Computer Science}
  \institution{University of Oxford}
  \country{UK}
}
\email{pedro.azevedo.de.amorim@cs.ox.ac.uk}

\begin{abstract}
  We propose a categorification of Kleene algebras and study some of its basic properties.
\end{abstract}

\maketitle


\section{Kleene Category}

Kleene algebras are an important tool in the theory of regular
languages. More broadly, they serve as a theoretical substrate to
studying various kinds of formal languages. Formally, they are a tuple
$(A, +, \cdot, (-)^*, 1, 0)$, where $A$ is a set, $+$ and $\cdot$
are binary operations over $A$, $(-)^*$ is a function over $A$, and
$1$ and $0$ are constants. These structures satisfy the axoims depicted
in Figure~\ref{fig:axioms}.

\begin{figure}
  \begin{align*}
    x + (y + z) &= (x + y) + z & x + y &= y + x\\
    x + 0 &= x & x + x &= x\\
    x(yz) &= (xy)z & x1 &= 1x = x\\
    x(y + z) &= xy + xz & (x + y)z &= xz + yz\\
    x0 &= 0x = x & & \\
    1 + aa^* &\leq a^* & 1 + a^*a &\leq a^*\\
     b + ax \leq x &\implies a^*b \leq x &  b + xa \leq x &\implies ba^* \leq x
  \end{align*}
  \label{fig:axioms}
  \caption{Kleene algebra axioms}
\end{figure}

The addition operation can be used to define the partial order
structure $a \leq b$ if $a + b = b$. In the theory of formal languages
this order structure can be used to model language containment. In this
section, we want to categorify the concept of Kleene algebra and
build on top of it in order to define an abstract theory of parsing.
We start by defining \emph{Kleene categories}.

\begin{definition}
  A Kleene category is a distributive monoidal category $\cat{K}$
  such that for every objects $A$ and $B$, the endofunctors $F_{A, B}
  = B + A \otimes X$ and $G_{A, B} = B + X \otimes A$ have initial
  algebras (denoted $\mu X.\, F_{A, B}(X)$) such that the canonical
  morphisms $\mu X.\, F_{A, B}(X) \to B \otimes (\mu   X.\, F_{A, 1})$
  $\mu X.\, G_{A, B}(X) \to (\mu   X.\, G_{A, 1})\otimes B $ are iso.
\end{definition}

As a sanity check, note that Kleene algebras are indeed examples of
Kleene categories.

\begin{example}
  Every Kleene algebra, seen a posetal category, is a Kleene category.
\end{example}

An unexpected example comes from the theory of substructural logics.

\begin{example}
  The opposite category of every Kleene category is a model of a variant of
  conjunctive ordered logic, where the Kleene star plays the role of the ``of
  course'' modality from substructural logics.
\end{example}

This proposed definition is sensible. After all, the proposed axioms
are a direct translation of the Kleene algebra axioms to a categorical
setting. Its most convoluted aspect is the axiomatization of the Kleene
star as a family os initial algebras satisfying certain
isomorphisms. If the Kleene category $\cat{K}$ has more structure,
then these isomorphisms hold ``for free''.

\begin{theorem}
  \label{th:kleeneclosed}
  Let $\cat{K}$ be a Kleene category such that it is also monoidal
  closed.  Then, the initial algebras isomorphisms hold automatically.
\end{theorem}
\begin{proof}
  We prove this by the unicity (up-to isomorphism) of initial
  algebras. Let $[hd, tl]: 1 + (\mu X.\, F_{A, 1}(X)) \otimes A \to
  (\mu X.\, F_{A, 1}(X))$ be the initial algebra structure of $(\mu
  X.\, F_{A, 1}(X))$ and consider the map $[hd, tl] : B + B \otimes
  (\mu X.\, F_{A, 1}(X)) \otimes A \to B\otimes (\mu X.\, F_{A,
    1}(X))$.

  Now, let $[f,g] : B + A \otimes Y \to Y$ be an $F_{A,B}$-algebra and
  we want to show that there is a unique algebra morphism $h : B
  \otimes \mu X.\, F_{A,1} \to Y$. We can show existence and uniqueness
  by showing that the diagram on the left commute if, and only if,
  the diagram on the right commutes:

  This equivalence follows by using the adjunction structure given
  by the monoidal closed structure of $\cat{K}$. A completely analogous
  argument for $G_{A,B}$ also holds.
\end{proof}

This result feels similar in spirit to the definition of action
algebras, which are algebras where the product also has adjoint
operations which results in the Kleene star being more easily
axiomatized \cite{kozen1994}. We are now ready to prove that our
concept of formal grammars fit nicely within our categorical
framework. We start by presenting a well-known construction
from presheaf categories.

\begin{definition}
  Let $\cat{C}$ be a locally small monoidal category and $F$, $G$ be
  two functors $\cat{C} \to \Set$. Their Day convolution tensor
  product is defined as the following coend formula:
  \[
  (F \otimes_{Day} G)(x) = \int^{(y,z) \in \cat{C}\times\cat{C}}\cat{C}(y\otimes z, x) \times F(y) \times G(z) 
  \]
  Dually, its internal hom is given by the following end formula:
  \[
  (F \lto_{Day} G)(x) = \int_{y} \Set(F(y), G(x \otimes y))
  \]
\end{definition}

\begin{lemma}[\cite{day1970}]
  Under the assumptions above, the presheaf category $\Set^{\cat{C}}$ is
  monoidal closed.
\end{lemma}

%% \begin{theorem}
%%   Let $\cat{K}$ be a Kleene category and $A$ a discrete category.
%%   The functor category $[A, \cat{K}]$.
%%   (HOW GENERAL SHOULD THIS THEOREM BE? BY ASSUMING ENOUGH STRUCTURE,
%%   E.G. K = Set, THIS THEOREM BECOMES SIMPLE TO PROVE)
%% \end{theorem}
\begin{theorem}
  If $\cat{C}$ is a locally small monoidal category, then
  $\Set^{\cat{C}}$ is a Kleene category.
\end{theorem}
\begin{proof}

  By the lemma above, $\Set^{\cat{C}}$ is monoidal closed, and since it
  is a presheaf category, it has coproducts, and since the tensor
  is a left adjoint, it preserves colimits and, therefore, it is
  a distributive category.

  As for the Kleene star, since presheaf categories admit small colimits,
  the initial algebra of the functors $F_{A,B}$ and $G_{A,B}$ can be
  defined as the filtered colimit of the diagrams:

  From Theorem~\ref{th:kleeneclosed} it follows that these initial
  algebras satisfy the required isomorphisms and this concludes the
  proof.
\end{proof}

\begin{corollary}
  For every alphabet $\Sigma$, the presheaf category $\grammar$
  is a Kleene category.
\end{corollary}
\begin{proof}
  Note that string concatenation and the empty string make the
  discrete category $\Sigma^*$ a strict monoidal category.
\end{proof}

\begin{definition}
  Let $K$ be a Kleene category over alphabet $\Sigma$. This
  induces a strong monoidal functor $\Sigma^* \to K$. The
  nerve functor $w \mapsto K(w, -)$ is the grammar interpretation
  functor of $K$.
\end{definition}

\section{Proof-Relevant Deterministic Automata}

Now, we will consider a categorification of deterministic automata.

\begin{definition}
  A proof-relevant DFA is an $F$-coalgebra, for the functor
  $F(\cat{C}) = \Set \times (\Sigma \Rightarrow \cat{C})$.
\end{definition}

Assuming that the underlying category $\cat{C}$ is not discrete,
arrows $f : X \to Y$ in $\cat{C}$ should be thought of as a proof that
the state $X$ is related to the $Y$. This can be seen as a
generalization of simulation relations. After all, functoriality of $F
: \cat{C} \times \Sigma \to \cat{C}$ implies that $F(f, id_c): F(X, c)
\to F(Y, c)$, i.e. after one transition, the end states will also be
related.

\begin{definition}
  We can equip $\grammar$ with the $F$-coalgebra structure $\langle \alpha_1, \alpha_2\rangle$,
  where $\alpha_1(F) = F(\varepsilon)$ and $\alpha_2(F, c) = \yo(c) \lto F$.
\end{definition}

\begin{theorem}
  The morphism $\alpha : \grammar \to F(\grammar)$ is the final
  $F$-coalgebra.
\end{theorem}
\begin{proof}
  Given an $F$-coalgebra $\langle \alpha, \delta\rangle : \cat{C} \to \Set \times \cat{C}^\Sigma$, we can define the functor $\cat{C} \times \Sigma^* \to \Set$ much like in the DFA case:
  by induction on $\Sigma^*$.
  
\end{proof}


\begin{comment}
\subsection{DFAs with simulation relation}

Let $\langle \tau, \delta\rangle : X \to 2 \times X^\Sigma$ be a DFA
over an alphabet $\Sigma$.

\begin{definition}
  A binary relation $\mathcal{R} \subseteq X \times X$ is a simulation if
  \begin{itemize}
  \item It is a preorder.
  \item If $x_1 \mathcal{R} x_2$ and $x_1$ is an accept state, so is $x_2$.
  \item If $x_1 \mathcal{R} x_2$ then for every $a : \Sigma$,
    $\delta(x_1, a) \mathcal{R} \delta(x_2, a)$.
  \end{itemize}
\end{definition}

We refine the notion of DFA morphism so that it also preserves the
simulation relation.

\begin{definition}
A morphism between DFAs with simulation $f : (X, \tau, \delta,
\mathcal{R}) \to (Y, \tau', \delta', \mathcal{S})$ is a function $f :
X \to Y$ such that it is a DFA morphism and if $x_1 \mathcal{R} x_2$,
then $f(x_1) \mathcal{S} f(x_2)$.
\end{definition}

These can be naturally organized as a category, which we denote
$\cat{SimDFA}$.

\begin{theorem}
  The forgetful functor $U : \cat{SimDFA} \to \cat{DFA}$ is a fibration.
\end{theorem}
\begin{proof}
  Since this functor is faithful, proving the universal property of
  fibrations is simpler. Consider the following lifting problem:

  We define the DFA with relation $(X, \mathcal{S}^*)$, where
  $\mathcal{S} = \{ (x_1, x_2) \mid f(x_1) \mathcal{S} f(x_2)\}$.
  We now prove that $\mathcal{S}^*$ is a simulation relation.
  \begin{description}
  \item[Preorder:] It follows directly from the fact that $\mathcal{S}$
    is a preorder.
  \item[Preserves acceptance:] Assume that $f(x_1) \mathcal{S} f(x_2)$
    and that $x_1$ is an accepting state. Since $f$ is a DFA morphism,
    it follows that $f(x)$ is an accepting state if, and only if, $x$
    is accepting. Since by assumption $\mathcal{S}$ is a simulation
    relation, we can conclude that $f(x_2)$ is an accepting state and
    so is $x_2$.
  \item[Stable under transition:] Assume that $f(x_1) \mathcal{S} f(x_2)$,
    $\delta_a(x_1) = x'_1$ and $\delta_a(x_2) = x'_2$. We want to
    show that $f(x'_1) \mathcal{S} f(x'_2)$. By assumption that $f$ is
    a DFA morphism, we have $f \circ \delta_a = \delta'_a \circ f$,
    for every character $a : \Sigma$. Therefore, we have to prove that
    $\delta'_a(f(x_1)) \mathcal{S} \delta'_a(f(x_2))$. This follows
    by the assumption that $\mathcal{S}$ is a simulation relation.
  \end{description}

  Therefore $(X, \mathcal{S}^*)$ is indeed a lifting of $X$.
  Showing that it is the Cartesian lifting follows by unfolding
  the definitions and using the fact that $U$ is faithful.
\end{proof}
\end{comment}


\section{Proposed construction for free Kleene categories}

For now, we ignore the subtleties behind pseudomonads, 2-monads,
lax algebras, etc. We begin by defining pre Kleene categories.

\begin{definition}
  A pre Kleene category is a distributive monoidal category such that
  the functors $F_{A, B}$ and $G_{A, B}$ have initial algebras, but
  they do not necessarily satisfy the isomorphism condition.
\end{definition}

These objects can pretty naturally be organized as a 2-category
$\cat{PKC}$: 1-cells are functors that strictly preserves all of the
structure while 2-cells are just natural transformations.

Let $\cat{C}$ be a category. It is more or less straightforward to
define free pre Kleene category $PKC(\cat{C})$. We can do that by
using the internal language of pre Kleene categories. Concretely, it
will be a type theory with (non-commutative) tensors, sum types and
inductive types corresponding to the initial algebras for $F_{A, B}$
and $G_{A, B}$, for every $A$, $B$. By type theoretic reasons, this
structure forces the coproduct to distribute over the tensor, where we
can write the morphism $A \otimes (B + C) \to (A \otimes B) + (A
\otimes C) $ as the term
\begin{align*}
  &z \vdash \mathsf{let}\, x \otimes y = z\, \mathsf{in}\\
  &\mathsf{case}\, y \, \mathsf{with}\\
  &| \mathsf{in}_1\, y_1 \Rightarrow \mathsf{in}_1\, (x \otimes y_1)\\
  &| \mathsf{in}_2\,y_2\Rightarrow \mathsf{in}_2 \,(x \otimes y_2)\\
  &\mathsf{end}
\end{align*}
%
We can prove directly by the equational theory of this type theory
that this is an inverse to the canonical map going in the other
direction. The inductive nature of type theories should make the
following theorem pretty straightforward to prove:
%
\begin{theorem}
  The forgetful functor $\cat{PKC} \to \cat{Cat}$ has a left adjoint
  in $\cat{2Cat}$.
\end{theorem}

Unfortunately, we can see that such a free construction will not be a
Kleene category, since there is no way of constructing the Kleene star
isomorphisms. That being said, there is a comparison map
$\mu F_{A,B} \to A \mu F_{1, B}$ for $F$ and $G$. This leads me to conjecture
the following:
%
\begin{conjecture}
  The forgetful functor $\cat{KC} \to \cat{PKC}$ is a reflective
  subcategory
\end{conjecture}

If the theorem above is true, we get a Kleene category
monad on $\cat{Cat}$ by composing the adjunctions above.
While I'm not sure how to define the reflection, Theorem~\ref{th:kleeneclosed}
might be important in our construction. I remember asking Daniel Gratzer
about this and he told me something about mapping a pre Kleene
category $\cat{C}$ to $\widehat{\cat{C}}$ and ``carving it out'' in order
to obtain an appropriate subcategory. Note that $\widehat{C}$ is monoidal
closed, so it will be Kleene as well.


\section{Revisiting Brzozowski's Distributive Law}

It has been observed that Brzozowski's derivative can be more conceptually
explained as a distributive law between the Kleene algebra monad and the
DFA functor $2 \times (\Sigma \Rightarrow -)$.

\begin{definition}
  We can equip the two element set $2 = \{0, 1 \}$ with the following KA structure
  \begin{itemize}
    \item $b_1 + b_2 = b_1 \lor b_2$
    \item $0 = 0$
    \item $b_1 b_2 = b_1 \land b_2$
    \item $1 = 1$
    \item $b^* = 1$
  \end{itemize}
\end{definition}

Since the axiomatization of Kleene algebras over an alphabet $\Sigma$
is a Horn theory, the category of Kleene algebras and Kleene algebra
morphisms is monadic over $\Set$. Let $E_\Sigma : \Set \to \Set$. be the 
Kleene algebra monad. It is possible to show the following theorem.

\begin{theorem}
  There is a distributive law $E_\Sigma(2 \times (\Sigma \Rightarrow X)) \to 2 \times (\Sigma \Rightarrow E_\Sigma(X))$.
\end{theorem}
\begin{proof}
  The component $E_\Sigma(2 \times (\Sigma \Rightarrow X)) \to 2$ is
  defined using the universal property of free algebraic structures,
  i.e. it suffices to define a morphism $2 \times (\Sigma \Rightarrow
  X)) \to 2$, which we choose it to be the first projection
  $\pi_1$. The component $E_\Sigma(2 \times (\Sigma \Rightarrow X))
  \to (\Sigma \Rightarrow E_\Sigma(X))$ requires us to be more
  clever. There is a ``non-standard'' Kleene algebra structure on
  $(\Sigma \Rightarrow E_\Sigma(X)) \times E_\Sigma(X)$ that makes
  this possible.
\end{proof}

\section{Generalization}

We now want to prove that the distributive law above generalizes to
the Kleene category case. We begin by showing that the ``non-standard''
structure defined in the previous section can be extended to the
proof-relevant setting.

\begin{lemma}
  Let $\cat{K}$ and $\cat{L}$ be two Kleene categories, $F : \cat{L}
  \to \cat{K}$ and $G : \cat{L} \to \cat{K}$ two Kleene category
  morphisms. We can equip $\cat{K} \times \cat{L}$ with the following
  Kleene category structure:
  \begin{itemize}
  \item $(A, X) + (B, Y) = (A + B, X + Y)$
  \item $0 = (0, 0)$
  \item $(A, X) \otimes (B, Y) = (A \otimes F(Y) + G(X) \otimes B, X \otimes Y)$
  \item $(A, X)^* = (A \otimes F(X^*), X^*)$
  \end{itemize}
\end{lemma}
%% \begin{proof}
%%   Since the morphism structure is defined componentwise, it is
%%   immediate to show that the proposed cocartesian structure is
%%   valid. For the monoidal structure, the bifunctoriality of $\otimes$
%%   follows from the bifunctoriality of the monoidal structures
%%   $\otimes$ and $+$ in $\cat{K}$. Next, we define the coherence
%%   isomorphisms. As suggested by the action on the $\cat{L}$ component,
%%   such isomorphisms are the same as in $\cat{L}$. The $\cat{K}$
%%   components of these morphisms are more interesting:
%% \end{proof}

With this lemma and the 2-monadicity of Kleene categories over $\cat{Cat}$,
we can define the distributive law between the proof-relevant
DFA functor and the Kleene category 2-monad.

\paragraph{Possible motivations}

People have considered derivative-based parsing before, this would provide
a semantic counterpart to those implementations.


\section{Gluing Kleene categories}

The following theorems should be true
\begin{theorem}
  Given Kleene categories $\cat{L}$, $\cat{K}$, $\cat{K}'$
  and functors $F : \cat{L} \to \cat{K}$, $p : \cat{K}' \to \cat{K}$
  such that $F$ is lax monoidal such that it laxly preserves the initial
  algebras and $p$ is a monoidal fibration strictly preserving the
  Kleene category structure, then in the pullback diagram
  % https://q.uiver.app/#q=WzAsNCxbMCwxLCJcXGNhdHtMfSJdLFsxLDEsIlxcY2F0e0t9Il0sWzEsMCwiXFxjYXR7S30nIl0sWzAsMCwiXFxjYXR7TH0nIl0sWzAsMSwiRiIsMl0sWzIsMSwicCJdLFszLDJdLFszLDBdLFszLDEsIiIsMSx7InN0eWxlIjp7Im5hbWUiOiJjb3JuZXIifX1dXQ==
\[\begin{tikzcd}
	{\cat{L}'} & {\cat{K}'} \\
	{\cat{L}} & {\cat{K}}
	\arrow[from=1-1, to=1-2]
	\arrow[from=1-1, to=2-1]
	\arrow["\lrcorner"{anchor=center, pos=0.125}, draw=none, from=1-1, to=2-2]
	\arrow["p", from=1-2, to=2-2]
	\arrow["F"', from=2-1, to=2-2]
\end{tikzcd}\]
$\cat{L}'$ is a Kleene category and the functor $\cat{L}'\to \cat{L}$ is a
monoidal fibration strictly preserving the Kleene category structure.
\end{theorem}
\begin{proof}
  It is known from linear logic reasons that the pullback is distributive monoidal.
  The Kleene star should be defined much like the exponential in glued models of
  linear logic. And the projection functor is a monoidal fibration because those
  are stable under pullback.
\end{proof}

Recently I have been enjoying comma categories a bit more, which leads me to the follwing
conjecture:

\begin{conjecture}
  The 2-category of Kleene categories has comma objects.
\end{conjecture}


\end{document}
