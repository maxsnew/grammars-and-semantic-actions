\documentclass[12pt]{article}
%AMS-TeX packages
\usepackage{tikz}
\usepackage[utf8]{inputenc}
\usepackage{amssymb,amsmath,amsthm,stmaryrd, wasysym} 
%geometry (sets margin) and other useful packages
\usepackage[margin=1.25in]{geometry}
\usepackage{graphicx,ctable,booktabs}
\usepackage{tikz-cd}
\usepackage{mathpartir}

\usepackage[sort&compress,square,comma,authoryear]{natbib}
\bibliographystyle{plainnat}

\newcommand{\leftmultimap}{\mathop{\rotatebox[origin=c]{180}{$\multimap$}}}
\newcommand{\Set}{\textrm{Set}}
\newcommand{\Lang}{\textrm{Lang}}
\newcommand{\Grammar}{\textrm{Grammar}}
\newcommand{\Action}{\textrm{Action}}
\newcommand{\sem}[1]{\llbracket{#1}\rrbracket}
\newtheorem{definition}{Definition}
\newtheorem{theorem}{Theorem}

\newcommand\rsepimp{\mathbin{-\mkern-6mu*}}
\newcommand\lsepimp{\mathbin{\rotatebox[origin=c]{180}{$-\mkern-6mu*$}}}
\newcommand\sepconj{\mathbin{*}}
\newcommand\bunch{\mathcal W}

\begin{document}
\title{Monoidal Toposes of Grammars and Semantic Actions}
\author{Max S. New}
\maketitle

\begin{abstract}
  We show that formal grammars (of unrestricted complexity) can be
  modeled as objects of a monoidal topos and algebraic operations on
  languages can be lifted to universal constructions in this
  category. Further, this can be extended to a monoidal topos of
  formal grammars equipped with a semantic action that produces a
  semantic value.

  Since the grammars are of unrestricted complexity, this topos can in
  addition to modeling regular and context-free languages, also
  constructs such as the \emph{Brzozowski derivative} of a language,
  and familiar facts about this derivative are derivable from
  manipulation of universal properties.
\end{abstract}

\section{Languages, Grammars and Semantic Actions}

A \emph{language} over an alphabet $\Sigma$ is a subset of strings
$\Sigma^*$. A \emph{grammar} $g$ is a syntactic object that presents a
language. There are many descriptions of formal grammars: regular
expressions, context-free grammars, context-sensitive etc. The problem
of \emph{language recognition} is to take a string $w \in \Sigma^*$
and a grammar $g$ and determine if $w$ is a member of the language
defined by $g$.

In programming applications such as compilation or communication over
a network, we are not just interested in the problem of language
\emph{recognition}, determining whether or not a word is in a
language, but in the problem of \emph{parsing}, the attempt to produce
a \emph{parse tree} from the string. We can think of the parsing as a
``constructive'' version of the language recognition problem: the
parse tree tells us not just \emph{whether} a word is a member of the
language, but also \emph{why}. A language is defined as a subset of
$\Sigma^*$, which is equivalent to a \emph{predicate} on $\Sigma^*$:
\[ L : \Sigma^* \to \Omega \]
where $\Omega$ is the set of \emph{truth values}\footnote{classically,
this is equivalent to any two element set, but constructively to
assume this set is boolean would require the language is
decidable.}. We can categorify this situation by replacing the
lattice of truth values $\Omega$ with the topos of sets $\Set$:
\[ g : \Sigma^* \to \Set \]
That is, a language is a \emph{functor} from $\Sigma^*$ (viewed as a
discrete category) to the category of sets and functions.
Functors from $\Sigma^*$ to $\Set$ are called \emph{presheaves} over
$\Sigma^*$ and the category of such functors is extremely
well-behaved. We show in the remainder of this note that this provides
us with a rich language for constructing languages algebraically.
\begin{quote}
  A grammar is a presheaf over the set of strings
\end{quote}
Normally a formal grammar is defined to be a syntactic object in some
formal system such as regular expressions or context-free grammars.
%
We see each of these notions of formal grammar as being a
\emph{presentation} of the semantic notion of grammar. In Lawvere's
terminology, we might call the typical formal grammars as
\emph{subjective} and our semantic notion an \emph{objective grammar}. 
%
These different classes of formal grammars can be described as
inductively generated subcategories closed under certain constructions
definable on all grammars.

Note that a presheaf $P : \Sigma^* \to \Set$ over a discrete category
like $\Sigma^*$ is equivalent to simply a set $\int P$ of ``parse
trees'' with a function $\pi : \int P \to \Sigma^*$ that shows what
string each parse tree is a parse of. This is a generalization of the
fact that a language can be thought of as a subset $L \subseteq
\Sigma^*$. We will use the presentation in terms of presheaves
throughout.

Returning to the parsing problem, the output of a parser usually isn't
the literal parse tree, but instead the result of some \emph{semantic
actions} run on the parse tree, which usually serve to remove some
syntactic details that are unnecessary for later processing.
%
Given a grammar in the guise of a presheaf $g$ over strings, we define
a semantic action to be a set $S$ of with a function that produces a
semantic element from any parse $a : \prod_{w \in \Sigma^*} g_w \to
S$. The semantic actions can also be arranged into a structured
category, as the category of semantic actions can be defined by the
following pullback, a construction in topos theory known as
\emph{Artin gluing}:

% https://q.uiver.app/?q=WzAsNCxbMCwxLCJcXFNldCJdLFsxLDEsIlxcR3JhbW1hciJdLFsxLDAsIlxcR3JhbW1hcl5cXHRvIl0sWzAsMCwiXFxBY3Rpb24iXSxbMywwXSxbMCwxXSxbMywyXSxbMiwxXSxbMywxLCIiLDEseyJzdHlsZSI6eyJuYW1lIjoiY29ybmVyIn19XV0=
\[\begin{tikzcd}
	\Action & {\Grammar^\to} \\
	\Set & \Grammar
	\arrow[from=1-1, to=2-1]
	\arrow[from=2-1, to=2-2]
	\arrow[from=1-1, to=1-2]
	\arrow[from=1-2, to=2-2]
	\arrow["\lrcorner"{anchor=center, pos=0.125}, draw=none, from=1-1, to=2-2]
\end{tikzcd}\]

The observations we present here are fairly mundane but we intend that
they should form the beginnings of a general theory of grammars that
can be extended to more structured forms such as type-checking.

\section{Grammar Combinators in Monoidal Toposes}

Next, we describe the structures present in the categories $\Grammar$
and $\Action$ described above and how they relate to familiar aspects
of formal grammars and parsing.

First, we should consider what the morphisms are.

\begin{definition}
  A language $L$ is included in another $L'$, written $L \subseteq L'$
  when every string in $L$ is in $L'$.
  
  A morphism of formal grammars $f : g \to g'$ is a natural
  transformation. More concretely, it is a family of functions
  \[ \prod_{w \in \Sigma^*} g_w \to g_w' \]

  A morphism of semantic actions $\phi : (a,g,X) \to (a', g', X')$ is
  a pair of a morphism of grammars $f : g \to g'$ and a function $s :
  X \to X'$ such that for every $w \in \Sigma^*$ and parse $p \in g_w$
  \[ s(a(p)) = a'(f_w(p)) \]
\end{definition}

Each of these notions is a refinement of the previous: a morphism of
grammars implies an inclusion of their underlying languages, and by
definition a morphism of semantic actions contains a morphism of the
underlying grammars. This is formalized in the fact that there are
functors

% https://q.uiver.app/?q=WzAsMyxbMSwwLCJcXEdyYW1tYXIiXSxbMiwwLCJcXExhbmciXSxbMCwwLCJcXEFjdGlvbiJdLFsyLDAsInxcXGNkb3R8Il0sWzAsMSwifHxcXGNkb3R8fCJdXQ==
\[\begin{tikzcd}
	\Action & \Grammar & \Lang
	\arrow["{|\cdot|}", from=1-1, to=1-2]
	\arrow["{||\cdot||}", from=1-2, to=1-3]
\end{tikzcd}\]

In fact these functors both have fully faithful right adjoints
exhibiting these as reflective subcategories. In this sense, the
theory of semantic actions subsumes the theory of grammars subsumes
the theory of languages.

Next, we can define the most basic possible grammars: the grammars
that accept exactly one string.
%
Since $\Grammar$ is a presheaf category, the \emph{Yoneda embedding}
defines for each string $w$, the representable presheaf $Y w$. Since
$\Sigma^*$ is a set, this takes a particularly simple form, there is a
single parse precisely when the input string is $w$:
\[ (Y w)_w' = \{ * | w = w' \} \]
%
This extends to a trivial semantic action.

Next, since $\Sigma^*$ is not just a set but a \emph{monoid} with
concatenation of strings\footnote{in fact the free monoid}, by the
\emph{Day convolution} structure we get that the presheaf category
$\Grammar$ has a \emph{monoidal} structure. The monoidal product is
the \emph{concatenation} of grammars:
\[ (g \sepconj g')w = \sum_{w_1w_2 = w} g_{w_1} \times g_{w_2}\]
In other words, a parse of the sequenced grammar $g \sepconj g'$
consists of a choice of splitting the input string into a prefix and a
suffix paired with a $g$-parse of the prefix and a $g'$ parse of the
suffix. This can be extended to a monoidal structure on semantic
actions where the semantic data is the product
\[ (a, g, X) \sepconj (a', g', X') = ((a\sepconj a')(p,p') = (a(p), a'(p'), g \sepconj g', X \times X'))\]
We use the separating conjunction symbol because the semantics
(presheaves over a monoidal category) are similar, but note that the
monoidal structure is \emph{not} symmetric in that $g \sepconj g'$ is
usually very different from $g' \sepconj g$.

This has a unit as well, which is the grammar with a unique parse of
the empty string (with trivial semantic action):
\[ I w = w = \epsilon \]

Since $\Sigma^*$ is a free monoid, every word $w$ is a concatenation
of finitely many characters $c \in \Sigma$. Given $w = c_1c_2\cdots$
we can now choose between two different representations:
\[ Y(c_1c_2\cdots) \qquad \textrm{or} \qquad Yc_1 \sepconj Yc_2 \sepconj \cdots\]
Since the Yoneda embedding preserves the monoidal structure up to
isomorphism, these are isomorphic as grammars and semantic actions as
well.

Next, $\Grammar$ has all limits and colimits. Binary coproducts give
us the ``sum'' of grammars, i.e., a parse is a parse in one of the two
languages. The initial object is the empty grammar, the grammar that
has no parses.
\[ (g_1 + g_2) w = (g_1 w) + (g_2 w) \]
\[ 0 w = \emptyset \]
The semantic actions ``lie over'' the corresponding coproduct in
$\Set$. These correspond to the union and empty languages in $\Lang$.

Products give us a kind of intersection of grammars, where a parse
tree is a parse from each language, with the terminal object being
the language that accepts all strings with a unique parse.
\[ (g_1 \times g_2) w =  g_1 w \times g_2 w\]
\[ 1 w =  1 \]
These lie over the product and unit semantic actions.

Inductively defined grammars should be initial algebras of functors $F
: \Grammar \to \Grammar$. To define regular and context free grammars
it should suffice to use Ad\'amek's construction generalizing Kleene's
fixed point theorem to categories. That is, if $F$ is
$\omega$-co-continuous, i.e., preserves $\omega$-colimits, then it has
an initial algebra, which can be constructed as the colimit of
\[ F^0(0) \to F^1(0) \to F^2(0) \to \cdots \]

Then, for example the Kleene star of a grammar $g$ is given by the
initial algebra of the functor $F(\gamma) = I + g \sepconj \gamma$,
which is $\omega$-co-continuous because $\sepconj$ and $+$ are
$\omega$-co-continuous.

So far the interesting structure has mostly been in the category of
grammars, and the corresponding sets of parses have been isomorphic to
the set of parses. The simple combinator that we need is a ``map'' for
semantic actions, that says given a semantic action $A$ that produces
$X$ values, and a function $f : X \to Y$, we can push forward and get
an action $f_*A$ that produces $Y$ values. If $A = (a, g, X)$ then we
can define this simply as
\[ f_*(a,g,X) = (f \circ a, g, Y) \]

\subsection{Closed Structure}

By general properties of presheaf categories, the category of
grammars has \emph{closed} structure with respect to the monoidal
products of sequencing $\sepconj$ and product $\times$.

First, the product has a right (multi-variable) adjoint:
\[ (g_1 \times g_2 \to  g_3) \cong g_1 \to (g_2 \Rightarrow g_3) \cong g_2 \to (g_1 \Rightarrow g_3) \]

Defined by
\[ (g \Rightarrow g')w = \textrm{Hom}((Y w \times g), g') = gw \to g'w \]
Informally, $(g \Rightarrow g')w$ is constructive \emph{implication}
of languages: a parse of $w$ is a function that takes $g$-parses of
$w$ to $g'$-parses of $w$

Next, the sequencing operator $\sepconj$, since it is not symmetric, has
two right multi-variable adjoints:
\[ (g_1 \sepconj g_2 \to g_3) \cong (g_1 \to g_3 \lsepimp g_2) \cong (g_2 \to g_2 \rsepimp g_3) \]

Defined by
\[ (g \rsepimp g')w = \Pi_{w_l} g w_l \to g'(w_lw) \]
\[ (g' \lsepimp g)w = \Pi_{w_r} g w_r \to g'(ww_r) \]

A parse of $(g \rsepimp g')w$ is a kind of ``partial'' parse of $g'$:
given any parse $g w_l$, we can get a complete parse $g' (w_lw)$. So
if $(g \rsepimp g') w$ successfully parses (and $g$ is non-empty),
then we have succesfully parsed a suffix of $g'$, that when
concatenated with a prefix that can by parsed as $g$ will yield a full
parse of $g'$.
%
This Day convolution closed structure is used in separation logic.

Dually, the left hom $(g' \lsepimp g)$ is a partial parse from
the right of the string.

These left and right hom operations can be used to define the
\emph{derivative} of a language with respect to a character or
word. If we focus on left-to-right parsing this would be:
\[ d^l_c(g) = Yc \rsepimp g \]
Why?
\[ (Yc \rsepimp g)w = \Pi_{w_l} \{ 1 | c = w_l \} \to g(w_lw) \cong g(cw) \]
I.e., a parse of $d^l_c(g)$ is precisely a parse of the string with
$c$ prepended.
%
This allows us, for any particular string $w$, to reduce the problem
of parsing $w$ with a grammar $g$ to the problem of producing a parse
of an empty string for the grammar $d^l_w(g)$. Then the typical
derivation of the Brzozowski derivative of a regular language follows
from abstract nonsense?

If we reduce to language recognition, this is a de Morgan-dual of the
``quotient'' of a formal language.
\[ (g_1 / g_2) w \iff \exists w' \in g_2. g_1(ww') \iff \neg (\forall w' \in g_2. \neg g_1(ww')) \iff \neg ((\neg g_1 \lsepimp g_2) w) \]


\section{Application: Derivative of a Regular Semantic Action}

Let $A$ be a semantic action inductively built using the constructions
of separating conjunction, unit, binary and empty coproducts,
characters, Kleene star, and pushforward.

Claim:
\begin{theorem}
  If $A$ is a regular semantic action, then for any character $c$,
  \[ d_c(A) = Yc \rsepimp A \]
  is isomorphic to a regular semantic action.
\end{theorem}
\begin{proof}
  By induction on the definition of $A$.
  \begin{enumerate}
  \item $A = I$, then $Y c\rsepimp I \cong 0$
  \item $A = A_1 \sepconj A_2$, then
    \begin{align*}
      Yc \rsepimp A_1 \sepconj A_2
      \cong ((Yc \rsepimp A_1) \sepconj A_2) + ((A_1 \times I) \sepconj (Yc \rsepimp A_2))
    \end{align*}
  \item $A = f_*A'$, then
    \begin{align*}
      Y c \rsepimp f_*A \cong g_*(Yc \rsepimp A)
    \end{align*}
    Where $g : (1 \Rightarrow X) \to (1 \Rightarrow Y)$ is $g(h)(x) =
    f(h(x))$.
  \item TODO: finish
  \end{enumerate}
\end{proof}


%% \subsection{Endofunctors}

%% TODO

%% \section{Associativity, Precedence, and Ordering using Category-valued Presheaves}

%% The presheaf of sets model of grammars allows us to reason about not
%% just whether a string parses, but also about \emph{ambiguity} of a
%% grammar: whether there exists \emph{more than one} parse. This is
%% modeled by the grammar $g$ having more than one parse of the same
%% string $w$. We can use this as above to reason about when a grammar is
%% ambiguous or not, and if the grammar is unambiguous, we have a clear,
%% deterministic specification for the parser.

%% But what about parsing \emph{ambiguous} grammars? We can simply say
%% that a parser should produce any parse, or perhaps all of them, but in
%% practice many parser generators and programming languages supporting
%% infix/mixfix notation allow for \emph{associativity} and
%% \emph{precedence} declarations. These are described by providing an
%% ambiguous grammar and disambiguating not by rewriting the grammar, but
%% instead by expressing a \emph{preference} between the multiple
%% possible parses. For instance, one of the most common sources of
%% ambiguity in parsing is in associativity and precedence of binary
%% operators. For an example of associativity, a grammar for subtraction
%% might initially be written as:
%% \begin{mathpar}
%%   \begin{array}{rcl}
%%     E &::= & E - E \,|\,(E)\,|\,Number\\
%%   \end{array}
%% \end{mathpar}

%% But then we note that there are multiple, semantically different
%% parses of $3 - 4 - 5$: trees semantically equivalent to $(3 - 4) - 5$
%% or $3 - (4 - 5)$. We can rule out the right-nested version by
%% \emph{factoring} the grammar:
%% \begin{mathpar}
%%   \begin{array}{rcl}
%%     E &::= & E - L \,|\,L\\
%%     L &::= & ( E ) \,|\, Number
%%   \end{array}
%% \end{mathpar}
%% This grammar has fewer parses than the previous: now $3 - 4 - 5$ will
%% only parse as a left-nested term since $4 - 5$ does not match the
%% ``leaf'' production $L$. Instead, most languages declare that $-$ is a
%% left associative operator. We can think of this as expressing a
%% \emph{preference} for certain parses over others, specifically the tree
%% % https://q.uiver.app/?q=WzAsNSxbMiwyLCJFXzIiXSxbMywxLCJFXzMiXSxbMiwwLCItIl0sWzEsMSwiLSJdLFswLDIsIkVfMSJdLFsyLDFdLFszLDRdLFszLDBdLFsyLDNdXQ==
%% \[\begin{tikzcd}
%% 	&& {-} \\
%% 	& {-} && {E_3} \\
%% 	{E_1} && {E_2}
%% 	\arrow[from=1-3, to=2-4]
%% 	\arrow[from=2-2, to=3-1]
%% 	\arrow[from=2-2, to=3-3]
%% 	\arrow[from=1-3, to=2-2]
%% \end{tikzcd}\]
%% is prefered over
%% % https://q.uiver.app/?q=WzAsNSxbMSwwLCItIl0sWzAsMSwiRV8xIl0sWzIsMSwiLSJdLFsxLDIsIkVfMiJdLFszLDIsIkVfMyJdLFswLDFdLFswLDJdLFsyLDNdLFsyLDRdXQ==
%% \[\begin{tikzcd}
%% 	& {-} \\
%% 	{E_1} && {-} \\
%% 	& {E_2} && {E_3}
%% 	\arrow[from=1-2, to=2-1]
%% 	\arrow[from=1-2, to=2-3]
%% 	\arrow[from=2-3, to=3-2]
%% 	\arrow[from=2-3, to=3-4]
%% \end{tikzcd}\]
%% We can express this as an \emph{ordering}, where the prefered tree is
%% ``bigger''.

%% Fortunately, we can express this with a small change. Rather than
%% taking functors from $\Sigma^*$ to \emph{sets} we can take functors to
%% \emph{partially ordered sets}, or even more generally into
%% \emph{categories}. So a grammar $g$ defines for each string $w$ a
%% category $g w$ whose objects are parses and for any two parses
%% $p_1,p_2 \in g w$, we have a set $(g w)(p_1,p_2)$ of ``reasons to
%% prefer $p_2$ over $p_1$''. Furthermore, preference is transitive and
%% reflexive, with associativity and unit of transitive reasoning.

%% Then the parsing problem can be rephrased as, given a syntactic
%% description of a grammar $g$ and a string $w$, what is the \emph{best}
%% parse in $g w$, if any? In categorical language, can we construct a
%% terminal object of $g w$?

%% \section{Ordered Separation Logic for Formal Grammars}

%% Next, based on our algebraic constructions in the previous section, we
%% describe a \emph{polymorphic ordered separation logic} for
%% grammars. While this may seem a surprising application of separation
%% logic, it is quite natural when considering the semantics: Day
%% convolution of presheaves over a \emph{commutative} monoid is one of
%% the oldest semantic interpretations of separation logic proofs. For
%% parsing, we are taking presheaves over a non-commutative monoid (if
%% $|\Sigma| \ge 2$). This non-commutativity is obvious for instance in
%% regular expressions: $r_1r_2$ almost never presents the same language
%% as $r_2r_1$.

%% This consists of three judgments.
%% \begin{enumerate}
%% \item An \emph{open} grammar $g$ parameterized by a context of grammar
%%   variables $\Gamma = \gamma_1,\ldots$:
%%   \[ \Gamma \vdash g \]
%% \item For each $\Gamma$ a judgment of \emph{parse terms} of a grammar
%%   $g$ parameterized by a ``bunch'' of parsed words $\bunch$, where $g$
%%   and $\bunch$ are both parameterized by $\Gamma$:
%%   \[ \Gamma | \bunch \vdash p : g \]
%% \end{enumerate}

%% The intended denotational semantics is
%% \begin{enumerate}
%% \item A grammar $\Gamma \vdash g$ denotes a (mixed-variance?) functor
%%   $\sem{g} : \Lang^{|\Gamma|} \to \Lang$
%% \item A parse term $\Gamma | \mathcal W \vdash p : g$ denotes a
%%   natural transformation $\mathcal W \Rightarrow g$ where a bunch $\mathcal W$
%%   is interpreted appropriately using the two monoidal structures on
%%   $\Lang$:
%%   \begin{align*}
%%     \sem{\cdot_\epsilon} &= \epsilon\\
%%     \sem{\mathcal W_m,w : g} &= \sem{\mathcal W_m}\sepconj \sem{g}\\
%%     \sem{\mathcal W_m,\mathcal W_a} &= \sem{\mathcal W_m}\sepconj \sem{\mathcal W_a}\\
%%     \sem{\cdot_\top} &= \top\\
%%     \sem{\mathcal W_a;\mathcal W_m} &= \sem{\mathcal W_a} \times \sem{\mathcal W_m}        
%%   \end{align*}
%% \end{enumerate}

%% To get a feel for the differences between the multiplicative and
%% additive structure here are some examples.
%% A parse term
%% \[ \cdot_\epsilon \vdash p : g\]
%% denotes a $g$-parse of the empty string, whereas a term
%% \[ \cdot_\top \vdash p : g\]
%% denotes a function from strings $w \in \Sigma^*$ to $g$-parses of $w$.
%% The grammar $g \sepconj g'$ presents the language that parses words
%% $w$ that can be split as $w_1w_2$ with a pair of parses $gw_1$ and
%% $g'w_2$. On the other hand, $g \times g'$ is the intersection of the
%% two languages: a parse of a word $w$ is a pair of a $g$ parse and a
%% $g'$ parse of the entire word.  $g \times 1$ is equivalent to $g$,
%% whereas a parse of $w$ using $g \sepconj 1$ is a $g$-parse of a prefix
%% $w'$ of $w$, and correspondingly $1 \sepconj g$ parses words with a
%% $g$ suffix. On the other hand $\epsilon \times g$ is the restriction
%% of $g$ to only the parses of the empty string.

%% \begin{figure}
%%   \begin{mathpar}
%%     \inferrule
%%     {\Delta, \gamma \vdash g \and \textrm{where $\gamma$ strictly positive}}
%%     {\Delta \vdash \mu \gamma. g}

    
%%   \end{mathpar}
%%   \caption{Ordered Separation Logic}
%% \end{figure}

%% Given any string $w$ over our alphabet, there is a corresponding
%% grammar $\lceil w \rceil$ that accepts precisely $w$ with a single
%% parse. Then we can formulate the \emph{parsing problem} as a
%% \emph{program synthesis} problem:
%% \begin{quote}
%%   Given a closed grammar $\cdot \vdash g$ and a word $w$, attempt to
%%   construct a parse $x : \lceil w \rceil \vdash p : g$
%% \end{quote}
%% Or, even more naturally, we can break w up into its constituent
%% characters:
%% \begin{quote}
%%   Given a closed grammar $\cdot \vdash g$ and a word $c_1\ldots_n$,
%%   construct a parse $x_1 : \lceil c_1 \rceil,\ldots_n \vdash p
%%   : g$
%% \end{quote}

%% \begin{figure}
%%   \begin{mathpar}
%%     \inferrule{g \in \mathcal G}{\mathcal G \vdash g}

    

%%   \end{mathpar}
%%   \caption{Context-free Expressions}
%% \end{figure}

%% \section{Semantic Actions and Grammar/Parser Combinators}

%% Finally, we arrive at the last main component of a parser, the
%% so-called ``semantic actions''. While the constructive parsing
%% approach does produce a parse \emph{tree}, it is not usually the case
%% that this parse tree is \emph{the same} as the desired typed of
%% ASTs. Instead, we need only that there is a \emph{function} from parse
%% trees to ASTs. In general this might fail to be injective (such as
%% forgetting whether parentheses were used) or surjective (if there are
%% internal compiler forms that do not directly correspond to source
%% programs).


%% We can formalize this by noting that there is a functor, the constant
%% functor $\Delta$ from $\Set$ to $\Set^{\Sigma^*}$ that sends a set $A$
%% to the constant presheaf $\Delta(A)_w = A$. Then we can take the comma
%% category $\Set^{\Sigma^*}/K$. The objects of this category are pairs
%% of a grammar $g$ and a set $A$ with a function
%% \[ \prod_{w \in \Sigma^*} g_w \to A \]

%% Then the projection functor $cod : \Set^{\Sigma^*}/\Delta \to \Set$ is
%% an opfibration. The pushforward is the ``map'' operation on
%% grammars. We can equivalently view an opfibration as a functor from
%% $\Set$ to $\Set^{\Sigma^*}$

%% While the presheaf presentation directly captures the notion of a
%% parse tree, in a program the parse tree described by a grammar is
%% often massaged to express precedence data and so the tree structure of
%% a parse is more complex than the actual data being parsed. For
%% instance, an AST data type for a single binary arithmetic operation
%% can be represented as a binary tree with numbers at the leaves, such
%% as the following Haskell/Agda-like data definition:
%% \begin{verbatim}
%% data Tree where
%%   op   : Tree -> Tree -> Tree
%%   leaf : Num  -> Tree
%% \end{verbatim}


%% If the single operator op is defined to be left-associative (such as
%% subtraction/division), a typical grammar for parsing these expressions
%% would be defined by \emph{factoring} the grammar into two mutually
%% recursive non-terminals, \emph{E} for expression and \emph{L} for
%% leaf:
%% \begin{mathpar}
%%   \begin{array}{rcl}
%%     E &::= & E - L \,|\,L\\
%%     L &::= & ( E ) \,|\, Number
%%   \end{array}
%% \end{mathpar}

%% This roughly corresponds to the following mutually recursive datatypes:
%% \begin{verbatim}
%% data Expr where
%%   op   : Expr -> Leaf -> Expr
%%   leaf : Leaf -> Expr

%% data Leaf where
%%   parens : Expr -> Leaf
%%   num    : Number -> Leaf
%% \end{verbatim}

%% I.e., an \texttt{Expr} is a non-empty list of \texttt{Expr + Number}.

%% This mismatch is usually handled by \emph{semantic actions}, a kind of
%% fold over the parse tree that is compositionally expressed alongside
%% the definition of the parser itself. This can be seen as a manually
%% performed optimization to make up for the compiler's inability to
%% \emph{fuse} the two functions of parsing (an unfold) with constructing
%% the intended AST (a fold). However, semantic actions become more
%% useful when we have a \emph{compositional} language of grammars with
%% corresponding semantic actions, such as in parser combinator
%% libraries.

%% Parser combinator libraries are typically based on some notion of
%% backtracking error monad: a term of type \texttt{Parser Sigma A} is a
%% parser of strings from the alphabet Sigma that on correct parses produces
%% an associated value of type \texttt{A}.
%% %
%% Intuitively, this should be semantically (though not performantly)
%% isomorphic to a grammar $g$ paired with a function from $g$-parses to
%% $A$.

\section{Internal Language for the Topos of Semantic Actions}

%% We can define some concepts on grammars using the internal language.

%% A grammar $g$ is
%% \begin{enumerate}
%% \item \emph{total} if for every $w$ there is at least one
%%   parse. Equivalently, $g$ has a global element.
%% \item \emph{deterministic} if for every $w$ there is at most one
%%   parse. In the internal language, $g$ is a proposition.
%% \item \emph{empty} if for every $w$ there are no parses. In the
%%   internal language, $g \cong 0$
%% \end{enumerate}

%% \begin{definition}
%%   A grammar $g$ is \emph{decidable} if its propositional truncation
%%   $||g||$ is a complemented truth value.
%% \end{definition}

%% \begin{definition}
%%   A grammar $g$ is \emph{uniquely decidable} if there exists a
%%   ``complement'' $\neg g$ such that $g \vee \neg g$ is total and
%%   deterministic.
%% \end{definition}

%% In other words, $g \vee g' \cong \top$

A powerful tool for expressing constructions in a topos is the use of
the \emph{internal language}, which for a generic topos is a form of
extensional dependent type theory.
%
To extract maximum benefit from the internal language, we add axioms
and judgments to type theory for the specific topos models we are
working with.
%
For the toposes of grammars and semantic actions we have a plethora of
structure, we highlight the three most relevant:

\begin{itemize}
\item Both are toposes so models of extensional dependent type theory
\item They are models of the (non-dependent) ordered logic of bunched
  implications in that they are non-symmetric monoidal closed as well
  as Cartesian closed.
\item The topos of semantic actions, being constructed by Artin
  gluing, contains a proposition $\Phi$ that presents the categories
  $\Set$ and $\Grammar$ as open and complementary closed
  subtoposes. Sterling and Harper refer to this situation as a ``phase
  distinction''.
\end{itemize}

Combining the internal language of dependent type theory with bunched
implications or even just monoidal structure is tricky. The reason is
that all constructions need to be stable under pullback to be used in
dependent type theory, but our monoidal structure is not stable under
pullback, as is typical.
%
Fortunately, the phase distinction allows for a fruitful combination
of these aspects, called \emph{quantitative type theory}
\cite{atkey-qtt,mcbride-nuttin}.
%
We will need a non-symmetric variant.

\section{Potential Extensions}

What we have presented so far is quite simple. We highlight a few
possible avenues for exploration.

\subsection{More General Parsing}

Our goal with this theory is to have a higher explanatory power for
real life parsing tools than existing theory of langauges. We consider
several ad-hoc extensions of langauges and how they might be modeled
semantically.

First, almost all parser generators support \emph{precedence} rules
for modularly defining the binding of different binary operators.
This technically goes outside of the notion of grammar, though can be
explained by translation to ordinary grammars.
%
Instead we can model this semantically by defining a \emph{precedence
grammar} as a functor from strings to preordered sets. Then the
parsing problem is not to find a matching parse, but instead find a
\emph{greatest} parse.

Another extension is to describe the parsing of morphisms of a
category other than strings. This can be used to model common ad hoc
extensions of parsing such as the end of string character \$ by
parsing arrows from $\bullet$ to $\top$ in a category with two
objects:
\[% https://tikzcd.yichuanshen.de/#N4Igdg9gJgpgziAXAbVABwnAlgFyxMJZARgBoAGAXVJADcBDAGwFcYkQAdDnCNEAX1LpMufIRTkK1Ok1bsuAI2aNGMHAOkwoAc3hFQAMwBOEALZIyIHkkkyWbRCAAkAoSGNmLNa4kuMIvESSBkxwMNKM9AowjAAKIngE7EZY2gAW6jQM9uwAxq6GJua+3hBeIP6BKMQAHMGh4TSR0XEJYsmpGSBZsg4g+fyU-EA
\begin{tikzcd}
\bullet \arrow[r, "\$"] \arrow["w"', loop, distance=2em, in=305, out=235] \arrow["w'"', loop, distance=2em, in=125, out=55] & \top
\end{tikzcd}\]
Here the endomorphisms of $\bullet$ are the strings and a single arrow
\$ allows you to transition to the end of string.  We could extend our
model by viewing a grammar over a category $\mathcal C$ as a functor
$|\mathcal C^\to| \to \Set$, which would mean a grammar gives you a
set of parses for all hom sets. This only has a partial monoidal
structure on the base category, but that is enough to define a
monoidal structure on the category of presheaves.

Finally, the applications of most interest to us are the extensions to
tree grammars, or better yet, grammars for trees with binding. For
example, given a polynomial functor $P$ describing a type of
``operations'', we could define a tree grammar as a functor from the
free operad of the polynomial to set $|P^*| \to \Set$, where each
operation of the polynomial would induce its own version of the
separating conjunction on tree grammars.

\bibliographystyle{plain}
\bibliography{grammar}

\end{document}

