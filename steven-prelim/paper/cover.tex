% -*- fill-column: 80; -*-
\documentclass[acmsmall,screen,nonacm]{acmart}
\usepackage{mathpartir}
\usepackage{tikz-cd}
\usepackage{lipsum}
\usepackage{enumitem}
\usepackage{wrapfig}
\usepackage{fancyvrb}
\usepackage{hyperref}
\usepackage{cleveref}
\usepackage{stmaryrd}
\usepackage{tikz}
\usetikzlibrary{automata, positioning, arrows}

\usepackage{todonotes}
\newcommand{\todoin}[1]{\todo[inline]{TODO:\@ #1}}

\newcommand{\sem}[1]{\llbracket{#1}\rrbracket}
\newcommand{\cat}[1]{\mathbf{#1}}
\newcommand{\lto}{\multimap}
\newcommand{\tol}{\mathrel{\rotatebox[origin=c]{180}{$\lto$}}}
\newcommand{\Set}{\mathbf{Set}}
\newcommand{\Gr}{\mathbf{Gr}}
\newcommand{\Type}{\mathbf{Type}}
\newcommand{\Prop}{\mathbf{Prop}}
\newcommand{\Bool}{\mathbf{Bool}}
\newcommand{\nat}{\mathbb{N}}

\newcommand{\gluedNL}{{\mathcal G}_S}
\newcommand{\gluedNLUniv}{{\mathcal G}_{S,i}}
\newcommand{\gluedL}{{\mathcal G}_L}

\newcommand{\simulsubst}[2]{#1\{#2\}}
\newcommand{\subst}[3]{\simulsubst {#1} {#2/#3}}
\newcommand{\letin}[3]{\mathsf{let}\, #1 = #2 \, \mathsf{in}\, #3}
\newcommand{\lamb}[2]{\lambda #1.\, #2}
\newcommand{\lamblto}[2]{\lambda^{{\lto}} #1.\, #2}
\newcommand{\lambtol}[2]{\lambda^{{\tol}} #1.\, #2}
\newcommand{\dlamb}[2]{\overline{\lambda} #1.\, #2}
\newcommand{\app}[2]{#1 \, #2}
\newcommand{\applto}[2]{#1 \mathop{{}^{\lto}} #2}
\newcommand{\apptol}[2]{#1 \mathop{{}^{\tol}} #2}
\newcommand{\PiTy}[3]{\Pi #1 : #2.\, #3}
\newcommand{\SigTy}[3]{\Sigma #1 : #2.\, #3}
\newcommand{\LinPiTy}[3]{\widebar\Pi #1 : #2.\, #3}
\newcommand{\LinSigTy}[3]{\widebar\Sigma #1 : #2.\, #3}
\newcommand{\amp}{\mathrel{\&}}
\newcommand{\GrTy}{\mathsf{Gr}}

\newcommand{\ctxwff}[1]{#1 \,\, \mathsf{ok}}
\newcommand{\ctxwffjdg}[2]{#1 \vdash #2 \,\, \mathsf{type}}
\newcommand{\linctxwff}[2]{#1 \vdash #2 \,\, \mathsf{ok}}
\newcommand{\linctxwffjdg}[2]{#1 \vdash #2 \,\, \mathsf{linear}}

\newif\ifdraft
\drafttrue
\newcommand{\steven}[1]{\ifdraft{\color{orange}[{\bf Steven says}: #1]}\fi}
\renewcommand{\max}[1]{\ifdraft{\color{blue}[{\bf Max says}: #1]}\fi}
\newcommand{\pedro}[1]{\ifdraft{\color{red}[{\bf Pedro says}: #1]}\fi}
\newcommand{\pipe}{\,|\,}

\begin{document}

\title{CSE Preliminary Examination}
\author{Steven Schaefer}
\affiliation{\department{Electrical Engineering and Computer Science}
  \institution{University of Michigan}
  \country{USA}
}
\email{stschaef@umich.edu}

\makeatletter
\let\@authorsaddresses\@empty
\makeatother

\maketitle
\pagestyle{plain}

\section*{Distribution of Work}
This work was conducted with Max New, one of my advisors, and Pedro H.A. de
Amorim, a collaborator at Oxford. The majority of this paper is adapted from a
draft written by Max, Pedro, and I. This paper is an adaptation of my contributions to this draft. In summary,
these works include the proofs presented, contribution to the formation of
inference rules, choice of axioms, and an ongoing formalization effort in Agda of
several of the included constructions.

I am omitting some work from Max and Pedro that may be
relevant to this paper, although not directly necessary. Some of these omissions
include a canonicity result for the type theory and a categorification of Kleene
algebras.

\end{document}
