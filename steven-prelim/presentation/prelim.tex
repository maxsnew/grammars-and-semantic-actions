%%% Local Variables:
%%% coding: utf-8
%%% mode: latex
%%% TeX-engine: xetex
%%% End:
\documentclass[10pt,notes]{beamer}
% NOTE May need to manually set tex engine in emacs

\usetheme[progressbar=frametitle]{metropolis}
\usepackage{appendixnumberbeamer}

\usepackage{svg}
\usepackage{ulem}
\usepackage{cancel}
\usepackage{wrapfig}
\usepackage{fancyvrb}
\usepackage{hyperref}
\usepackage{cleveref}
\usepackage{stmaryrd}
\usepackage{amsmath}
\usepackage{listings}
\usepackage{pdfpc}
\newcommand<>{\talknote}[1]{\only#2{\pdfpcnote{- #1}\relax}}

\usepackage{pifont}
\newcommand{\cmark}{\ding{51}}
\newcommand{\xmark}{\ding{55}}

\usepackage[dvipsnames]{xcolor}
\definecolor{mLightGreen}{HTML}{14B03D}

\usepackage{booktabs}
\usepackage[scale=2]{ccicons}

\usepackage{graphicx}

\usepackage{pgfplots}
\usepgfplotslibrary{dateplot}

\usepackage{todonotes}
\newif\ifdraft
\drafttrue
\newcommand{\todoin}[1]{\ifdraft{\todo[inline]{TODO:\@ #1}}\fi}

\usepackage{xspace}
\usepackage{mathpartir}
\newcommand{\themename}{\textbf{\textsc{metropolis}}\xspace}

\newcommand{\sem}[1]{\llbracket{#1}\rrbracket}
\newcommand{\cat}[1]{\mathbf{#1}}
\newcommand{\lto}{\multimap}
\newcommand{\tol}{\mathrel{\rotatebox[origin=c]{180}{$\lto$}}}
\newcommand{\String}{\Sigma^{*}}
\newcommand{\Set}{\mathbf{Set}}
\newcommand{\Gr}{\mathbf{Gr}}
\newcommand{\Type}{\mathbf{Type}}
\newcommand{\Prop}{\mathbf{Prop}}
\newcommand{\Bool}{\mathbf{Bool}}
\newcommand{\nat}{\mathbb{N}}
\newcommand{\simulsubst}[2]{#1\{#2\}}
\newcommand{\subst}[3]{\simulsubst {#1} {#2/#3}}
\newcommand{\letin}[3]{\mathsf{let}\, #1 = #2 \, \mathsf{in}\, #3}
\newcommand{\lamb}[2]{\lambda #1.\, #2}
\newcommand{\lamblto}[2]{\lambda^{{\lto}} #1.\, #2}
\newcommand{\lambtol}[2]{\lambda^{{\tol}} #1.\, #2}
\newcommand{\dlamb}[2]{\overline{\lambda} #1.\, #2}
\newcommand{\app}[2]{#1 \, #2}
\newcommand{\applto}[2]{#1 \mathop{{}^{\lto}} #2}
\newcommand{\apptol}[2]{#1 \mathop{{}^{\tol}} #2}
\newcommand{\PiTy}[3]{\Pi #1 : #2.\, #3}
\newcommand{\SigTy}[3]{\Sigma #1 : #2.\, #3}
\newcommand{\LinPiTy}[3]{\overline\Pi #1 : #2.\, #3}
\newcommand{\LinSigTy}[3]{\overline\Sigma #1 : #2.\, #3}
\newcommand{\amp}{\mathrel{\&}}
\newcommand{\GrTy}{\mathsf{Gr}}

\newcommand{\ctxwff}[1]{#1 \,\, \mathsf{ok}}
\newcommand{\ctxwffjdg}[2]{#1 \vdash #2 \,\, \mathsf{type}}
\newcommand{\linctxwff}[2]{#1 \vdash #2 \,\, \mathsf{ok}}
\newcommand{\linctxwffjdg}[2]{#1 \vdash #2 \,\, \mathsf{linear}}

\usetikzlibrary{automata, positioning, arrows, fit}
\usetikzlibrary{shapes,arrows,calc,positioning,trees}
\tikzset{
  invisible/.style={opacity=0},
  visible on/.style={alt={#1{}{invisible}}},
  alt/.code args={<#1>#2#3}{%
    \alt<#1>{\pgfkeysalso{#2}}{\pgfkeysalso{#3}} % \pgfkeysalso doesn't change the path
  },
}
\usepackage{jigsaw}


\title{\large Formal Grammars as Types in Non-commutative Linear-non-linear Type Theory}
\date{\today}
\author{Steven Schaefer}
\institute{University of Michigan}
% \titlegraphic{\includegraphics[height=1.5cm]{figures/umich.png}}
%

% \usepackage[backend=bibtex]{biblatex}
% \addbibresource{refs}
%
\AtBeginSection{
\begin{frame}
  \setbeamertemplate{section in toc}[sections numbered]
  \tableofcontents[currentsection]
\end{frame}
}

\begin{document}

\maketitle

\metroset{block=fill}

\begin{frame}
  \setbeamertemplate{section in toc}[sections numbered]
  \tableofcontents
\end{frame}

\section{Introduction}

\begin{frame}{Parsing is Everywhere}
  A \emph{parser} emits structured data from an input string

  \onslide<2->{
  \begin{center}
  \begin{tikzpicture}[
    treenode/.style = {shape=rectangle, rounded corners,
    draw, align=center, font=\large,
    top color=white, bottom color=white!20},
    line/.style={draw, -latex'},
    edge from parent/.style={draw,-latex'},
    level 1/.style={sibling distance=20mm, level distance=0.5cm},
    level 2/.style={sibling distance=1cm, level distance=1cm}]

    \node (src) {\Large \texttt{"1+2*3"}};

    \node[right=2cm of src, shape=rectangle, draw] (parser) {\Large Parser};

    \node[right=3cm of parser, treenode] (tree) {$+$}
    child { node[treenode] (one) {$1$} }
    child { node[treenode] (times){$\ast$}
        child { node[treenode] {$2$} }
        child { node[treenode] {$3$} }
      };


  \draw [->, thick, shorten <= 0.3cm, shorten >= 0.3cm] (src) -- (parser);
  \draw [->, thick, shorten <= 0.3cm, shorten >= 1cm] (parser) -- (tree);

  \end{tikzpicture}
  \end{center}
  }

  \onslide<3->{
  Ubiquitous in computer science,
  \begin{itemize}
    \item<4-> Data deserialization
    \item<5-> \textbf{Programming language implementation}
      \begin{itemize}
        \item<6-> High-level source code must be understood by the machine
        \item<7-> Interpreters and compilers
      \end{itemize}
  \end{itemize}
  }
\end{frame}

\begin{frame}{Compilers: Do They Work?}
  \begin{center}
  \begin{tikzpicture}[node distance=3.7cm]
    \node (sourceCode) [visible on=<2->] {\includegraphics[width=2cm]{figures/srccode.png}};
    \node (srcLabel) [above=-0.3cm of sourceCode,visible on=<2->] {Source Code};
    \node (compiler) [right of=sourceCode, visible on=<1->] {\includegraphics[width=1cm]{figures/c.png}};
    \node (compilerLabel) [above=-0.3cm of compiler, visible on=<1->] {C Compiler};
    \node (outputCode) [right of=compiler, visible on=<3->] {\includegraphics[width=2.5cm]{figures/binary.png}};
    \node (outputLabel) [above=-0.55cm of outputCode, visible on=<3->] {Machine Code};


    \draw [->, thick, visible on=<2->, shorten >= 0.6cm] (sourceCode) -- (compiler);
    \draw [->, thick, visible on=<3->, shorten <= 0.6cm] (compiler) -- (outputCode);
    \node[draw,line width=2pt, rounded corners=5pt, fit=(compiler)(compilerLabel)] {};

    \node (front) [visible on=<4->, below=0.5cm of sourceCode] {\Large Front};
    \node (frontDesc) [visible on=<5->, below=0.2cm of front, xshift=1cm]
    {
      \begin{minipage}{5cm}
        \footnotesize
      \begin{itemize}
        \setlength\itemsep{-0.5em}
        \item Parsing
        \item Lexing
        \item Typechecking
      \end{itemize}
      \end{minipage}
    };
    \node (frontbug) [below=1.5cm of front,visible on=<8->] {\color{magenta} 0 GCC, 10 LLVM};
    \node (middle) [right of=front, visible on=<4->] {\Large Middle};
    \node (middleDesc) [visible on=<6->, below=0.2cm of middle, xshift=0.8cm]
    {
      \begin{minipage}{5cm}
        \footnotesize
      \begin{itemize}
        \setlength\itemsep{-0.5em}
        \item IR Optimizations
      \end{itemize}
      \end{minipage}
    };
    \node (middlebug) [below=1.5cm of middle,visible on=<9->] {\color{magenta} 49 GCC, 75 LLVM};
    \node (back) [right of=middle, visible on=<4->] {\Large Back};
    \node (dummy) [below=0.4 of middle]{};
    \node (backDesc) [visible on=<7->, below=0.2cm of back, xshift=1cm]
    {
      \begin{minipage}{5cm}
        \footnotesize
      \begin{itemize}
        \setlength\itemsep{-0.5em}
        \item Assembly Generation
        \item Register Allocation
      \end{itemize}
      \end{minipage}
    };
    \node (backbug) [below=1.5cm of back,visible on=<10->] {\color{magenta} 17 GCC, 74 LLVM};

    \node (citation) [below=2cm of middle, visible on=<8->] {Bug Finding \bf (Yang
      et al., 2011)};

    \draw [dashed, gray, visible on=<4->, shorten >= 0.3cm, shorten <= 0.7cm] (compiler) -- (front);
    \draw [dashed, gray, visible on=<4->, shorten >= 0.3cm, shorten <= 0.7cm] (compiler) -- (back);

    \draw [->, thick, visible on=<4->, shorten >= 0.3cm, shorten <= 0.3cm] (front) -- (middle);
    \draw [->, thick, visible on=<4->, shorten <= 0.3cm, shorten >= 0.3cm] (middle) -- (back);
    \node[draw,line width=2pt, visible on=<4->, rounded corners=2pt, fit=(front)] {};
    \node[draw,line width=2pt, visible on=<4->, rounded corners=2pt, fit=(middle)] {};
    \node[draw,line width=2pt, visible on=<4->, rounded corners=2pt, fit=(back)] {};
    % \node[draw,line width=2pt, visible on=<8->, rounded corners=2pt, color=red,
      % fit=(middle)(back)(dummy), inner xsep =0.4cm, inner ysep =0.5cm,
      % label={[name=l, visible on=<8->] \color{red} Verified}] {};
  \end{tikzpicture}

  \vspace{-1cm}

  \begin{tikzpicture}[node distance=3.5cm]
  \end{tikzpicture}
  \end{center}
\end{frame}

\begin{frame}{Verified Compilers}
  \talknote{verification is an immesnse human effort. 6 person years}
  \begin{center}
  {\Large Avoid bugs through verification!}

  \onslide<2->{
  \begin{tikzpicture}
  \node (clogo) {\includegraphics[width=2cm]{figures/c.png}};
  \node[below=0cm of clogo] (c) {CompCert \textbf{(Leroy et al., 2009)}};
    \node (compCertDesc) [below=0cm of c]
    {
      \begin{minipage}{3cm}
      \begin{itemize}
        \setlength\itemsep{-0.5em}
        \item<2-> Verified in Coq
        \item<2-> 100,000 lines
      \end{itemize}
      \end{minipage}
    };

  \node[right=4cm of clogo] (cakelogo) {\includegraphics[width=1.5cm]{figures/cakeml.png}};
  \node[right=1.5cm of c] (cake) {CakeML \textbf{(Kumar et al., 2014)}};
    \node (cakeDesc) [below=0cm of cake]
    {
      \begin{minipage}{4cm}
      \begin{itemize}
        \setlength\itemsep{-0.5em}
        \item<2-> Verified in HOL4
      \end{itemize}
      \end{minipage}
    };

  \end{tikzpicture}}
  \end{center}
\end{frame}

\begin{frame}{CompCert: A Case Study in Verified Compilation}
  \begin{center}
  \begin{tikzpicture}[node distance=3.7cm]
    \node (sourceCode) [visible on=<1->] {\includegraphics[width=2cm]{figures/srccode.png}};
    \node (srcLabel) [above=-0.3cm of sourceCode,visible on=<1->] {Source Code};
    \node (compiler) [right of=sourceCode, visible on=<1->] {\includegraphics[width=1cm]{figures/c.png}};
    \node (compilerLabel) [above=-0.3cm of compiler, visible on=<1->] {CompCert};
    \node (outputCode) [right of=compiler, visible on=<1->] {\includegraphics[width=2.5cm]{figures/binary.png}};
    \node (outputLabel) [above=-0.55cm of outputCode, visible on=<1->] {Machine Code};


    \draw [->, thick, visible on=<1->, shorten >= 0.6cm] (sourceCode) -- (compiler);
    \draw [->, thick, visible on=<1->, shorten <= 0.6cm] (compiler) -- (outputCode);
    \node[draw,line width=2pt, rounded corners=5pt, fit=(compiler)(compilerLabel)] {};

    \node (front) [visible on=<1->, below=0.5cm of sourceCode] {\Large Front};
    \node (frontbug) [below=1cm of front,visible on=<2->] {\color{magenta} 0 GCC, 10 LLVM};
    \node (frontbugc) [below=0cm of frontbug,visible on=<5->] {\color{blue} 6 CompCert???};
    \node (middle) [right of=front, visible on=<1->] {\Large Middle};
    \node (middlebug) [below=1cm of middle,visible on=<2->] {\color{magenta} 49 GCC, 75 LLVM};
    \node (middlebugc) [below=0cm of middlebug,visible on=<4->] {\color{blue} 0 CompCert};
    \node (back) [right of=middle, visible on=<1->] {\Large Back};
    \node (dummy) [below=0.4 of middle]{};
    \node (backbug) [below=1cm of back,visible on=<2->] {\color{magenta} 17 GCC, 74 LLVM};
    \node (backbugc) [below=0cm of backbug,visible on=<3->] {\color{blue} 0 CompCert};

    \node (citation) [below=2cm of middle, visible on=<2->] {\bf (Yang
      et al., 2011)};

    \draw [dashed, gray, visible on=<1->, shorten >= 0.3cm, shorten <= 0.7cm] (compiler) -- (front);
    \draw [dashed, gray, visible on=<1->, shorten >= 0.3cm, shorten <= 0.7cm] (compiler) -- (back);

    \draw [->, thick, visible on=<1->, shorten >= 0.3cm, shorten <= 0.3cm] (front) -- (middle);
    \draw [->, thick, visible on=<1->, shorten <= 0.3cm, shorten >= 0.3cm] (middle) -- (back);
    \node[draw,line width=2pt, visible on=<1->, rounded corners=2pt, fit=(front)] {};
    \node[draw,line width=2pt, visible on=<1->, rounded corners=2pt, fit=(middle)] {};
    \node[draw,line width=2pt, visible on=<1->, rounded corners=2pt, fit=(back)] {};
    \node[draw,line width=2pt, visible on=<6->, rounded corners=2pt, color=blue,
      fit=(middle)(back), inner xsep =0.5cm, inner ysep =0.5cm] (verifBox) {};
    \node[below=0.1cm, inner sep=0pt, visible on=<6->] at (verifBox.south) {\color{blue} Verified};
  \end{tikzpicture}
  \end{center}
\end{frame}

\begin{frame}{Contributions of This Project}
  \begin{alertblock}{Goals}
    \begin{enumerate}
       \item<1-> Provide a unifying framework to reason about formal grammars
       \item<2-> Implement the framework in Agda
       \item<3-> Instantiate this framework to build verified parsers
    \end{enumerate}
  \end{alertblock}

  \onslide<4->{
  \begin{block}{Roadmap}
    \begin{enumerate}
      \item<4-> Give a type theory that describes formal grammars
      \item<4-> Internalize parsing algorithms as terms
      \item<4-> Express proofs of parsing algorithm correctness
    \end{enumerate}
  \end{block}
  }

  \onslide<5>{
    \begin{center}
    \textbf{Domain specific language} for building verified parsers
    \end{center}
  }
\end{frame}

\section{Background on Formal Grammars}

\begin{frame}{Formal Grammars}
  \talknote{These is Chomsky's notion of grammar. It is the standard for parsing reasoning}
  A \emph{generative grammar} is usually given as a set of production rules
  for producing parse trees
  \begin{itemize}
    \item In the style of Chomsky
  \end{itemize}
  \vspace{-0.7cm}
  \begin{center}
      \begin{gather*}
        S \to E + E \qquad E \to E * E \\
        E \to 1 \qquad E \to 2 \qquad E \to 3
      \end{gather*}
  \end{center}

  \begin{exampleblock}{Derivation of $1+2*3$}
    \begin{center}
  \begin{tikzpicture}[%opacity=0.5,
  treenode/.style = {shape=rectangle, rounded corners,
    draw, align=center,
    top color=white, bottom color=white!20},
    line/.style={draw, -latex'},
    edge from parent/.style={draw,-latex'},
    level 1/.style={sibling distance=20mm, level distance=0.7cm},
    level 2/.style={sibling distance=1cm, level distance=0.7cm}
    ]
    \node[treenode] (tree) {$S$}
    child { node[treenode] {$E$} child {node[treenode] {$1$}} }
    child { node[treenode] {$+$} }
    child { node[treenode] {$E$}
      child {node[treenode] {$E$} child {node[treenode] {$2$}}}
      child {node[treenode] {$*$}}
      child {node[treenode] {$E$} child {node[treenode] {$3$}}}
    }
      ;
\end{tikzpicture}
    \end{center}
  \end{exampleblock}
\end{frame}

\begin{frame}{Formal Languages and Equivalence}
  \talknote{A language gathers together all of the strings in a grammar}
  \talknote{Also a Chomskyan idea}
  The \emph{formal language} for a grammar $g$ is the set of strings that match $g$

  \[
    L_{g} := \{ w \in \Sigma^{*} \mid \exists \text{ parse matching } w \text{ to } g\}
  \]

  \talknote<2->{Because they are subsets can also view them via their characteristic functions}
  \onslide<2->{
  \[
    L_{g} : \Sigma^{*} \to \mathbf{Bool}
  \]
  }

  \onslide<3->{\textbf{(Chomsky, 1963)} Grammars $g$ and $h$ are\dots}
  \begin{itemize}
    \item<4-> \emph{weakly equivalent} if they generate the same language, $L_{g} = L_{h}$
    \item<4-> \emph{strongly equivalent} if there is a bijection between their sets of parse trees, $g \cong h$
  \end{itemize}
\end{frame}

\begin{frame}{Grammars as Functions}
  \talknote{This is our first contribution. A generalization over the generative grammar}
  \talknote{Demonstrate how they generalize the old notion of grammar}
  \talknote{Use regular expression syntax to compactly draw}
  An \emph{abstract formal grammar} $g$ is a function $g : \Sigma^{*} \to \Set$
  \begin{itemize}
    \item Strings are mapped to the \textbf{set of parse trees} matching $g$
  \end{itemize}

  \begin{minipage}[t]{.4\textwidth}
    \[
      \onslide<2->{g = {\color{blue} a} \otimes ({\color{magenta} b} \oplus {\color{orange} c})}
    \]
  \begin{tikzpicture}[%opacity=0.5,
  treenode/.style = {shape=rectangle, rounded corners,
    draw, align=center,
    top color=white, bottom color=white!20},
    line/.style={draw, -latex'},
    edge from parent/.style={draw,-latex'},
    level 1/.style={sibling distance=20mm, level distance=0.5cm},
    level 2/.style={sibling distance=1cm, level distance=1cm},
    visible on=<3->
    ]
  \node[treenode] (tree) {$\otimes$}
    child { node[treenode] (a) {${\color{blue} a}$} }
    child { node[treenode] (b){${\color{magenta} \mathsf{inl}}$}
        child { node[treenode] (c) {${\color{magenta} b}$} }
      };
      \node[draw, fit=(tree)(a)(b)(c),
      label={[name=l] $g(ab)$}, visible on=<3->] {};
    \node[below=2.5cm of tree] (mt) {};
    \node[below=2.5cm of a] (mta) {};
    \node[below=2.5cm of b] (mtb) {};
    \node[below=-0.2cm of mt, visible on=<4->] (mtset) {};
      \node[draw, fit=(mt)(mtset)(mta)(mtb),
      label={[name=l, visible on=<4->] $g(aa)$}, visible on=<4->] {};
\end{tikzpicture}
  \end{minipage}%
  \begin{minipage}[t]{.6\textwidth}
    \[
      \onslide<5->{h = {\color{blue} a^{*}} \otimes {\color{magenta} a^{*}}}
    \]
  \begin{tikzpicture}[%opacity=0.5,
  treenode/.style = {shape=rectangle, rounded corners,
    draw, align=center,
    top color=white, bottom color=white!20},
    line/.style={draw, -latex'},
    edge from parent/.style={draw,-latex'},
    level 1/.style={sibling distance=2cm, level distance=0.5cm},
    level 2/.style={sibling distance=1cm, level distance=1cm},
    visible on=<6->
    ]

   \node[treenode] (tree) {$\otimes$}
   child { node[treenode] (a) {${\color{blue} \mathsf{cons}}$}
     child {node[treenode] (b) {${\color{blue} a}$}}
     child {node[treenode] (c) {${\color{blue} \mathsf{nil}}$}
       child {node[treenode] {${\color{blue} \varepsilon}$}}
     }
   }
   child {node[treenode] {${\color{magenta} \mathsf{nil}}$}
     child {node {${\color{magenta} \varepsilon}$}}
     }
    ;
   \node[treenode, right=2.5cm of tree] (tree2) {$\otimes$}
   child {node[treenode] {${\color{blue} \mathsf{nil}}$}
     child {node {${\color{blue} \varepsilon}$}}
   }
   child { node[treenode] (x) {${\color{magenta} \mathsf{cons}}$}
     child {node[treenode] (y){${\color{magenta} a}$}}
     child {node[treenode] (z){${\color{magenta} \mathsf{nil}}$}
       child {node[treenode] (w){${\color{magenta} \varepsilon}$}}
     }
   }
    ;
      \node[draw, fit=(tree)(a)(b)(c)(x)(y)(z)(w),
      label={[name=l] $h(a)$}, visible on=<6->] {};
\end{tikzpicture}
  \end{minipage}%
\end{frame}

\begin{frame}[fragile]{Productions as Constructors}
  \talknote{Read this as giving a recipe for building the set from before}
  Productions can be viewed like inductive data types
  \begin{center}
    \begin{minipage}{.5\textwidth}
      \begin{gather*}
        S \to E + E \qquad E \to E * E \\
        E \to 1 \qquad E \to 2 \qquad E \to 3
      \end{gather*}
    \end{minipage}%
    \begin{minipage}{.5\textwidth}
      \begin{lstlisting}[language=haskell]
  data Expr =
    Sum of Expr * Expr |
    Prod of Expr * Expr |
    One |
    Two |
    Three
      \end{lstlisting}
    \end{minipage}%
  \end{center}
\end{frame}

\section{A Type Theory for Formal Grammars}

\begin{frame}{Why Type Theory for Parsing?}
  \textbf{(Firsov and Uustalu, 2015)} provide a verified context-free grammar parser, up to weak equivalence

  \begin{itemize}
    \item<1-> Suggest an extension where parse trees are first-class objects
    \item<1-> Parse trees serve as proofs of language membership
    \item<2-> Under Curry-Howard, expect a corresponding type system
  \end{itemize}

  \begin{center}
    \begin{tikzpicture}
      \node[visible on=<3->] (gram) {Grammars};
      \node[below=0.2cm of gram, visible on=<3->] (pts) {Parse Trees};
      \node[visible on=<3->, right= of gram] (prop) {Propositions};
      \node[visible on=<3->, below=0.2cm of prop] (proof) {Proofs};
      \node[visible on=<4->, right= of prop] (ty) {\bf Types};
      \node[visible on=<4->, below=0.2cm of ty] (term) {\bf Terms};
      \draw[visible on=<3->, <->] (gram) -- (prop);
      \draw[visible on=<4->, <->] (prop) -- (ty);
      \draw[visible on=<3->, <->] (pts) -- (proof);
      \draw[visible on=<4->, <->] (proof) -- (term);
    \end{tikzpicture}
  \end{center}

  \onslide<5->{Grammar
  $g : \Sigma^{*} \to \Set$
  can be viewed as a functor}

  \talknote{Presheaves give a recipe for designing the type theory}
  \talknote{Similar techniques are used to extract out models of separation logic}
  \talknote{Won't talk about category theory again}
  \begin{itemize}
    \item<6-> Functors into $\Set$ --- presheaves --- have significant structure
    \item<7-> Presheaves model dependent type theory
  \end{itemize}
\end{frame}

\begin{frame}{A Type Theory for Grammars}
  \talknote{We begin describing the syntax for our type theory}
  Types are \textbf{grammars}, and terms are
  \only<1-5>{\textbf{parse trees}}\only<6->{\textbf{parse transformers}}

  \[
    w : \Sigma^{*} \vdash p : g
  \]

  \onslide<2->{
  Think of $p$ as a \textbf{proof}
  \begin{itemize}
    \item Evidence that $w$ matches $g$
  \end{itemize}}

  \vspace{-0.5cm}
  \talknote<3>{If we have the string aa, we can show it parses a star}
  \onslide<3->{
  \[
    x : a \otimes a \vdash \mathsf{let}~(a_{1},a_{2}) = x~\mathsf{in}~ \mathsf{cons}(a_{1}, \mathsf{cons}(a_{2}, \mathsf{nil})) : a^{*}
  \]
  }
  \vspace{-0.5cm}
  \onslide<4->{
  \[
    \alert<6>{x : a , y : b \oplus c} \vdash (x , y) : a \otimes (b \oplus c)
  \]
  }
  \vspace{-0.5cm}
  \onslide<5->{
    \[
      \Delta \vdash M : g
    \]
    $\Delta$ is an ordered list of resources that get turned into a parse $M$ of $g$
  }
  \talknote<5>{Parse trees isn't quite accurate, must be more general}

\end{frame}

\begin{frame}{Separation Logic and Linear Logic Influence}
  Separation logic
  \begin{itemize}
    \item<1-> Reason about disjoint heap regions (disjoint substrings)
    \item<2-> Context extension concatenates resources
  \end{itemize}
  \onslide<2->{
  \[
    \inferrule
    {\Delta \vdash x : g \\ \Delta' \vdash y : h}
    {\Delta , \Delta' \vdash (x , y) : g \otimes h}
  \]}

  \onslide<3->{Ordered, linear logic}
  \begin{itemize}
    \item<3-> Resource-sensitive reasoning
    \item<3-> Using a linear resource consumes it, no duplication or discarding
    \item<3-> i.e.\ characters appear as written, and in order
  \end{itemize}
  \begin{align*}
    \onslide<4->{x : c, y : a, z : t & \vdash (x,y,z) : c \otimes a \otimes t} \\
    \onslide<5->{{\color{red} x : c, y : a, z : t} & {~\color{red} \not \vdash (y,x,z) : a \otimes c \otimes t}} \\
    \onslide<6->{{\color{red} x : c, y : a, z : t} & {~\color{red} \not \vdash (x,y,z,z) : c \otimes a \otimes t \otimes t}} \\
    \onslide<7->{{\color{red} x : c, y : a, z : t} & {~\color{red} \not \vdash (y,z) : a \otimes t}} \\
  \end{align*}
\end{frame}

\begin{frame}{Type Constructors}
  \begin{center}
  \begin{tabular}{c c}
    \textbf{Constructor} & \textbf{Syntax} \\
    Empty String & $\varepsilon$ \\
    Literal & $c \in \Sigma$ \\
    Concatenation & $g \otimes h$ \\
    Disjunction & $g \oplus h$ \\
    Conjunction & $g \amp h$ \\
    \onslide<2->{Top} & \onslide<2->{$\top$} \\
    \onslide<2->{Bottom} & \onslide<2->{$\bot$} \\
    \onslide<3->{Implication} & \onslide<3->{$g \Rightarrow h$} \\
    \onslide<3->{Negation} & \onslide<3->{$\neg g$} \\
    \onslide<4->{Least Fixed Point} & \onslide<4->{$\mu x . g$} \\
    \onslide<5->{Linear Functions} & \onslide<5->{$g \lto h$} \\
     & \onslide<5->{$g \tol h$} \\
    \onslide<6->{LNL Dependent Types} & \onslide<6->{$\LinPiTy {x} {X} {g}$} \\
    & \onslide<6->{$\LinSigTy {x} {X} {}g}$
  \end{tabular}
  \end{center}
\end{frame}

\begin{frame}{Fixed Point Operator}
  \talknote{Necessary for any programming language implementation}
  Recursive grammars are ubiquitous in practice
  \begin{itemize}
    \item Using only $c \in \Sigma$, $\varepsilon$, $\otimes$, $\oplus$, and $\amp$ we cannot describe infinite languages
  \end{itemize}

  \onslide<3->{
    \[
      \inferrule{\Delta \vdash e : A[\mu x . A / x]}{\Delta \vdash \mathsf{cons}~e : \mu x . A}
    \]
  }
  \onslide<4->{
    \[
      g^{*} := \mu {\color{YellowOrange} x} . ({\color{blue}\varepsilon} \oplus ({\color{magenta} g} \otimes {\color{YellowOrange} x}))
    \]
  }
  \onslide<5->{
    \[
      g^{*} \cong {\color{blue} \varepsilon} \oplus ({\color{magenta} g} \otimes {\color{YellowOrange} g^{*}})
    \]
  }
  \onslide<6->{
    Inductive data type with constructors,
    \begin{itemize}
      \item<6-> ${\color{blue} \bullet} \vdash \mathsf{nil} : g^{*}$
      \item<6->
            ${\color{magenta} x : g}, {\color{YellowOrange} y : g^{*}} \vdash \mathsf{cons}(x , y) : g^{*}$
    \end{itemize}
  }
\end{frame}

\begin{frame}{Linear Function Types}
  \talknote{Enables parsing algorithms that make use of derivatives}
   $g \lto h$ are the parse trees that complete to $h$ when prepended with $g$

   \[
     (c \otimes a) \lto (c \otimes a \otimes t) \cong t
   \]
   \[
     a^{*} \lto a^{*} \cong a^{*}
   \]

   \onslide<2->{
     Generalizes notion of language derivative \textbf{(Brzozowski, 1964)}
   }

   \onslide<3->{
     There is also a type with the opposite handedness $g \tol h$
   }
\end{frame}

\begin{frame}{Non-linear Types}
  \talknote{Don't say that you're lying!}
  \talknote{Inspired by previous work with an intened applicaiton}
  \talknote{choosing to not highlight the linear fragment, or its interaction with non linear, in this talk}
  Types up until now have been \emph{linear} types

  We make use of a linear fragment and a non-linear fragment
  \begin{itemize}
    \item Synthesizes dependent types and linear types
    \item Linear-non-linear theory \textbf{(Krishnaswami et al., 2015)} for parsing
  \end{itemize}

  \onslide<2->{
  The non-linear fragment is ordinary Martin-L\"of type theory
  \begin{itemize}
    \item Can have linear types depend on non-linear types
    \begin{gather*}
      \LinPiTy {x}{X}{g} \\
      \LinSigTy {x}{X}{g}
    \end{gather*}
    \item Used for bringing ordinary mathematical constructs into scope
    \item i.e. can define a type of non-linear stacks for PDAs
  \end{itemize}
  }
\end{frame}

\begin{frame}{Notions of Equivalence}
  \talknote{These are emergent, we did not plan them}
  \talknote{These show that we have sensible generalizations}
  \talknote{Transition into finishing type theory goal}
  Logical equivalence from functional programming\quad\onslide<3->{\textbf{(weak equivalence)}}
  \begin{gather*}
    x : A \vdash p(x) : B \\
    y : B \vdash q(y) : A
  \end{gather*}

  \onslide<2->{
  Induced notion of isomorphism\quad\onslide<4->{\textbf{(strong equivalence)}}
  \begin{gather*}
    p(q(y)) = y \\
    q(p(x)) = x
  \end{gather*}
  }
\end{frame}

\begin{frame}{Type Theory Implementation}
  \talknote{Don't want to implement a language from scratch}
  Implement the type theory by writing an embedded DSL in Agda

  \begin{itemize}
    \item Leverage Agda's type system as a host for our proof
    \item 2500 lines of Agda code
  \end{itemize}
  \begin{center}
  \includegraphics[width=4cm]{figures/agda.png}
  \end{center}
\end{frame}

\section{Building an Example Parser}

\begin{frame}{A Sample Parsing Algorithm}
  \talknote{To test the efficacy of the theory, we build a simple parser}
  \talknote{Recall, the theory is meant to be a language for verification. So it can internalize existing parsing algorithms as terms}
  \begin{alertblock}{Benchmark Parser}
  Build a verified regular expression parser
  \end{alertblock}
  \begin{block}{Sample Algorithm}
    \begin{enumerate}
      \item Prove that regular expressions are equivalent to NFAs \textbf{(Formalized)}
      \item Prove that NFAs are equivalent to DFAs \textbf{(WIP)}
      \item Build a parser term for DFA grammars \textbf{(Formalized)}
    \end{enumerate}
  \end{block}
\end{frame}

\begin{frame}{A Specification for Parsers}
  \onslide<1->{\alert<1>{Input string $w$}} \qquad \onslide<2->{\alert<2>{Grammar $g$}}

  \onslide<3->{
  \begin{block}{\alert<3>{Parser Term}}
    $$ \alert<4>{w : \Sigma^{*}} \vdash \alert<7>{\mathsf{parse}_{g}(w)} : \alert<5>{g} \oplus \alert<6>{\neg g} $$
  \end{block}}

  \onslide<8->{
  \begin{exampleblock}{Example}
    \begin{minipage}{.7\textwidth}
    \onslide<8->{\alert<8>{$g = a \otimes (b \oplus c)$}}

    \onslide<9->{\alert<9>{$w = ab$}}
    \onslide<10->{$\longrightarrow \mathsf{parse}_{g}(w) = {\color{blue} \mathsf{inl}(p_{g})} : \alert<10>{g} \oplus \neg g$
    \quad {\LARGE \color{blue} \cmark}}

    \onslide<11->{\alert<11>{$w = ad$}}
    \onslide<12->{$\longrightarrow \mathsf{parse}_{g}(w) = {\color{magenta} \mathsf{inr}(p_{\neg g})} : g \oplus \alert<12>{\neg g}$
    \quad {\LARGE \color{magenta} \xmark}}
    \end{minipage}%
    \begin{minipage}{.3\textwidth}
      \begin{center}
        \begin{tikzpicture}[%opacity=0.5,
        treenode/.style = {shape=rectangle, rounded corners,
          draw, align=center,
          top color=white, bottom color=white!20},
          line/.style={draw, -latex'},
          edge from parent/.style={draw,-latex'},
          level 1/.style={sibling distance=20mm, level distance=1cm},
          level 2/.style={sibling distance=1cm, level distance=1cm},
          scale = 0.7, visible on=<10->
          ]
        \node[treenode] (tree) {$\otimes$}
          child {node[treenode] {$a$}}
          child {node[treenode] {$\mathsf{inl}$}
            child {node[treenode] {$b$}}}
          ;

        \node[above=0.1cm of tree, visible on=<10->] {$\color{blue} p_{g}$};
        \end{tikzpicture}
      \end{center}
    \end{minipage}%
  \end{exampleblock}}
\end{frame}

\begin{frame}{Grammars and Their Automata}
  \talknote{We stratify grammars based on production structure}
  \talknote{Can encode the entire chomsky hierarchy}
  \talknote{Parsers are often built by running the abstract machine}
  \talknote{We encode the corresponding machines and show that they are equivalent}
  Characterize classes of grammars based on structure of their productions

  \begin{block}{Chomsky Hierarchy}
  \begin{center}
    \begin{tabular}{c c}
      \textbf{Grammar Class} & \textbf{Automata} \\
      Regular & Finite \\
      Context-free & Pushdown \\
      Context-sensitive & Linear-bounded \\
      Unrestricted & Turing machine
    \end{tabular}
  \end{center}
  \end{block}

  \begin{itemize}
    \item<2-> Build parsers by encoding automata as grammars in the type theory
  \end{itemize}
\end{frame}

\begin{frame}{A Nondeterministic Finite Automaton} $g = a^{*} \otimes (b \oplus c)$

  \only<1-21>{\alert<21>{$\alert<7>{w} = \alert<8-10>{a} \alert<11-13>{a} \alert<14-16>{b}
    \only<18-19>{\otimes \alert{\varepsilon}}$} \quad \only<21>{ {\Large \color{blue} \cmark}}}

  \onslide<3->{
  \begin{center}
  \begin{tikzpicture}[node distance = 25mm ]
    \Large
    \alert<4,8>{\node (0) {start};}
    \alert<4,8,10-11,13-14>{\node[state, right of=0] (1) {$q_{1}$};}
    \alert<5,20>{\node[state, right of=1, accepting] (2) {$q_{2}$};}
    \alert<16-17>{\node[state, below of=1] (3) {$q_{3}$};}

    \alert<15>{\draw[->] (1) edge[left] node{$b$} (3);}
    \draw[->] (1) edge[above] node{$c$} (2);
    \alert<9,12>{\draw[->] (1) edge[loop above] node{$a$} (1);}
    \alert<19>{\draw[->] (3) edge[below] node{$\varepsilon$} (2);}
    \alert<4,8>{\draw[->] (0) edge[above] node{} (1);}
  \end{tikzpicture}
  \end{center}}
\end{frame}

\begin{frame}{Automata as Grammars}
  $g = a^{*} \otimes (b \oplus c)$

  \onslide<2-21>{\alert<21>{$\alert<7>{w} = \alert<8-10>{a} \otimes  \alert<11-13>{a} \otimes \alert<14-16>{b}
    \only<18-19>{\otimes \alert{\varepsilon}}$} \quad \only<21>{ {\Large \color{blue} \cmark}}}

  \begin{center}
  \begin{minipage}{.5\textwidth}
  \onslide<3->{
  \begin{tikzpicture}[node distance = 15mm, scale=0.4]
    \alert<4,8>{\node (0) {start};}
    \alert<4,8,10-11,13-14>{\node[state, right of=0] (1) {$q_{1}$};}
    \alert<5,20>{\node[state, right of=1, accepting] (2) {$q_{2}$};}
    \alert<16-17>{\node[state, below of=1] (3) {$q_{3}$};}

    \alert<15>{\draw[->] (1) edge[left] node{$b$} (3);}
    \draw[->] (1) edge[above] node{$c$} (2);
    \alert<9,12>{\draw[->] (1) edge[loop above] node{$a$} (1);}
    \alert<19>{\draw[->] (3) edge[below] node{$\varepsilon$} (2);}
    \alert<4,8>{\draw[->] (0) edge[above] node{} (1);}
  \end{tikzpicture}
  }
  \end{minipage}%
  \begin{minipage}{.5\textwidth}
    \onslide<4->{
    Define a type of traces between two states

    \onslide<4->{$$ \mathsf{Trace}(q_{1}, q_{2}) =$$
    \begin{equation*}
      \mu
      \begin{pmatrix}
        \alert<8,10,11,13,14>{g_{q_{1}}} := & \alert<9,12>{(a \otimes g_{q_{1}})} \\ & \oplus  ( c \otimes g_{q_{2}} ) \\ & \oplus \alert<15>{(b \otimes g_{q_{3}})} \\
        \alert<20>{g_{q_{2}}} := & \alert<5,20>{\varepsilon} \\
        \alert<16,17>{g_{q_{3}}} := & \alert<19>{g_{q_{2}}} \\
      \end{pmatrix}. \alert<4>{\color<21>{blue} g_{q_{1}}}
    \end{equation*}}
    }
  \end{minipage}
  \end{center}
\end{frame}

\begin{frame}{A Sample Parsing Algorithm}
  \begin{alertblock}{Benchmark Parser}
  Build a verified regular expression parser
  \end{alertblock}
  \begin{block}{Sample Algorithm}
    \begin{enumerate}
      \item Prove that regular expressions are equivalent to NFAs \textbf{(Formalized)}
      \item Prove that NFAs are equivalent to DFAs \textbf{(WIP)}
      \item Build a parser term for DFA grammars \textbf{(Formalized)}
    \end{enumerate}
  \end{block}
\end{frame}

\section{Next Steps and Future Work}

\begin{frame}{More Expressive Grammar Classes}
  \begin{itemize}
    \item<1-> Conclude the regular expression benchmark
    \item<2-> Extend to (deterministic) context free grammar parsing algorithms
        \begin{itemize}
          \item<3-> Emit a verified parser generator for CFGs
        \end{itemize}
    \item<4-> Restricted forms of context sensitivity
        \begin{itemize}
          \item<5-> Such as the interval parsing grammars in \textbf{(Zhang et al., 2023)}
        \end{itemize}
  \end{itemize}
\end{frame}

\begin{frame}{Related Work}
  \talknote{KA encapsulated with weak equivalence}
  \talknote<2>{They use types for parsing but terms are grammars and types are LL(1)-ness}
  \onslide<1->{Grammar Formalisms}
  \begin{itemize}
    \item Kleene Algebra --- algebraic reasoning about regular expressions
    \item<2-> \textbf{(Krishnaswami and Yallop, 2019)} develop LL(1) parsers for context-free expressions using typing
  \end{itemize}

  \talknote<3>{They later give a validation procedure for CompCert}
  \talknote<3>{incomplete validation, and limited at LR(1)}
  \talknote<3>{LR(1) more powerful than LL(1). LR:target string back to start symbol. LL:start to target}
  \talknote<4>{Several other works build verified parsers, but each have some caveats}
  \talknote<4>{internalize each of these within the type theory}
  \onslide<3->{Verified Parsers}
  \begin{itemize}
    \item<3-> CompCert team rewrites parser and gives a validation procedure for LR(1) parsers \textbf{(Jourdan et al, 2012)}
    \item<4-> CoStar \textbf{(Lasser et al., 2023)} gives a verified predictive
          parser for non-left-recursive context free grammars (CFGs)
    \item<4-> \textbf{(Danielsson, 2010)} gives a verified parser combinator library
          in Agda for non-left recursive CFGs
    \item<4-> \textbf{(Firsov and Uustalu, 2015)} give a verified parser for arbitrary CFGs, up to weak equivalence
  \end{itemize}
\end{frame}

\begin{frame}{Generalizations}
  \talknote{Semantic actions give computation at parse-time}
  Semantics actions
  \begin{itemize}
    \item Perform semantic analysis of code during parsing
    \item i.e.\ evaluate arithmetic expressions while parsing
  \end{itemize}

  \talknote<2>{Is it possible to provide a verified unification algorithm, or verified Hindley-Milner}
  \onslide<2->{
    In this talk, we made a syntax to reason about parsing as a function $\Sigma^{*} \to \Set$

    We can instead map from tree structured data instead of $\Sigma^{*}$
    \begin{itemize}
      \item Type theory for reasoning about typechecking algorithm correctness
      \item Stranger type theory
    \end{itemize}
  }
\end{frame}

\begin{frame}[standout]
  Questions?
\end{frame}

\end{document}
