
% -*- fill-column: 80; -*-
\documentclass[sigconf,anonymous,review,screen]{acmart}
\usepackage{mathpartir}
\usepackage{tikz-cd}
\usepackage{lipsum}
\usepackage{enumitem}
\usepackage{wrapfig}
\usepackage{fancyvrb}
\usepackage{hyperref}
\usepackage{cleveref}
\usepackage{stmaryrd}
\usepackage{tikz}
\usetikzlibrary{automata, positioning, arrows, fit}

\usepackage{pdfpages}

\usepackage{todonotes}

\newcommand{\sem}[1]{\llbracket{#1}\rrbracket}
\newcommand{\cat}[1]{\mathbf{#1}}
\newcommand{\lto}{\multimap}
\newcommand{\tol}{\mathrel{\rotatebox[origin=c]{180}{$\lto$}}}
\newcommand{\String}{\Sigma^{*}}
\newcommand{\Set}{\mathbf{Set}}
\newcommand{\Gr}{\mathbf{Gr}}
\newcommand{\Type}{\mathbf{Type}}
\newcommand{\Prop}{\mathbf{Prop}}
\newcommand{\Bool}{\mathbf{Bool}}
\newcommand{\nat}{\mathbb{N}}

\newcommand{\gluedNL}{{\mathcal G}_S}
\newcommand{\gluedNLUniv}{{\mathcal G}_{S,i}}
\newcommand{\gluedL}{{\mathcal G}_L}

\newcommand{\simulsubst}[2]{#1\{#2\}}
\newcommand{\subst}[3]{\simulsubst {#1} {#2/#3}}
\newcommand{\letin}[3]{\mathsf{let}\, #1 = #2 \, \mathsf{in}\, #3}
\newcommand{\lamb}[2]{\lambda #1.\, #2}
\newcommand{\lamblto}[2]{\lambda^{{\lto}} #1.\, #2}
\newcommand{\lambtol}[2]{\lambda^{{\tol}} #1.\, #2}
\newcommand{\dlamb}[2]{\overline{\lambda} #1.\, #2}
\newcommand{\app}[2]{#1 \, #2}
\newcommand{\applto}[2]{#1 \mathop{{}^{\lto}} #2}
\newcommand{\apptol}[2]{#1 \mathop{{}^{\tol}} #2}
\newcommand{\PiTy}[3]{\Pi #1 : #2.\, #3}
\newcommand{\SigTy}[3]{\Sigma #1 : #2.\, #3}
\newcommand{\LinPiTy}[3]{\widebar\Pi #1 : #2.\, #3}
\newcommand{\LinSigTy}[3]{\widebar\Sigma #1 : #2.\, #3}
\newcommand{\amp}{\mathrel{\&}}
\newcommand{\GrTy}{\mathsf{Gr}}

\newcommand{\ctxwff}[1]{#1 \,\, \mathsf{ok}}
\newcommand{\ctxwffjdg}[2]{#1 \vdash #2 \,\, \mathsf{type}}
\newcommand{\linctxwff}[2]{#1 \vdash #2 \,\, \mathsf{ok}}
\newcommand{\linctxwffjdg}[2]{#1 \vdash #2 \,\, \mathsf{linear}}

\newif\ifdraft
\drafttrue
\newcommand{\steven}[1]{\ifdraft{\color{orange}[{\bf Steven says}: #1]}\fi}
\renewcommand{\max}[1]{\ifdraft{\color{blue}[{\bf Max says}: #1]}\fi}
\newcommand{\pedro}[1]{\ifdraft{\color{red}[{\bf Pedro says}: #1]}\fi}
\newcommand{\pipe}{\,|\,}

\newcommand{\todoin}[1]{\ifdraft{\todo[inline]{TODO:\@ #1}}\fi}

\begin{document}

\pagestyle{plain}

\pagebreak

\ifdraft{
\listoftodos{list of todos}
\pagebreak}\fi

\title{Formal Grammars as Types in Non-commutative Linear-Non-Linear Type Theory}

\author{Steven Schaefer}
\affiliation{\department{Electrical Engineering and Computer Science}
  \institution{University of Michigan}
  \country{USA}
}
\email{stschaef@umich.edu}

\author{Max S. New}
\affiliation{
  \department{Electrical Engineering and Computer Science}
  \institution{University of Michigan}
  \country{USA}
}
\email{maxsnew@umich.edu}

\author{Pedro H. Azevedo de Amorim}
\affiliation{
  \department{Department of Computer Science}
  \institution{University of Oxford}
  \country{UK}
}
\email{pedro.azevedo.de.amorim@cs.ox.ac.uk}

\makeatletter
\let\@authorsaddresses\@empty
\makeatother

\begin{abstract}
  In this paper, we propose a new syntactic formalism for formal grammars: a
  dependent linear/non-linear type theory in which the tensor product is
  non-commutative. The linear types in this calculus may be interpreted as
  grammars, and linear terms as parse transformers. This provides a logical
  characterization of grammar classes based on a substructural logic rather than
  ordinary first-order logic.

  Internally to our calculus, we carry out many classical constructions from
  the theory of formal grammars. Inspired by the common
  structure underlying various ideas in formal language
  theory, our type theory makes these relationships explicit
  and provides a unifying syntax to reason about formal
  grammars.

  We give this type theory a semantics in the category of grammars and prove a
  canonicity theorem showing that every term in a context corresponding to a
  fixed string is equal to a term in a canonical form encoding a parse tree of
  that string. Finally, we discuss the application to the design and
  implementation of correct-by-contruction parsers.
\end{abstract}

\maketitle

\setcounter{page}{1}

\section{Introduction}
The notion of formal language is central to the theory of parsing. A
\emph{formal language} $L$ over an alphabet $\Sigma$ is classically defined as a
\emph{subset} of strings $L \subseteq \String$. This definition is
especially useful as it gives a semantics to formal grammars that is completely
independent of any particular syntactic grammar formalism. Any new
notion of grammar can be given a language semantics and it provides a precise
mathematical specification for implementing a \emph{recognizer} of a language.

That is, a recognizer should be extensionally equivalent to the indicator function
$\chi_{L} : \String \to \Prop$ mapping each string $w$ to the the proposition
that it is in the language $w \in L$. However, language recognition is
insufficient for specifying a parser --- a function that decomposes a string
into semantic components that adhere to the structure of a formal grammar. In
practice, parsers do more than simply return whether a string is valid or
invalid; instead, they emit structured \emph{parse trees}\footnote{Such parse
  trees are usually not materialized in memory, as semantic actions often perform
  computation over the trees while they are being constructed.}



% To aid in the development of verified parsers, the primary contribution
% of this paper is
% an type
% theory whose types natively capture the semantics of formal
% grammars. Types in this theory are grammars, and terms are
% parse transformers. This kind of type theory naturally
% requires (non-commutative) linear and dependent types.
% Within this formalism, we may define parsers as terms
% defined in the context of an input string. Moreover, by virtue of
% typechecking, these parsers are \emph{correct-by-construction}. The intrinsic
% verification gained from the type system therefore removes the need for a
% post-hoc verification in a language like Coq, thus lowering the barrier to
% compiler frontend verification.

% Substructural logics, like linear and separation logic, offer precise control
% over how values from a type may be used, in particular how many times a value
% may be used in a derivation. This control enables \emph{resource-sensitive}
% reasoning, where we may treat a resource as \emph{consumed} after usage. In the
% context of parsing, we may treat the characters composing a string as finite resources
% that are consumed at parse-time.

% Dependent type theory extends the simply-typed $\lambda$-calculus by allowing type
% definitions to contain the occurrence of a term. In lieu of set-based
% foundations, this extension can then be taken as a foundation for mathematics.
% The expressive power unlocked by this insight has enabled the development of
% dependently-typed proof assistants
% such as Coq or Agda. Within each of these languages, systems may be given very
% precise functional specifications of their behavior.

% In 1995, Benton et al.
% \cite{bentonMixedLinearNonlinear1995} first gave categorical models of a logic
% that synthesized linear and non-linear logic. Later, Krishnaswami et al.
% \cite{krishnaswami_integrating_2015} built upon this work to develop a type
% system that unifies both linear and
% dependent types --- referred to as a \emph{linear-non-linear} (LNL) type theory. In
% such settings, we have two sub-theories --- one linear and one non-linear. The
% linear sub-theory embeds ordinary linear types without dependence, while the
% non-linear sub-theory embeds standard Martin-L\"of type theory. The interplay of
% linear types with dependence lives at the boundary between these two fragments.
% Benton et al.\ gives an accounting of how we may jump between these
% fragments via a monoidal adjunction $F \dashv G$ between linear terms and
% non-linear terms.

% To the end of defining real verified
% parsers, in our LNL theory we define a parser as a term,

% \crefname{equation}{}{}
% \begin{equation}
%   \label{eq:parser}
%   w : \String \vdash \mathsf{parse}_{g}(w) : g \oplus \top
% \end{equation}

% We will investigate the precise meaning of judgments like \cref{eq:parser} in
% \cref{sec:tt}, but first let's get an intuitive feel for the meaning behind such a
% term. On the left of the turnstile occurs a single string $w$, we may think of
% this describing what resources are linearly available to us. Using each piece of
% our the context on the left of the turnstile, we build a term
% $\mathsf{parse}_{g}(w)$ of type $g \oplus \top$. Notably, the type being
% instantiated is a sum type, and the data of which side we inject into denotes
% whether the string is accepted or rejected by the parser. That is all to say, a
% parser takes in an input string and either matches it to the desired grammar or
% rejects the string.

% Practical parsers are often implemented via an abstract automaton. For instance,
% a parser may translate a grammar into a state machine recognizes the
% same language as the original grammar, as the automata
% point-of-view lends itself to straightforward execution. The
% strength of our syntax is that it allows us to describe
% these automata as grammars themselves. Moreover, the
% equivalence between the grammar and automata presentations
% is derivable as a proof term within the theory. This allows
% us access to simultaneously internalization of the equivalence
% between grammars and automata as well as the ability to
% export to automaton implementations of a parser that are executable.

% To realize an executable implementation of our type theory,
% we shallowly embed it in Agda. Because our implementation is
% defined as a function in Agda, we can leverage the Agda
% ecosystem for code generation. Further, in lieu of
% implementing a complete language from scratch, we make use of the
% Agda proof system as a setting to guarantee correctness of
% our generate parsers.

% Within this paper, we give our progress in defining a verified
% parser for regular expressions. The construction of such a parser entails
% the definition of a parser for deterministic finite automata (DFA) grammars,
% and an equivalence proof between DFAs and regular expressions. We have thus far
% completed the construction of a parser for DFAs $D$,

% \[
%   w : \String \vdash \mathsf{parse}_{D}(w) : D \oplus \top
% \]

% and half of the equivalence proof between DFAs and regular expressions, both
% formalized in Agda. Upon completion of the second half of this proof, we would
% have a system where a user could simply give a regular grammar and then extract
% an intrinsically verified parser for it. In future work, we
% hope to show an analogous construction involving pushdown
% automata (PDAs) and context-free grammars.

% The contributions of this paper include an LNL theory for reasoning about formal
% grammars, the encoding of automata as formal grammars in this theory, and the
% definition of parser terms for DFA grammars. The contents include,

% \begin{itemize}
%   \item A syntax-independent definition of formal grammar
%   \item The formal rules governing our type system
%   \item Interpretations of the type system for reasoning about formal grammars
%         and formal languages
%   \item The presentation of automata as grammars
%   \item The definition of a parser within our type system
% \end{itemize}

% \section{A Syntax-Independent Notion of Syntax}
% \label{sec:synindsyn}

% % \max{this section doesn't flow naturally from the intro, which is all about the
% %   type theory and parsing. This is where you want to outline in the intro and
% %   possibly say here that we want to revisit ``formal grammar theory'' and then
% %   we will make a type theory that has formal grammar theory as its semantics.}

% In this section we will revisit some foundational ideas in formal
% grammar theory, before giving a type
% theory for which formal grammars serves as its semantics.

% Formal languages and grammars are among some of the oldest and most studied areas in
% theoretical computer science. The developments of the 1950s and 1960s gave rise
% to expressive formalism for reasoning about grammars, and further practical
% algorithms for constructing parsers for regular and context-free grammars.

% A grammar $G$ is often described generatively as a set of production rules for
% generating strings. Often this is presented as a finite set of nonterminal
% symbols $N$, a finite set of terminal symbols $\Sigma$, and a finite set of rewriting
% rules for substitution of nonterminals. In this setting, we continually apply
% rewrite rules until we are left only with a string of terminals
% $w \in \String$. We will give a more precise notion of
% grammar in a moment, but for now let us take this intuitive view of grammars as
% string generators.

% One of the primary objects of study is that of a \emph{formal language} over
% a fixed alphabet $\Sigma$. A language $L$ is a \emph{subset} of strings
% $L \subseteq \String$. This notion of formal language gives an intuitive
% set-based semantics for grammars. That is, we may interpret a grammar $G$ as the
% language $L(G)$ that is generated by it. We are yet to give any precise
% definition of what we mean by ``grammar'', although we note that any suitable
% notion of grammar can be given a language semantics as above.

% Formal languages serve as a basis for mathematically specifying a
% \emph{recognizer} --- a
% function $r : \String \to \Bool$ that determines if a string belongs to a
% given language. However, they are insufficient for specifying a \emph{parser}
% --- a function that decomposes a string into semantic components that adhere to
% the structure of a formal grammar. In practice, parsers don't just tell you
% if a string is valid or invalid. Instead, they emit structured data such as
% abstract syntax trees (ASTs).

% This motivates our first conceptual contribution, a simple and
% syntax-independent notion of formal grammar:

% \begin{definition}
%   \label{def:grammar}
%   A \emph{formal grammar} $G$ over a fixed alphabet $\Sigma$ is a function
%   $G : \String \to \Set$.
% \end{definition}

% We say that a grammar $G$ associates to every string $w$ the set\footnote{We
%   make little comment as to which foundations we use to encode sets. We argue\max{drop ``We argue''}
%   that our construction is agnostic to the particular foundation, and so without
%   loss of generality\max{incorrect use of ``without loss of generality''} we may take a class of small sets, a Grothendieck universe,
%   a type theoretic universe, or any similar foundation. Here we choose a type
%   theoretic universe.} $G w$ of parse
% trees showing that $w$ matches $G$.

% The simplest class of formal grammar on the Chomsky\max{first mention of Chomsky hierarchy, why not bring it up earlier when you are describing what a grammar is? Also cite}
% hierarchy are the \emph{regular expressions}

% \max{inaccurate reflection of common terminology. These are regular expressions,
%   regular grammars usually means a CFG with a restriction on recursion that
%   ensures you only allow Kleene star.}.

% \begin{gather*}
%   \mathsf{Char} \in \Sigma \\
%   \mathsf{Reg} ::= \mathsf{Char}~|~I~|~\mathsf{Reg} \otimes \mathsf{Reg}~|~\mathsf{Reg} \oplus \mathsf{Reg}~|~\mathsf{Reg}^{*}
% \end{gather*}

% Here, we use a non-standard syntax inspired by linear logic. Disjunction of
% regular expressions is written $\oplus$, and concatenation of regular
% expressions is written like the linear tensor product $\otimes$. The
% empty grammar is sometimes written in the same manner as the empty string,
% $\varepsilon$, but we instead choose to represent the empty grammar by $I$. This
% is also linear logic syntax to represent the unit $I$ of the operation $\otimes$.

% To see an illustration of the notion of grammar given in \cref{def:grammar}, consider the regular expression
% $G = (a \otimes b)^{*} \otimes a$ --- a sequence of 0 or more $ab$'s followed by
% an $a$. We may think of $G$ as a function that takes in strings and outputs all
% of the parse trees of that string. Passing the input string $baa$ through $G$
% would output the empty set, because there are no $G$-parses of $baa$. However,
% passing $aba$ through $G$ would output a singleton set, as there is only a
% single $G$-parse of $aba$. Namely, the one that breaks up $aba$ as $(ab)a$.

% Observe that for any input string $w$ for the above grammar $G$, $G(w)$ is
% either empty or a singleton set. Either the string isn't parsed by $G$,
% or it is parsed by $G$ for exactly one reason. That is, $G$ is
% \emph{unambiguous}. If we instead consider the grammar
% $H = (a \otimes b)^{*} \oplus (b \otimes a)^{*}$ there are now ambiguities. For
% instance, what is the action of $H$ on the empty string $\varepsilon$?
% The empty string $\varepsilon$ matches both $(a \otimes b)^{*}$ and $(b \otimes a)^{*}$, so
% $\varepsilon$ can match either the left or the right pattern
% of $H$. As a result, $H(\varepsilon)$ is a two-element set.

% For every grammar, we can specify a formal language recognized by that grammar. By ranging over all input strings, we can
% see that every formal grammar $G$ induces a formal language $L_{G}$,

% \[
%   L_{G} = \{ w \in \String : G w \text{ is inhabited} \}
% \]

% As they are just sets, formal languages naturally are
% endowed with a partial
% order via the subset inclusion ordering. Similarly, formal grammars are
% naturally endowed with the structure of a \emph{category} where a morphism
% $f : G \to H$ is given by a \emph{family} of functions:


% \[
%   f^{w} : Gw \to Hw, w \in \String
% \]

% Intuitively, we read a morphism of grammars as a \emph{parse transformer}: $f$
% translates a
% $G$-parse of $w$ into an $H$-parse of $w$. The notion of isomorphism induced by
% this categorical structure encodes that two grammars are equivalent if there is
% a bijective translation of parses. Isomorphisms in the category of
% grammars are precisely Chomsky's notion of \emph{strong equivalence} of
% grammars \cite{chom1963}.

% The category of grammars is quite rich in structure. It can
% equivalently be described as the functor category $\Set^{\String}$, where
% $\String$ is a discrete category. That is, the objects in $\String$ are strings
% and the only morphisms are identity morphisms. Such functor categories into
% $\Set$ --- often called presheaf categories --- carry a remarkable amount of
% structure. In fact, any presheaf category naturally models dependent type
% theory, which we exploit to motivate the design of the formalism
% presented in \cref{sec:tt}.


% \section{A Type Theory to Reason About Formal Grammars}
% \label{sec:tt}

% \subsection{Syntax}
% \label{subsec:syntax}

% In this section we will walk through most of the syntax of our type theory. We
% include the full syntax in \cref{sec:infer}.

% The linear typing judgment in our syntax takes on the following schematic form,
% \[
%   \Gamma ; \Delta \vdash M : A
% \]

% Let us briefly digest what this judgment is trying to say. First, $A$ represents
% a \emph{linear type} in our syntax. The intended semantics of these linear types
% are formal grammars. That is, the linear typing system is designed to syntactically
% reflect the behavior of formal grammars. For this reason, we may often
% interchangeably use the terms ``linear type'' and ``grammar''. To see the
% correspondence between types and grammars, note that the constructors for linear types
% mirror operations on formal grammars. For example, we have linear type constructors
% $\otimes$, $\oplus$, and $\amp$ which correspond to grammar concatenation,
% disjunction, and conjunction, respectively. The other type constructors also have
% corresponding grammar intuition which may be seen in \cref{subsubsec:other}.

% $M$ is an inhabitant of type $A$, which we want to think of as a parse tree of
% the grammar $A$. The core idea of this entire paper follows precisely from this
% single correspondence: grammars are types, and the inhabitants of these types
% are parse trees for the grammars!

% $\Gamma$ represents a \emph{non-linear context} --- also referred to as an
% \emph{intuitionistic} context. The non-linear fragment of the theory
% corresponds to usual Martin-L\"of type theory. That is, the intuitionistic
% contexts obey all of the usual structural rules --- contraction, weakening, and
% exchange --- so we may use their contents as many times as we'd like.

% On the other hand $\Delta$ represents a \emph{linear context}. These linear
% contexts
% behave substructurally --- in particular they have no contraction, weakening, or exchange
% rule. This means that everything within $\Delta$ must be used \emph{exactly
%   once} and \emph{in order of occurrence}. Thus, we can think of the linear
% contexts as an ordered list of limited
% resources. Once a resource is consumed, we cannot make reference to it again.
% We may then
% think of the variables in a linear context akin to building blocks for
% constructing patterns over strings.

% In the realm of parsing, our substructural
% restrictions make it so that the linear contexts behave quite similarly to
% formal grammars. For instance, linear context extension mirrors concatenation of
% grammars which may be seen via the introduction rule for $\otimes$,

% \[
%   \inferrule{\Gamma ; \Delta \vdash e : A \\ \Gamma ; \Delta' \vdash e' : B}{\Gamma ; \Delta , \Delta' \vdash e \otimes e' : A \otimes B}
% \]

% Just as two grammars may be concatenated, if we can build an $A$ with the
% contents of $\Delta$, and a $B$ with the contents of $\Delta'$, then we we may
% concatenate $\Delta$ and $\Delta'$ to build an inhabitant $A \otimes B$.

% This is perhaps better illustrated through a concrete example. Consider the
% regular expression $A = (a \oplus b) \otimes (b \oplus c)^{*}$ and the string
% $w = abc$. We then have the following derivation that $w$ matches $A$,

% \begin{equation}
%   \label{eq:samplederiv}
%   \inferrule
%   {
%     w_{1} : a \vdash \mathsf{inl}(w_{1}) : a \oplus b \\
%     \inferrule
%     {
%       w_{2} : b \vdash \mathsf{inl}(w_{2}) : b \oplus c \\
%       w_{3} : c \vdash \mathsf{inr}(w_{3}) : b \oplus c
%     }
%     {
%       w_{2} : b, w_{3} :c \vdash \mathsf{cons}(w_{2},w_{3}) : (b \oplus c)^{*}
%     }
%   }
%   {w_{1} : a, w_{2} : b, w_{3} : c \vdash \mathsf{inl}(w_{1}) \otimes \mathsf{cons}(w_{2}, w_{3}) : (a \oplus b) \otimes (b \oplus c)^{*} }
% \end{equation}

% When parsing $w$ into $A$ by hand, we perform the same process as reflected in
% the derivation \cref{eq:samplederiv}. That is, we split the string $w$ up into pieces, one of which matches
% $a \oplus b$, and and another that matches $(b \oplus c)^{*}$. When further
% parsing this Kleene star bit, we break it up into smaller pieces each of which
% matches the patterns $b \oplus c$ and then we combine them all into one parse of
% the Kleene star $(b \oplus c)^{*}$. Note that this exact process is echoed in
% the above proof tree.

% \subsubsection{Dependence of the Linear on the Non-Linear}

% We may now however ask, ``what are the $\Gamma$'s doing in the $\otimes$ introduction rule?''
% Each linear context $\Delta$ is dependent on a non-linear context $\Gamma$, so
% we may use variables drawn from $\Gamma$ when defining the types in $\Delta$.

% In the grammar setting, this dependence lets
% us use variables from regular dependent type theory to
% keep track of data without \emph{consuming any resources}, so to speak. That is,
% using a piece of a linear context consumes it, so we cannot linearly encode anything that
% needs to be referenced more than once.

% We will see a concrete instance of this dependence in \cref{subsubsec:pda}. In
% that case, we encode the traces of a pushdown automaton as a grammar. When doing so,
% we make use of our non-linear types to define a type of stacks. The fact that
% our encoding of stacks is non-linear reflects
% that a pushdown automaton may consume a character from the input string without
% affecting the state of the stack.

% \subsubsection{Other Type Constructors}
% \label{subsubsec:other}

% \paragraph{Brzozowksi Derivative}
% In the 1950s Janusz Brzozowksi developed a notion of derivative for formal
% languages \cite{brzozowskiDerivativesRegularExpressions1964}. The derivative of language $L$ by string $w$ is given as,

% \[
%   w^{-1}L = \{ w' \in \String : w \otimes w' \in L \}
% \]

% That is, the derivative of $L$ with respect to $w$ are precisely the strings
% that complete to an $L$-parse when prefixed by a $w$.

% Brzozowski initially gave an accounting of this operation for
% generalized regular expressions, but later work demonstrates that the same
% construction can be generalized to context free grammars \cite{mightParsingDerivativesFunctional2011}.

% The ``magic wand'' from linear logic takes on the role of Brzozowski's
% derivative in our syntax.

% \begin{equation}
%   \label{eq:linfun}
% \inferrule{\Gamma ; a : A, \Delta \vdash e : B}{\Gamma ; \Delta \vdash \lamblto a e : A \lto B}
% \end{equation}

% We read this as saying that $\Delta$ constructs a $A \lto B$-parse when
% prepending $\Delta$ with an $A$-parse completes to a $B$ parse. In this sense,
% the type $A \lto B$ behaves like a linear function type. The only distinction
% between $A \lto B$ and an intuitionistic function type is that linear functions
% consume their input.

% Note that the introduction rule presented in \cref{eq:linfun} does indeed
% generalize Brzozowski's initial presentation of the derivative. First, to match
% his initial presentation
% let $A$ be the grammar accepting a single string $w$, which we may write as the
% discrete grammar $\lceil w \rceil$.
% Additionally, our construction lifts from a language-theoretic construction
% to a grammar point-of-view.
% In this sense, we carry around not just the data of what strings belong to
% $\lceil w \rceil \lto B$ but also are endowed the data of a function that
% constructs parse trees of $B$ from pairs of $\lceil w \rceil$-parses and
% $(\lceil w \rceil \lto B)$-parses.

% \paragraph{Top and Bottom}
% We include linear versions of the unit and empty types. The
% grammar $\top$ matches every input string, while the grammar
% $\bot$ represents falsity and matches no strings. These two
% are often used together to give syntactic shorthand to
% grammars that encode propositions.

% For instance, we may encode the proposition
% $\mathsf{isVowel}$ as,

% \[
%   \mathsf{isVowel}(w) :=
%     \begin{cases}
%       \top, \quad w \in \{a, e, i, o, u\} \\
%       \bot, \quad \text{otherwise}
%     \end{cases}
% \]

% \paragraph{Fixed Point Operator}
% We include a least fixed point operator on types, $\mu$, which is the means by
% which we define recursive grammars.

% Observe that we need not take the Kleene star as a primitive
% grammar constructor, as it is definable as a fixed point.
% The Kleene star of a grammar $g$ is given as,

% \[
%   g^{*} := \mu X . I \oplus (g \otimes X)
% \]

% \begin{figure}[h!]
% \begin{mathpar}
%   \inferrule
%   {\Gamma ; \Delta \vdash p : I}
%   {\Gamma ; \Delta \vdash \mathsf{nil} : g^{*}}

%   \and

%   \inferrule
%   {\Gamma ; \Delta \vdash p : g \\ \Gamma ; \Delta' \vdash q
%   : g^{*}}
%   {\Gamma ; \Delta \vdash \mathsf{cons}(p , q) : g^{*}}

%   \and

%   \inferrule
%   {
%     \Gamma ; \Delta \vdash p : g^{*} \\
%     \Gamma ; \cdot \vdash p_{\varepsilon} : h \\
%     \Gamma ; x : g , y : h \vdash p_{\ast} : h
%   }
%   {\Gamma ; \Delta \vdash \mathsf{foldr}(p_{\varepsilon} , p_{\ast}) : g^{*}}
% \end{mathpar}
% \caption{Kleene Star Rules}
% \label{fig:star}
% \end{figure}

% Likewise, $g^{*}$ has admissible introduction and
% elimination rules, shown in \cref{fig:star}. Note that this
% definition of $g^{*}$ and these
% rules arbitrarily assigns a handedness to the Kleene star.
% We could have just as well took it to be a fixed point of
% $I \oplus (X \otimes g)$. In fact, the definitions are
% equivalent, as the existence of the $\mathsf{foldl}$ term below
% shows that $g^{*}$ is also a fixed point of
% $I \oplus (X \otimes g)$.

% \begin{equation}
%   \label{eq:foldl}
%   \inferrule
%   {
%     \Gamma ; \Delta \vdash p : g^{*} \\
%     \Gamma ; \cdot \vdash p_{\varepsilon} : h \\
%     \Gamma ; y : h , x : h \vdash p_{\ast} : h
%   }
%   {\Gamma ; \Delta \vdash \mathsf{foldl}(p_{\varepsilon} , p_{\ast}) : g^{*}}
% \end{equation}

% In fact, the $\mathsf{foldl}$ term is defined using
% $\mathsf{foldr}$ --- much in the same way one
% may define a left fold over lists in terms of a right fold
% in a functional programming language\footnote{The
%   underlying trick is to fold over a list of functions
%   instead of the original string. We curry each character $c$
%   of the string into a function that concatenates $c$, and
%   right fold over this list of linear functions. Because function
%   application is left-associative, this results in a left
%   fold over the original string. }.

% We only take fixed points of a single variable as a
% primitive operation in the type theory, but we may apply
% Beki\`c's theorem \cite{Bekić1984} to define an admissible
% notion of multivariate fixed point. This is particularly
% useful for defining grammars that encode the states of an
% automaton. In \cref{fig:multifix} we provide the
% introduction and elimination principles for such a fixed
% point, where $\sigma$ is the substitution that unrolls the
% mutually recursive definitions one level. That is,

% \begin{align*}
%   \sigma = \{ & \mu(X_{1} = A_{1} \dots, X_{n} = A_{n}).X_{1} / X_{1} , \dots, \\
%   & \mu(X_{1} = A_{1}, \dots, X_{n} = A_{n}).X_{n} / X_{n} \}
% \end{align*}

% \begin{figure}[h!]
% \begin{mathpar}
%   \inferrule
%   {\Gamma ; \Delta \vdash e : \simulsubst {A_{k}} {\sigma}}
%   {\Gamma ; \Delta \vdash \mathsf{cons}~e : \mu(X_{1} = A_{1}, \dots, X_{n} = A_{n}).X_{k}}

%   \\

%   \inferrule
%   {\Gamma ; \Delta \vdash e : \mu(X_{1} = A_{1}, \dots, X_{n} = A_{n}).X_{k} \\
%              \Gamma ; x_{1} : \simulsubst {A_{1}}{\gamma} \vdash e_{1} : B_{1} \\
%              \dots \\
%              \Gamma ; x_{n} : \simulsubst {A_{n}}{\gamma} \vdash e_{n} : B_{n}
%   }
%   {\Gamma; \Delta \vdash \mathsf{mfold}(x_{1}.e_{1}, \dots, x_{n}.e_{n})(e) : B_{k}}
% \end{mathpar}
% \caption{Multi-fixed Points}
% \label{fig:multifix}
% \end{figure}

% The most prevalent application of these multi-fixed points
% is for defining the traces through automata, as we will see
% later in \cref{eq:nfatrace}.

% \todoin{with citation, talk somewhere about how CFGs are just regular + fixed points}
% \max{this is a good point to give an overview of how regular expressions and context-free grammars/expressions are essentially sub-languages of our type system.}

% \subsection{Semantics in Grammars}
% \label{subsec:seming}
% As mentioned earlier, the interpretation of grammars as functions from
% $\String$ to $\Set$ serves as our intended semantics. When designing the syntax of this formalism, we were often
% informed by this model. That is, there is an interpretation
% $\llbracket \cdot \rrbracket$ of our syntax that provides grammar-based
% semantics.

% \begin{enumerate}
%   \item A non-linear context $\Gamma$ denotes a set $\sem \Gamma$
%   \item A non-linear type $\Gamma \vdash X : U_{i}$ denotes a family of sets
%         $\sem X : \sem \Gamma \to \Set_{i}$
%   \item A non-linear term $\Gamma \vdash e : X$ denotes a section
%         $\sem e : \Pi(\gamma : \sem \Gamma)\sem{X} \gamma$
%   \item Linear contexts $\Gamma \vdash \Delta$ and types
%         $\Gamma \vdash A : L_{i}$ both denote families of grammars
%         $\sem \Gamma \to \Gr_{i}$
%   \item A linear term $\Gamma ; \Delta \vdash M : A$ denotes a family of parse
%         transformers
%         $\sem M : \Pi(\gamma : \sem \Gamma)\Pi(w : \String) \sem \Delta \gamma w \Rightarrow \sem M \gamma w$
% \end{enumerate}

% Most of the non-linear semantics is standard, so let us investigate in more
% depth the linear semantics. The interpretation of each
% connective is exactly what is suggested by the syntax, in
% large part to the inspiration taken from this interpretation
% when designing the type theory.

% \begin{enumerate}
%   % \item $\sem{G A} \gamma = \sem{A} \gamma \varepsilon$
%   \item $\sem{\LinPiTy x X A} \gamma w = \Pi(x:\sem{X}\gamma) \sem{A}(\gamma,x) w$
%         \begin{itemize}
%           \item The semantics of the linear $\Pi$-type is
%                 indeed a dependent function out of the
%                 semantics of $X$.
%         \end{itemize}
%   \item $\sem{\LinSigTy x X A} \gamma w = \Sigma(x:\sem{X}\gamma) \sem{A}(\gamma,x) w$
%         \begin{itemize}
%           \item The semantics of the linear $\Sigma$-type is
%                 a dependent pair.
%         \end{itemize}
%   \item $\sem{I} \gamma w = \{ () \pipe w = \epsilon \}$
%         \begin{itemize}
%           \item The only string accepted by $I$ is the empty string.
%         \end{itemize}
%   \item
%         $\sem{A \otimes B} \gamma w = \Sigma(w1,w2:\Sigma^*) \left( (w_{1}w_{2} = w) \wedge \sem{A} \gamma w_1 \times \sem{B} \gamma w_2 \right)$
%         \begin{itemize}
%           \item The string $w$ matches $A \otimes B$
%                 precisely when it may be split into two
%                 pieces such that the left one matches $A$
%                 and the right one matches $B$.
%         \end{itemize}
%   \item
%         $\sem{A \lto B} \gamma w = \Pi(w_a:\Sigma^*) \left( A \gamma w_a \Rightarrow B\gamma (w_aw) \right)$
%         \begin{itemize}
%           \item The semantics of the linear function type is
%                 a dependent function that turns $A$-parses
%                 of a substring into $B$-parses of an
%                 extended string.
%         \end{itemize}
%   \item
%         $\sem{B \tol A} \gamma w = \Pi(w_a:\Sigma^*) \left( A \gamma w_a \Rightarrow B\gamma (ww_a) \right)$
%   \item $\sem{\mu x. A} \gamma = \mu (x:\Gr_i). \sem{A}(\gamma,x)$
%         \begin{itemize}
%           \item The semantics of the least fixed point
%                 operator is indeed the least fixed point of sets.
%         \end{itemize}
% \end{enumerate}





% \subsection{Semantics in Languages}
% Recall from \cref{sec:synindsyn} that every formal grammar induces a formal language. This corresponds to another
% interpretation $L(\cdot)$ of our syntax. In particular, the semantics of a
% linear type $A : L_{i}$
% from \cref{subsec:seming} are augmented such that

% \[
%   L(A) = \{ w \in \String : \sem A w \text{ is nonempty} \}
% \]

% \subsubsection{Weak Equivalence}
% Recall that an isomorphism of grammars defined a strong equivalence. We may
% additionally define a \emph{weak equivalence} of grammars when two grammars
% define the same language.

% That is, $A$ and $B$ are weakly equivalent when $L(A) = L(B)$. Note that this is
% a semantic notion, but this just as well captured syntactically by the existence
% of two judgments $\Gamma ; x : A \vdash M : B$ and
% $\Gamma ; x : B \vdash N : B$, which are not necessarily mutually inverse.

% \section{Automata as Grammars}
% Classically, formal language theory is closely related to the study of automata.
% In the Chomsky hierarchy, each language class is associated to a class of
% automata that serve as recognizers. Internal to the type theory of \cref{sec:tt},
% we can characterize these language classes syntactically; moreover, we
% demonstrate the equivalence of these language classes to its associated automata class as a
% proof term within our logic.

% Internalizing these equivalences is particularly useful for building a regular
% expression parser. That is, it is relatively easy for use to write down a parser
% for deterministic finite automata (DFAs). Then we may compose this parser with
% a proof equivalence between DFAs and regular expressions to extract a verified
% regular expression parser.

% As we will see in \cref{subsubsec:tm}, we may very quickly encode arbitrary
% recursively enumerable languages --- since we can simulate the
% tape of Turing machine in the dependent fragment of the language. We reflect
% here on this expressive power to motivate why we put stringent syntactic
% restrictions on particular classes of automata. The type theory we present truly
% is general enough to easily capture all of the Chomsky hierarchy.
% To begin, let's focus on the simplest languages and their corresponding
% automata --- regular expressions and finite automata.

% \subsection{Non-deterministic Finite Automata}
% \label{subsec:finiteaut}
% \begin{figure}
%   \begin{tikzpicture}[node distance = 25mm ]
%     \node[state, initial, accepting] (1) {$1$};
%     \node[state, below left of=1] (2) {$2$};
%     \node[state, right of=2] (3) {$3$};

%     \path[->] (1) edge[above] node{$b$} (2)
%               (1) edge[below, bend right, left=0.3] node{$\epsilon$} (3)
%               (2) edge[loop left] node{$a$} (2)
%               (2) edge[below] node{$a, b$} (3)
%               (3) edge[above, bend right, right=0.3] node{$a$} (1);
%   \end{tikzpicture}
%   \caption{An example NFA}
%   \label{fig:NFA}
% \end{figure}

% Classically, a \emph{nondeterministic finite automaton} (NFA) is a finite state machine where
% transitions are labeled with characters from a fixed alphabet $\Sigma$. These
% are often represented formally as a 5-tuple $(Q, \Sigma, \delta, q_{0}, F)$,

% \begin{itemize}
%   \item $Q$ a finite set of states
%   \item $\Sigma$ a fixed, finite alphabet
%   \item $\delta : Q \times (\Sigma \cup \{ \varepsilon\}) \to \mathcal{P}(Q)$ the labeled transition function
%   \item $q_{0} \in Q$ the start state
%   \item $F \subset Q$ a set of accepting states
% \end{itemize}

% Intuitively, this can be thought of like a directed graph with nodes in $Q$ with
% an edge $q \overset{c}{\to} q'$ whenever $q' \in \delta(q, c)$. Note that
% transitions in an NFA may be labeled with the empty string $\varepsilon$ --- such
% transitions are referred to as \emph{$\varepsilon$-transitions}. We may see an
% example of an NFA in \cref{fig:NFA}.

% From an NFA, we may construct a grammar of traces as follows:

% First, we define a mutual fixed point grammar that describes the traces through the NFA.\ Then, we have another
% grammar that tells us if we're currently in an accepting state of the automaton.
% A parse of the NFA grammar is then a pair of a trace and the data that we're in
% an accepting state.

% Consider the NFA $N$ pictured in \cref{fig:NFA}. There are three states, $1$, $2$,
% and $3$. We introduce a non-linear type $Q$ with three inhabitants $q_{1}$,
% $q_{2}$, and $q_{3}$ to represent each of these states, respectively. Given a
% $q, q'$, we can
% then define the type of traces from $q$ to $q'$. For instance, let's construct
% the traces starting at $q_{1}$ and ending at $q_{2}$ as an example.

% \begin{equation}
%   \label{eq:nfatrace}
%   \mathsf{Trace}_{N}(q_{1}, q_{2}) = \mu
%   \begin{pmatrix}
%     g_{q_{1}} := g_{q_{3}} \oplus ( b \otimes g_{q_{2}} ) \\
%     g_{q_{2}} := ( a \otimes g_{q_{2}} ) \oplus ( a \otimes g_{q_{3}} ) \oplus ( b \otimes g_{q_{3}} ) \oplus I \\
%     g_{q_{3}} := a \otimes g_{q_{1}} \\
%   \end{pmatrix}. g_{q_{1}}
% \end{equation}

% We should read this as defining three mutually recursive grammars, one for each
% state. The definitions of these mutually recursive grammars capture the
% transitions of the automaton. To ensure that we only encode traces that end in
% state 2, we only include the unit $I$ as a summand in
% $g_{q_{2}}$. That is, by only including $I$ at this location, it makes it the
% only place where derivations of $\mathsf{Trace}(q_{1}, q_{2})$ can terminate. We
% can think of all of these definitions underneath
% of the $\mu$ binder as bringing some local grammars into scope. With these local
% grammars in scope we are ultimately constructing a term of type $g_{q_{1}}$ to denote
% only traces starting in state 1.

% To encode the acceptance criteria of $N$, we want to internalize a proposition
% over each state of the NFA.\ That is, for each $q \in Q$ we define a term

% \[
%   \mathsf{acc}(q) := q \text{ is accepting}
% \]

% In this example, $\mathsf{acc}(q_{1}) = \top$ and
% $\mathsf{acc}(q_{2}) = \mathsf{acc}(q_{3}) = \bot$. An accepting trace of the NFA $N$ is then given by the following dependent grammar,
% \[
%  \mathsf{AccTrace}_{N} := \LinSigTy q Q {\left( \mathsf{Trace}_{N}(q_{0} , q) \amp \mathsf{acc}(q) \right)}
% \]

% where $q_{0}$ is the initial state. That is, a trace is accepted by the NFA if
% we can construct the trace, and the trace ends at an accepting state. This idea
% is simple enough and aligns with how we intuitively treat these automata.

% Generalizing over the above example, we want to define the type of traces as,

% \[
%   \mathsf{Trace}_{N}(q_{0}, q_{1}) = \mu
%   \begin{pmatrix}
%     g_{q} := \mathsf{Trans}(q), & q \neq q_{1} \\
%     g_{q} := \mathsf{Trans}(q) \oplus I , & q = q_{1}
%   \end{pmatrix}. g_{q_{0}}
% \]

% where $\mathsf{Trans}(q)$ is an iterated disjunction that describes which
% transition you should take. For an NFA, $\mathsf{Trans}(q)$ must take on the
% following syntactic form,

% \begin{gather*}
%  \mathsf{State} \in \{ g_{q} : q \in Q \} \\
%  \mathsf{Char} \in \Sigma \\
%  \mathsf{Trans}_{N}(q) ::= \mathsf{Char} \otimes \mathsf{State}~|~\mathsf{State}~|~\mathsf{Trans}_{N}(q) \oplus \mathsf{Trans}_{N}(q)
% \end{gather*}

% That is, $\mathsf{Trans}(q)$ is a disjunction of literals followed by grammars
% that encode states.

% When conducting proofs involving NFA grammars, we often to
% need either construct terms of type $\mathsf{Trace}_{N}(q , q')$. To
% this end, we give three admissible rules for constructing
% traces,

% \begin{figure}[h!]
%   \label{fig:admissibleintro}
%   \begin{mathpar}
%     \inferrule
%     {~}
%     {\Gamma ; \cdot \vdash \mathsf{nil} : \mathsf{Trace}_N
%       (q , q)}

%     \and

%     \inferrule
%     {\Gamma ; \Delta \vdash M : \mathsf{Trace}_N
%       (dst , q') \\
%     \exists \text{ transition } src \overset{c}{\to} dst
%     }
%     {\Gamma ; x : c , \Delta \vdash \mathsf{cons}(M) : \mathsf{Trace}_N
%     (src , q')}

%     \and

%     \inferrule
%     {\Gamma ; \Delta \vdash M : \mathsf{Trace}_N (dst , q')
%       \\
%     \exists~\varepsilon\text{-transition } src
%     \overset{\varepsilon}{\to} dst}
%     {\Gamma ; \Delta \vdash \mathsf{\varepsilon cons}(M) : \mathsf{Trace}_N
%     (src , q')}
%   \end{mathpar}
%   \caption{Admissible Trace Constructors}
% \end{figure}

% That is, we may use $\mathsf{nil}$ to terminate a trace that
% begins and ends at state $q$. The rules $\mathsf{cons}$ and
% $\mathsf{\varepsilon cons}$ are then used to inductively
% stitch traces together when sound. For instance, we can read
% the $\mathsf{cons}$ rule as saying that we can
% create a trace coming from a state $src$ provided that we may first
% transition via the character $c$ to state $dst$ and
% inductively build a trace starting from $dst$. The
% $\mathsf{\varepsilon cons}$ rule says something similar, but
% with an $\varepsilon$-transition instead of a transition
% labelled by a character. Recall that
% these rules are \emph{admissible}. They are not strictly
% necessary as primitives to conduct our proofs; however, they
% do provide convenient shorthand notation for building terms
% of type $\mathsf{Trace}_{N}(q , q')$.

% Dual to constructing traces, we often want to construct
% other terms in a context containing values of type
% $\mathsf{Trace}_{N}(q , q')$. For this purpose, we make use
% of the elimination principle for multiple-fixed points ---
% which we write as $\mathsf{mfold}$ --- given
% in \cref{subsubsec:other}.

% Note that this general construction will readily generalize to other types of
% automata. If we wanted to define say deterministic finite automata\footnote{DFAs
% \emph{could} just be defined as NFAs that happen to be deterministic. This is
% one way to do so, or you may choose to present the transition relation for the
% automaton as a transition \emph{function} instead. Concerns like these don't
% matter so much when defining things on paper, but at formalization time these
% are important and can make some proofs much easier.}, pushdown automata, Turing
% machines, etc, we just swap out the type of traces for a different, but very
% similar, trace construction.

% With this setting for finite automata, we can now internalize some classical theorems inside of our formal system.


% \subsection{A DFA Parser}
% \label{subsec:regexparser}

% Just as we encoded traces of NFAs as grammars, we likewise
% encode the traces of a DFA as grammars. The key difference
% between NFAs and DFAs is \emph{determinism} --- meaning,
% that in a state $q$ inside of DFA, given a character $c$ there
% will be exactly one transition that we may take leaving $q$
% with label $c$. For us, this changes the definition of valid
% transitions for a DFA, instead of the definition of
% $\mathsf{Trans}$ provided in \cref{subsec:finiteaut} DFAs
% obey

% \begin{gather*}
%  \mathsf{State} \in \{ g_{q} : q \in Q \} \\
%  \mathsf{Trans}(q) ::= \bigoplus_{c \in \Sigma} (c \otimes \mathsf{State})
% \end{gather*}

% Meaning, each state has a transition for every character.

% We now wish to define a parser term for DFA grammars. In
% particular, for a DFA $D$ we want to build a term,

% \[
%   w : \String \vdash \mathsf{parse}_{D} : \mathsf{AccTrace}_{D} \oplus \top
% \]

% where left injection into the output type denotes acceptance
% by the parser, and right injection denotes rejection. To
% build such a parser, it will suffice to construct a term

% \[
%   w : \String \vdash \mathsf{parse}_{D} : \LinSigTy q Q {\mathsf{Trace}_{D}(q_{0} , q)}
% \]
% This is because given a trace of a DFA, we may easily check
% if we should accept or reject by simply testing
% if the final state is accepting.

% Because $w$ is a Kleene star of characters, we may construct
% our desired $\mathsf{parse}_{D}$ term as a $\mathsf{foldl}$
% over $w$. In the empty case, we just have the trace that
% ends at the accepting state. In the recursive case, we
% effectively add to our trace by transitioning one character
% at a time, as we read them moving across $w$.

% \todoin{Derivation tree for this construction}

% Perhaps this derivation is not too surprising. All it says
% is that a DFA may be ran by transitioning a single character
% at a time, and then accepting or rejecting based on the
% final state. This is exactly what DFAs did initially, so
% what did we gain? Well, this has the benefit of our type
% system to ensure its correctness. Moreover, this construction exports to an
% intrinsically verified and executable DFA via Agda.


% \subsection{Regular Grammars and DFAs}
% \label{subsec:deriv}

% In order to extend the DFA
% parser from \cref{subsec:regexparser} to the construction of
% a verified parser
% generator for regular grammars we need to perform some
% plumbing establishing an equivalence between regular
% expressions and DFAs.

% There are several routes we may hope to take in establishing
% this equivalence. First, we could prove an equivalence
% between NFAs and regular expressions, and separately prove
% an equivalence between NFAs and DFAs.
% In \cref{subsec:eqproofs}, we include a version of
% Thompson's construction --- which established the
% equivalence between regular grammars and DFAs. We may additionally
% hope to internalize a variant of the powerset construction \cite{rabinFiniteAutomataTheir1959}
% --- which takes as input an NFA and constructs a DFA that
% recognizes the same language --- and combine the results of
% Thompson's construction and the powerset constructions to give an equivalence
% between regular expressions and DFAs.  This route is alluring, as it
% internalizes several classic grammar-theoretic constructions. However, it may
% necessitate extensions to the LNL theory, like
% a propositional truncation, and we have not yet investigated
% how this would interact with the existing types in the
% theory. The addition of a propositional truncation may seem
% harmless, but it is not always immediately clear how
% distinct constructions will interact. For instance, when
% exploring LNL models, Benton discovered that the synthesis
% of linear and dependent types require a new presentation of
% the $!$ modality from linear logic
% \cite{bentonMixedLinearNonlinear1995}. That is all to say,
% this is a work in progress and
% it is not immediate that the addition of a propositional
% truncation is adequate for establishing the weak equivalence
% between NFAs and DFAs.

% We may instead hope to internalize an equivalence between
% regular grammars and DFAs by using Brzozowski derivatives to
% directly create a DFA that is weakly equivalent to a given
% regular expression, as described by Owens et al.
% \cite{owensRegularexpressionDerivativesReexamined2009}.
% One characterization of regular grammars is that they are
% precisely those grammars which have finitely many inequivalent Brzozowski
% derivatives
% \cite{brzozowskiDerivativesRegularExpressions1964}.
% The algorithm used by Owens takes in a
% regular grammar and generates a DFA that recognizes the same
% language, and the states in this DFA are the finitely many
% derivative equivalence classes. We initially had a version of
% this theorem very roughly internalized in the LNL theory.
% To our taste, too much of this presentation relied on
% meta-arguments that lived outside of
% our formalism, and thus this particular phrasing of the
% theorem did not translate well into formalization.

% In any case, we believe
% that revisiting these lines of thought will lead to a
% satisfactory internalization of the equivalence between
% regular grammars and DFAs, and thus would bridge the gap
% between our DFA parser and a full regular expression parser.

% \subsection{Equivalence Between Regular Grammars and Finite Automata}
% \label{subsec:eqproofs}
% In this section, we describe a version of Thompson's
% construction \cite{thompsonProgrammingTechniquesRegular1968}
%    where we construct an NFA that recognizes a given regular
% expression. Moreover, we will show that this NFA is strongly equivalent to the
% original grammar. Witnessing this construction in our syntax has two benefits
% \begin{enumerate}
%   \item It reinforces this high-level view that the syntax is a natural and
%         general setting for formal grammar reasoning, as we demonstrate that
%         this formal system subsume results from existing systems, and
%   \item Following the
%         development of Thompson's construction, we then need
%         only establish the equivalence of NFAs and DFAs to
%         complete the full regular expression parser
% \end{enumerate}

% \begin{theorem}[Thompson]
%   \label{thm:thompson}
%   For $g$ a regular grammar $g$, there is an NFA $N$ that recognizes the same
%   language as $g$.
% \end{theorem}

% We make a pretty straightforward adaptation of Thompson's theorem to our setting,

% \begin{theorem}[Typed Thompson]
%   \label{thm:typthompson}
%   For $g$ a regular grammar $g$, there is an NFA $N$ such that $g$ is isomorphic
%   to $\mathsf{AccTrace}_{N}$.
% \end{theorem}

% \begin{proof}[Proof Sketch]
%   The following proof is currently formalized in Agda. Below
%   we give a walk through the structure of the proof.

%   Recall that regular grammars are inductively defined via
%   disjunction, concatenation, and Kleene star over literals
%   and the empty grammar. By induction over regular grammars,
%   we will construct an NFA that is equivalent to $g$.

%   First, define the recognizing NFA for the empty grammar
%   $I$.

%   \begin{figure}[h!]
%   \begin{tikzpicture}[node distance = 25mm ]
%     \node[state, initial] (1) {$1$};
%     \node[state, right of=1, accepting] (2) {$2$};

%     \path[->] (1) edge[below] node{$\varepsilon$} (2);
%   \end{tikzpicture}
%   \caption{$NFA(I)$}
%   \label{fig:emptyNFA}
%   \end{figure}

%   The type of traces from the initial state of $NFA(I)$ to the single
%   accepting state is given by,

%   \[
%     \mathsf{Trace}_{NFA(I)}(q_{1}, q_{2}) = \mu
%       \begin{pmatrix}
%          g_{q_{1}} := g_{q_{2}} \\
%          g_{q_{2}} := I
%       \end{pmatrix}. g_{q_{1}}
%   \]

%   The accepting traces through $NFA(I)$ are then described
%   as,

%   \[
%     \mathsf{AccTrace}_{NFA(I)} = \LinSigTy q {\{1 , 2\}} {\left( \mathsf{Trace}_{NFA(I)}(q_{1} , q) \amp \mathsf{acc}(q) \right)}
%   \]

%   A quick inspection of \cref{fig:emptyNFA} reveals that the
%   only reasonable choice for $q$ is $q_{2}$ --- because
%   state 2 is accepting while state 1 is not. Therefore,

%   \begin{align*}
%     \mathsf{AccTrace}_{NFA(I)}
%     & \cong \mathsf{Trace}_{NFA(I)}(q_{1} , q_{2}) \amp \mathsf{acc}(q_{2}) \\
%     & \cong \mathsf{Trace}_{NFA(I)}(q_{1} , q_{2}) \amp \top \\
%     & \cong \mathsf{Trace}_{NFA(I)}(q_{1} , q_{2})
%   \end{align*}

%   From here, to prove
%   $I \cong \mathsf{AccTrace}_{NFA(I)}$ it suffices to show
%   $I \cong \mathsf{Trace}_{NFA(I)}(q_{1} , q_{2})$. Below we
%   give two parse transformers, one from $I$ to
%   $\mathsf{Trace}_{NFA(I)}(q_{1} , q_{2})$ and vice versa. The formalized
%   artifact provides the low-level details that these are
%   indeed mutually inverse.

%   \[
%     \inferrule
%     {p : I \vdash \mathsf{nil} : \mathsf{Trace}(q_{2} , q_{2}) \\
%      \exists \text{~transition~} q_{1} \overset{\varepsilon}{\to} q_{2}
%     }
%     {p : I \vdash \mathsf{\varepsilon cons}(\mathsf{nil}) : \mathsf{Trace}_{NFA(I)}(q_{1} , q_{2})}
%   \]

%   Let $\gamma$ be the substitution $\{ g_{q_{2}} / g_{q_{1}}, I / g_{q_{2}} \}$,

%   \[
%     \inferrule
%     {
%       p : \mathsf{Trace}_{NFA(I)}(q_{1} , q_{2}) \vdash p :
%         \mathsf{Trace}_{NFA(I)}(q_{1} , q_{2}) \\
%       x_{1} : \simulsubst {g_{q_{2}}} {\gamma} = I \vdash x_{1} : I \\
%       x_{2} : \simulsubst {I} {\gamma} = I \vdash x_{2} : I
%     }
%     {
%       p : \mathsf{Trace}_{NFA(I)}(q_{1} , q_{2}) \vdash \mathsf{mfold}(x_{1}.x_{1} , x_{2}.x_{2})(p) : I
%     }
%   \]

%   This concludes the proof for the case of the empty
%   grammar. Let's now walk through the construction for
%   literal grammars. Given a character $c$, we construct an
%   NFA that recognizes only the string containing the single
%   character $c$ as,


%   \begin{figure}[h!]
%   \begin{tikzpicture}[node distance = 25mm ]
%     \node[state, initial] (1) {$1$};
%     \node[state, right of=1, accepting] (2) {$2$};

%     \path[->] (1) edge[below] node{$c$} (2);
%   \end{tikzpicture}
%   \caption{$NFA(c)$}
%   \label{fig:literalNFA}
%   \end{figure}

%   The automaton in \cref{fig:literalNFA} induces the
%   following type of traces from $q_{1}$ to $q_{2}$.

%   \[
%     \mathsf{Trace}_{NFA(c)}(q_{1}, q_{2}) = \mu
%       \begin{pmatrix}
%          g_{q_{1}} := c \otimes g_{q_{2}} \\
%          g_{q_{2}} := I
%       \end{pmatrix}. g_{q_{1}}
%   \]

%   Through the same argument as the empty grammar, the only
%   state that is accepting is $q_{2}$ and thus,

%   \[
%     \mathsf{AccTrace}_{NFA(c)} \cong \mathsf{Trace}_{NFA(c)}(q_{1} , q_{2})
%   \]

%   To show the desired isomorphism of $c \cong NFA(c)$ we
%   make a similar argument as we did for the empty grammar
%   $I$, except we leverage the $\mathsf{cons}$ rule instead
%   of $\mathsf{\varepsilon cons}$. That is, the parse
%   transformers in either direction are given as,

%   \[
%     \inferrule
%     {\cdot \vdash \mathsf{nil} : \mathsf{Trace}(q_{2} , q_{2}) \\
%      \exists \text{~transition~} q_{1} \overset{c}{\to} q_{2}
%     }
%     {p : c \vdash \mathsf{cons}(\mathsf{nil}) :
%       \mathsf{Trace}_{NFA(c)}(q_{1} , q_{2})}
%   \]


%   \[
%     \inferrule
%     {
%       p : \mathsf{Trace}_{NFA(c)}(q_{1} , q_{2}) \vdash p :
%         \mathsf{Trace}_{NFA(c)}(q_{1} , q_{2}) \\
%       x_{1} : \simulsubst {(c \otimes g_{q_{2}})} {\gamma} = c \otimes I \vdash \mathsf{unitR}(x_{1}) : c \\
%       x_{2} : \simulsubst {I} {\gamma} = I \vdash x_{2} : I
%     }
%     {
%       p : \mathsf{Trace}_{NFA(c)}(q_{1} , q_{2}) \vdash \mathsf{mfold}(x_{1}.\mathsf{unitR}(x_{1}) , x_{2}.x_{2})(p) : c
%     }
%   \]

%   Where $\gamma$ is the substitution
%   $\{ c \otimes g_{q_{2}} / g_{q_{1}}, I / g_{q_{2}} \}$ and
%   $\mathsf{unitR}$ is a witness to the isomorphism
%   $c \otimes I \cong c$. Again, we may see that these do
%   indeed mutually invert each other in the Agda code.

%   It remains to show that the desired isomorphisms are
%   preserved by $\otimes$, $\oplus$, and Kleene star. Here,
%   we will give the argument for the disjunction case, the
%   others are defined quite similarly.

%   Given two NFAs $N$ and $M$, we may define a new NFA that
%   encodes the disjunction of $N$ and $M$. Denote the
%   internal states of $N$ by $q_{j}$'s and the internal states
%   of $M$ by $r_{k}$'s,

%   \begin{figure}[h!]
%   \begin{tikzpicture}[node distance = 20mm ]
%     \node[state] (2) {$q_{init}$};
%     \node[state, initial, below left of=2] (1) {$init$};
%     \node[state, below right of=1] (3) {$r_{init}$};
%     \node[right of=2] (4) {$\cdots$};
%     \node[right of=3] (5) {$\cdots$};
%     % \node[state, below right of=1] (3) {$3$};
%     % \node[state, below of=2] (4) {$4$};

%     \path[->] (1) edge[below] node{$\varepsilon$} (2);
%     \path[->] (1) edge[below] node{$\varepsilon$} (3);
%     \path[->] (2) edge[below] node{} (4);
%     \path[->] (3) edge[below] node{} (5);

%     \node[label={[name=l] $N$}, draw,line width=2pt,rounded corners=5pt, fit=(2)(4)] {};
%     \node[label={[name=l] $M$}, draw,line width=2pt,rounded corners=5pt, fit=(3)(5)] {};
%   \end{tikzpicture}
%   \caption{$N \oplus_{NFA} M$}
%   \label{fig:disjunctionNFA}
%   \end{figure}

%   \cref{fig:disjunctionNFA} shows the process of
%   disjunctively combining NFAs. Precisely, we add a single
%   new state and we included copies of the states from each of $N$
%   and $M$. The new state acts as the initial state and has
%   $\varepsilon$-transitions to the initial states of $N$ and
%   $M$. We include all of the internal transitions from $N$
%   and $M$, and the accepting states of $N \oplus_{NFA} M$
%   are exactly the accepting states in each subautomaton.

%   Let $g$ and $g'$ be two regular grammars such that
%   $g \cong NFA(g)$ and $g' \cong NFA(g')$. As a matter of
%   notation\footnote{We shall similarly abuse notation for
%     $\otimes$ and Kleene. That is, for a regular grammar
%     $g$, when we write $NFA(g)$ we mean the NFA inductively
%     built up with the NFA-analogues to the constructors that
%     built up $g$.}, we will write $NFA(g \oplus g')$ for
%   $NFA(g) \oplus_{NFA} NFA(g')$. The traces of
%   $NFA(g \oplus g')$ are then given by,

%   \[
%     \mathsf{Trace}_{NFA(g \oplus g')}(src , dst) = \mu
%       \begin{pmatrix}
%         g_{init} := g_{q_{0}} \oplus g_{r_{0}} \\
%         g_{q_{j}} := \mathsf{Trans}_{NFA(g)}(q_{j}) \oplus \mathsf{isDst}(q_{j}) \\
%         g_{r_{k}} := \mathsf{Trans}_{NFA(g')}(r_{k}) \oplus \mathsf{isDst}(r_{k})
%       \end{pmatrix}.g_{src}
%     \]

%   where $\mathsf{Trans}$ is used to echo the same syntactic
%   definitions that appear in the $NFA(g)$ and $NFA(g')$.
%   Also, $src$ and $dst$ may take on any value in
%   $Q := \{init\} \cup \{q_{j}\} \cup \{r_{k}\}$, and
%   $\mathsf{isDst}(q)$ checks if $q$ is equal to $dst$. Which
%   is all to say, the traces of $NFA(g \oplus g')$ comprise
%   either a trace of $NFA(g)$, or a trace of $NFA(g')$, and
%   the transition coming out of $g_{init}$ determines which
%   subautomaton we step into.

%   The parse transformer from $g \oplus g'$ checks which side
%   of the sum type we are on, then takes the appropriate step
%   from $g_{init}$ in the automaton.

%   \[
%     \inferrule
%     {
%       u : g \vdash \iota (\phi (u)) : \mathsf{AccTrace}_{NFA(g \oplus g')} \\
%       v : g' \vdash \iota' (\psi (v)) : \mathsf{AccTrace}_{NFA(g \oplus g')}
%     }
%     {p : g \oplus g' \vdash \mathsf{case}~p \{ \mathsf{inl}(u) \mapsto s , \mathsf{inr}(v) \mapsto r \} : \mathsf{AccTrace}_{NFA(g \oplus g')}}
%   \]

%   with $\iota$ and $\iota'$ as embeddings from $NFA(g)$ and
%   $NFA(g')$, respectively, into $NFA(g \oplus g')$,
%   $\phi: g \cong \mathsf{AccTrace}_{NFA(g)}$, and $\psi: g' \cong \mathsf{AccTrace}_{NFA(g')}$. On a
%   high level, all this construction does is turn a parse of
%   $g$ into a parse of $NFA(g)$ and then embeds that inside
%   of the larger automaton $NFA(g \oplus g')$. Likewise for $g'$.

%   In the other direction, recall that the data of an
%   accepting trace for $NFA(g \oplus g')$ is a pair of a
%   trace and a proof that
%   the end state $q'$ of that trace is accepting. By
%   multifolding over the first part of that pair, we turn the
%   term of type
%   $\mathsf{Trace}_{NFA(g \oplus g')}(init , q')$ into a
%   trace of either of the subautomata,

%   \[
%     p : \mathsf{Trace}_{NFA(g \oplus g')}(init , q') \vdash \mathsf{mfold}_{NFA(g \oplus g')} : \mathsf{Trace}_{NFA(g)}(q_{0} , q') \oplus \mathsf{Trace}_{NFA(g')}(r_{0} , q')
%   \]

%   Additionally, we leverage the fact that the only accepting
%   states for $NFA(g \oplus g')$ are those from the
%   subautomata to extract that $q'$ must be an accepting
%   state from a subautomaton.

%   \[
%     x : \mathsf{acc}_{NFA(g \oplus g')}(q') \vdash M : \mathsf{acc}_{NFA(g)}(q') \oplus \mathsf{acc}_{NFA(g')}(q')
%   \]

%   We then combine the trace and proof of acceptance into an
%   accepting trace of one of the subautomata,

%   \[
%     p : \mathsf{AccTrace}_{NFA(g \oplus g')} \vdash N : \mathsf{AccTrace}_{NFA(g)} \oplus \mathsf{AccTrace}_{NFA(g')}
%   \]

%   Lastly, we then inductively use the isomorphisms $\phi$
%   and $\psi$ to turn the accepting traces into a parse of
%   $g$ or $g'$,

%   \[
%      N : \mathsf{AccTrace}_{NFA(g)} \oplus \mathsf{AccTrace}_{NFA(g')} \vdash \mathsf{case}~N~\{\mathsf{inl}(n) \mapsto \phi^{-1}(n), \mathsf{inr}(n') \mapsto \psi^{-1}(n')\} : g \oplus g'
%   \]

%   As discussed above, the other cases and low-level details
%   of isomorphism are reserved for the formalization. This
%   concludes the proof of our variant of Thompson's construction.
% \end{proof}

% \subsection{Other Automata}
% \subsubsection{Pushdown Automata}
% \label{subsubsec:pda}
% A (nondeterministic) \emph{pushdown automaton} is an automaton that employs a
% stack. Just like NFAs, they have transitions labeled with characters from a
% fixed string alphabet $\Sigma$. Additionally, they maintain a stack of
% characters drawn from a stack alphabet $\Gamma$. They are often represented
% formally as a 7-tuple $(Q, \Sigma, \Gamma, \delta, q_{0}, Z, F)$,

% \begin{itemize}
%   \item $Q$ a finite set of states
%   \item $\Sigma$ a fixed, finite string alphabet
%   \item $S$ a fixed, finite stack alphabet
%   \item
%         $\delta \subset Q \times (\Sigma \cup \{ \varepsilon \}) \times S \to \mathcal{P}(Q \times S^{*})$
%         the transition function
%   \item $q_{0} \in Q$ the start state
%   \item $Z \in S$ the initial stack symbol
%   \item $F \subset Q$ the accepting states
% \end{itemize}

% We encode the traces of a pushdown automaton very similarly to those of an NFA,
% except the transitions of a PDA are instead encoded via the linear-non-linear
% $\Pi$-type. This is because of simply transitioning via character, a PDA must
% also pop and push characters onto a stack, which is used as the argument to
% these dependent functions.

% Let $S$ be a non-linear type encoding the stack
% alphabet, and build lists over $S$ as the (non-linear) least fixed-point
% $\mathsf{List}(S) := \mu X . 1 + S \times X$. Then, the type of states for a
% PDA $P$ with stack alphabet $S$ are given as a functions that takes in lists $L$,
% and then makes a case distinction between possible transitions based on what was witnessed as $\mathsf{head}(L)$. The choice of transition
% will then determine which character to transition by and what word $w$ should be
% pushed onto the stack. The word that is added to the top of the stack is
% appended to $\mathsf{tail}(L)$ and then we recursively step into another state
% called on argument $w + \mathsf{tail}(L)$.

% % \begin{gather*}
% %   \mathsf{State} \in {g _{q} : q \in Q} \\
% %   \mathsf{Word} \in \String \\
% %   \mathsf{Char} \in \Sigma   \\
% %   \mathsf{StackChar} \in \Gamma \\
% % \end{gather*}
% % \begin{align*}
% %   \mathsf{Trans}_{P}(q) ::=~&
% %                            \LinPiTy {(hd :: tl)} {\mathsf{List}(S)} {\left( \mathsf{Char} \otimes \mathsf{State}(\mathsf{Word} + tl) \right)}~| \\
% %   & \LinPiTy {(hd :: tl)} {\mathsf{List}(S)} {\left(  \mathsf{State}(\mathsf{Word} + tl) \right)}~|~\mathsf{Trans}_{P}(q) \oplus \mathsf{Trans}_{P}(q)
% % \end{align*}

% % That is to say, when transitioning a PDA pops off the head $hd$ of the stack

% \subsubsection{Turing Machines}
% \label{subsubsec:tm}

% In
% \cref{subsubsec:pda}, we gave a grammar presentation of
% traces through a PDA by using a non-linear type $S$
% to encode the stack. We may similarly use pairs $S \times S$
% to encode the tape of a Turing machine. With two stacks we can simulate the behavior of the
% infinite tape of a Turing machine. The intuition behind this correspondence is
% that the left half of the tape is on one stack, the right
% half of the tape the other, and we treat the tops of stacks
% like the head of the tape.

% Due to computability limitations, we are not able to extract
% a verified parser for all recursively enumerable languages.
% Yet we should still reflect on the generality of our syntax,
% as it allows a natural encoding of Turing machines.

% \section{Future Work}
% \label{sec:future}

% \subsection{Implementation for Context-Free Grammars}
% As suggested throughout the paper and briefly explored in
% \cref{subsubsec:pda}, the first extension to the work in
% this paper will be to bring analogous constructions to
% context-free grammars and their accompanying pushdown automata.

% \subsection{Beyond Strings}
% \label{subsec:beyond}

% While parsing typically refers to the generation of semantic
% objects from string input, many tasks in programming can be
% viewed as parsing of objects with more structure, such as
% trees with binding structure or graphs. Fundamental to the
% frontend of many
% programming language implementations are type systems. In
% particular, \emph{type checking}
% --- analogous to language recognition --- and \emph{typed
%   elaboration} --- analogous to parsing --- arise when
% producing a semantic object subject to some analysis. Just
% as our string grammars were given as functors from $\String$
% to $\Set$, we envision adapting the same philosophy
% to functors from $\String$ to \emph{trees} to craft a syntax
% that natively captures typed elaboration. This suggests an
% unusual sort of bunched type theory, where context extension
% no longer resembles concatenation of strings but instead
% takes on the form of tree constructors.

% Thus far, our theory has proved useful for internalizing long-standing
% grammar-theoretic constructions, but there has been decades of research
% conducted since. A fruitful avenue for future work includes
% testing if our formalism can also internalize more recently proposed grammar
% mechanisms, such as the interval parsing grammars given by Zhang et al. \cite{zhangIntervalParsingGrammars2023}.

% \newpage

% \bibliographystyle{plain}
% \bibliography{refs.bib}

% \newpage

% \appendix

% \section{Inference Rules}
% \label{sec:infer}

% \begin{figure}[h!]
%   \label{fig:structjdg}
%   \begin{mathpar}
%     \inferrule{~}{\ctxwff \cdot}
%     \and
%     \inferrule{\ctxwff \Gamma \\ \ctxwffjdg \Gamma X}{\ctxwff {\Gamma, x : X}}

%     \\

%     \inferrule{~}{\linctxwff \Gamma \cdot}
%     \and
%     \inferrule{\linctxwff \Gamma \Delta \\ \linctxwffjdg \gamma A}{\linctxwff \Gamma {\Delta, a : A}}

%     \\

%     \inferrule{\Gamma \vdash X : U_i}{\ctxwffjdg \Gamma X}

%     \and

%     \inferrule{\Gamma \vdash A : L_i}{\linctxwffjdg \Gamma A}

%     \\

%     \inferrule{\Gamma \vdash X \equiv Y : U_i}{\ctxwffjdg \Gamma {X\equiv Y}}

%     \and

%     \inferrule{\Gamma \vdash A \equiv B : L_i}{\linctxwffjdg \Gamma {A \equiv B}}

%   \end{mathpar}
%   \caption{Structural judgments}
% \end{figure}

% \begin{figure}
%   \label{fig:typewf}
%   \begin{mathpar}
%     \inferrule{~}{\Gamma \vdash U_i : U_{i+1}}
%  %
%     \and
% %
%     \inferrule{~}{\Gamma \vdash L_i : U_{i+1}}
% %
%     \\
% %
%     \inferrule{\Gamma \vdash X : U_i \\ \hspace{-0.1cm} \Gamma, x : X \vdash Y : U_i}{\Gamma \vdash \PiTy x X Y : U_i }%
% %
%     \and
% %
%     \inferrule{\Gamma\vdash X : U_i \\ \hspace{-0.1cm} \Gamma, x : X \vdash Y : U_i}{\Gamma \vdash \SigTy x X Y : U_i}
% %
%     \\
% %
%     \inferrule{~}{\Gamma \vdash 1 : U_i}
% %
%     \and
% %
%     \inferrule{\Gamma \vdash A : L_i}{\Gamma \vdash G A : U_i}
% %
%     \\
% %
%     \inferrule{~}{\Gamma \vdash I : L_i}
%  %
%     \and
% %
%     \inferrule{\Gamma \vdash A : L_i \\ \hspace{-0.1cm}\Gamma \vdash B : L_i}{\Gamma \vdash A \otimes B : L_i}
% %
%     \and
% %
%     \inferrule{\Gamma \vdash A : L_i \\ \hspace{-0.1cm}\Gamma \vdash B : L_i}{\Gamma \vdash A \lto B : L_i}
% %
%     \and
% %
%     \inferrule{\Gamma \vdash A : L_i \\ \hspace{-0.1cm}\Gamma \vdash B : L_i}{\Gamma \vdash B \tol A : L_i}
% %
%     \\
% %
%     \inferrule{\Gamma \vdash X : U_i \\ \Gamma, x : X \vdash A : L_i}{\Gamma \vdash \LinPiTy x X A : L_i}
% %
%     \and
% %
%     \inferrule{\Gamma \vdash X : U_i \\ \Gamma, x : X \vdash A : L_i}{\Gamma \vdash \LinSigTy x X A : L_i}
% %
%     \\
% %
%     \inferrule{\Gamma \vdash X : U_i \quad \{\Gamma \vdash e_i : X\}_i}{\Gamma \vdash e_1 =_X e_2 : U_i}
%     %
%     \and
%     %
%     \inferrule{~}{\Gamma \vdash \top : L_i}
% %
%     \and
% %
%     \inferrule{\Gamma \vdash A : L_i \quad \Gamma \vdash B : L_i}{\Gamma \vdash A \amp B : L_i}
% %
%     \\
%     %
%     \inferrule{c \in \Sigma}{\Gamma \vdash c : L_0}
%     %
%     \and
%     %
%     \inferrule{\Gamma, x : L_i \vdash A : L_i \and A \textrm{ strictly positive}}{\Gamma \vdash \mu x.\, A : L_i}
%   \end{mathpar}
%   \caption{Type well-formedness}
% \end{figure}

% \begin{figure}
%   \label{fig:jdgeq}
%   \begin{mathpar}
%     \inferrule{\Gamma \vdash p : e =_X e'}{\Gamma \vdash e \equiv e' : X}
% %
%     \and
% %
%     \inferrule{~}{\Gamma \vdash \app {(\lamb x e)} {e'} \equiv \subst x e {e'} : X}
% %
%     \and
% %
%     \inferrule{~}{\Gamma \vdash e \equiv \lamb x {\app e x} : \PiTy x X Y}
% %
%     \and
% %
%     \inferrule{~}{\Gamma \vdash \pi_1\, (e_1, e_2) \equiv e_1 : X}
% %
%     \and
% %
%     \inferrule{~}{\Gamma \vdash \pi_2\, (e_1, e_2) \equiv e_2 : \subst x {e_1} Y}
% %
%     \and
% %
%     \inferrule{~}{\Gamma \vdash e \equiv (\pi_1\, e, \pi_2\, e) : \SigTy x X Y}
% %
%     \and
% %
%     \inferrule{~}{}
% %
%     \inferrule{~}{\Gamma \vdash t \equiv t' : 1}
% %
%     \and
% %
%     \inferrule{~}{\Gamma \vdash G\, (G^{-1} \, t) \equiv t : G A}
% %
%     \and
% %
%     \inferrule{~}{\Gamma; \cdot \vdash G^{-1}\, (G \, t ) \equiv t: A}
% %
%     \and
% %
%     \inferrule{~}{\Gamma; \Delta \vdash \app {(\lamblto a e)} {e'} \equiv \subst e x {e'} : C}
% %
%     \and
% %
%     \inferrule{~}{\Gamma; \Delta \vdash e \equiv \lamblto a {\app e a} : A \lto B}
% %
%     \and
% %
%     \inferrule{~}{\Gamma; \Delta \vdash \app {(\lambtol a e)} {e'} \equiv \subst e x {e'} : C}
% %
%     \and
% %
%     \inferrule{~}{\Gamma; \Delta \vdash e \equiv \lambtol a {\app e a} : A \tol B}
% %
%     \and
% %
%     \inferrule{~}{\Gamma; \Delta \vdash \app {(\dlamb x e)} {e'} \equiv \subst x a {e'} : C}
% %
%     \and
% %
%     \inferrule{~}{\Gamma; \Delta \vdash e \equiv \dlamb x {\app e x} : \LinPiTy x X A}
% %
%     \and
% %
%     \inferrule{~}{\Gamma; \Delta \vdash e \equiv e' : \top}
% %
%     \and
% %
%     \inferrule{~}{\Gamma; \Delta \vdash e_i \equiv \pi_i (e_1, e_2) : A_i}
% %
%     \and
% %
%     \inferrule{~}{\Gamma; \Delta \vdash e \equiv (\pi_1 e, \pi_2 e) : A\& B}
% %
%     \and
% %
%     \inferrule{~}{\Gamma; \Delta \vdash \letin {()} {()} e \equiv e : C}
%     %
%     \and
%     %
%     \inferrule{~}{\Gamma; \Delta \vdash \letin {()} e {\subst {e'} {()} a} \equiv \subst {e'} a e : C}
% %
%     \and
% %
%     \inferrule{~}{\Gamma; \Delta \vdash \letin {e \otimes e'} {a \otimes a'} e'' \equiv \subst {e''} {a, a'} {e, e'} : C}
% %
%     \and
%     %
%     \inferrule{~}{\Gamma; \Delta \vdash \letin {a \otimes b} e {\subst {e'} {a \otimes b} c} \equiv \subst {e'} c e : C}
% %
%     \and
% %
%     \inferrule{~}{\Gamma;\Delta \vdash \letin {(x, a)} {(e, e')} {e''} \equiv \subst {e''} {x, a} {e, e'} : C}
%     %
%     \and
%     %
%     \inferrule{~}{\Gamma; \Delta \vdash \letin {(x, a)} e {\subst {e'} {(x, a)} y} \equiv \subst {e'} e y : C}
% \end{mathpar}
%   \caption{Judgmental equality}
% \end{figure}

% \begin{figure}
%   \label{fig:inttyping}
%   \begin{mathpar}
%   \inferrule{~}{\Gamma, x : X, \Gamma' \vdash x : X}
%   %
%   \and
%   %
%   \inferrule{\Gamma \vdash e : Y \quad \ctxwffjdg \Gamma {X \equiv Y}}{\Gamma \vdash e : X}
%   %
%   \\\
%   %
%   \inferrule{~}{\Gamma \vdash () : 1}
%   %
%   \and
%   %
%   \inferrule{\Gamma \vdash e : X \\ \Gamma \vdash e : \subst Y e x}{\Gamma \vdash (e, e') : \SigTy x X Y}
%   %
%   \\
% %
%   \inferrule{\Gamma \vdash e : \SigTy x X Y}{\Gamma \vdash \pi_1\, e : X}
%   %
%   \and
%   %
%   \inferrule{\Gamma \vdash e : \SigTy x X Y}{\Gamma \vdash \pi_2\, e : \subst Y {\pi_1\, e} x}
%   \and
%   \inferrule{\Gamma, x : X \vdash e : Y}{\Gamma \vdash \lamb x e : \PiTy x X Y}
%   %
%   \and
%   %
%   \inferrule{\Gamma \vdash e : \PiTy x X Y \\ \Gamma \vdash e' : X}{\Gamma \vdash \app e {e'} : \subst Y {e'} x}
%   %
%   \\
%   %
%   \inferrule{\Gamma \vdash e \equiv e' : X}{\Gamma \vdash \mathsf{refl} : e =_X e'}
%   \and
%   \inferrule{\Gamma ; \cdot \vdash e : A}{\Gamma \vdash \mathsf G e : \mathsf G A}
%   \end{mathpar}
%   \caption{Intuitionistic typing}
% \end{figure}

% \begin{figure}
%   \label{fig:linsyntax}
%   \begin{mathpar}
%     \inferrule{~}{\Gamma ; a : A \vdash a : A}
%     \and
%     \inferrule{\Gamma ; \Delta \vdash e : B \\ \linctxwffjdg \Gamma {A \equiv B}}{\Gamma ; \Delta \vdash e : A}
%     %
%     \\
%     %
%     \inferrule{~}{\Gamma ; \cdot \vdash () : I}
%     \and
%     \inferrule{\Gamma ; \Delta \vdash e : I \\ \Gamma ; \Delta_1',\Delta_2' \vdash e' : C}{\Gamma ; \Delta_1',\Delta,\Delta_2' \vdash \letin {()} e {e'} : C}
%     %
%     \\
%     %
%     \inferrule{\Gamma ; \Delta \vdash e : A \\ \Gamma ; \Delta' \vdash e' : B}{\Gamma ; \Delta, \Delta' \vdash e \otimes e' : A \otimes B}
%     %
%     \\
%     %
%     \inferrule{\Gamma ; \Delta \vdash e : A \otimes B \\ \Gamma ; \Delta'_1, a : A, b : B, \Delta'_2 \vdash e'}{\Gamma ;  \Delta_1', \Delta, \Delta'_2 \vdash \letin {a \otimes b} e {e'}}
%     \\
%     %
%     \inferrule{\Gamma ; a : A , \Delta \vdash e : B}{\Gamma ; \Delta \vdash \lamblto a e : A\lto B}
%     \and
%     \inferrule{\Gamma ; \Delta' \vdash e' : A \\ \Gamma ; \Delta \vdash e : A \lto B}{\Gamma ; \Delta', \Delta \vdash \applto {e'} {e} : B}
%     \\
%     %
%     \inferrule{\Gamma ; \Delta , a : A \vdash e : B}{\Gamma ; \Delta \vdash \lambtol a e : B\tol A}
%     \and
%     \inferrule{\Gamma ; \Delta \vdash e : B \tol A \\ \Gamma ; \Delta' \vdash e' : A}{\Gamma ; \Delta, \Delta' \vdash \apptol e {e'} : B}
%     %
%     \\
%     %
%     \inferrule{\Gamma, x : X ; \Delta  \vdash e : A}{\Gamma ; \Delta \vdash \dlamb x e : \LinPiTy x X A}
%     \and
%     \inferrule{\Gamma ; \Delta \vdash e : \LinPiTy x X A \\ \Gamma \vdash e' : X}{\Gamma ; \Delta \vdash \app e {e'} : \subst A {e'} x}
%     %
%     \\
%     %
%     \inferrule{\Gamma \vdash e : X \quad \Gamma ; \Delta \vdash e' : \subst A e x}{\Gamma ; \Delta \vdash (e, e') : \LinSigTy x X A}
%     %
%     \\
%     %
%     \inferrule{\Gamma ; \Delta \vdash e : \LinSigTy x X A \quad \Gamma, x : X ; \Delta'_1, a : A, \Delta'_2 \vdash e' : C}{\Gamma; \Delta'_1, \Delta, \Delta'_2 \vdash \letin {(x, a)} e {e'}: C}
%     %
%     \\
%     %
%     \inferrule{~}{\Gamma ; \Delta \vdash () : \top}
%     %
%     \\
%     %
%     \inferrule{\Gamma ; \Delta \vdash e_1 : A_1 \quad \Gamma ; \Delta \vdash e_2 : A_2}{\Gamma ; \Delta \vdash (e_1, e_2) : A_1 \amp A_2}
%     \and
%     \inferrule{\Gamma ; \Delta \vdash e : A_1 \amp A_2 }{\Gamma ; \Delta \vdash \pi_i \, e : A_i}
%     %
%     \\
%     %
%     \inferrule{\Gamma \vdash e : \mathsf{G} A}{\Gamma ; \cdot \vdash \mathsf{G}^{-1}\, e : A}
%     %
%     \\
%     %
%     \inferrule{\Gamma; \Delta \vdash e : \subst A {\mu x.\, A} x}{\Gamma; \Delta \vdash \mathsf{cons}\, e : \mu x.\, A}
%     \and
%     \inferrule{\Gamma;\Delta\vdash e' : \mu x.\,A \and \Gamma; a:\subst A B x \vdash e : B}{\Gamma;\Delta\vdash \mathsf{fold}(a.e)(e') : B}
%   \end{mathpar}
%   \caption{Linear typing}
% \end{figure}
\end{document}
