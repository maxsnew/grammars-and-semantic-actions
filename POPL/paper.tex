% -*- fill-column: 80; -*-
\documentclass[sigconf,review,screen]{acmart}
\usepackage{mathpartir}
\usepackage{tikz-cd}
\usepackage{enumitem}
\usepackage{wrapfig}
\usepackage{fancyvrb}
\usepackage{hyperref}
\usepackage{cleveref}
\usepackage{stmaryrd}

\usepackage{todonotes}
\newcommand{\todoin}[1]{\todo[inline]{TODO:\@ #1}}

\newcommand{\sem}[1]{\llbracket{#1}\rrbracket}
\newcommand{\cat}[1]{\mathbf{#1}}
\newcommand{\lto}{\multimap}
\newcommand{\tol}{\mathrel{\rotatebox[origin=c]{180}{$\lto$}}}
\newcommand{\Set}{\mathbf{Set}}
\newcommand{\Gr}{\mathbf{Gr}}
\newcommand{\Type}{\mathbf{Type}}
\newcommand{\Prop}{\mathbf{Prop}}
\newcommand{\Bool}{\mathbf{Bool}}
\newcommand{\nat}{\mathbb{N}}

\newcommand{\gluedNL}{{\mathcal G}_S}
\newcommand{\gluedNLUniv}{{\mathcal G}_{S,i}}
\newcommand{\gluedL}{{\mathcal G}_L}

\newcommand{\simulsubst}[2]{#1\{#2\}}
\newcommand{\subst}[3]{\simulsubst {#1} {#2/#3}}
\newcommand{\letin}[3]{\mathsf{let}\, #1 = #2 \, \mathsf{in}\, #3}
\newcommand{\lamb}[2]{\lambda #1.\, #2}
\newcommand{\lamblto}[2]{\lambda^{{\lto}} #1.\, #2}
\newcommand{\lambtol}[2]{\lambda^{{\tol}} #1.\, #2}
\newcommand{\dlamb}[2]{\overline{\lambda} #1.\, #2}
\newcommand{\app}[2]{#1 \, #2}
\newcommand{\applto}[2]{#1 \mathop{{}^{\lto}} #2}
\newcommand{\apptol}[2]{#1 \mathop{{}^{\tol}} #2}
\newcommand{\PiTy}[3]{\Pi #1 : #2.\, #3}
\newcommand{\SigTy}[3]{\Sigma #1 : #2.\, #3}
\newcommand{\LinPiTy}[3]{\widebar\Pi #1 : #2.\, #3}
\newcommand{\LinSigTy}[3]{\widebar\Sigma #1 : #2.\, #3}
\newcommand{\amp}{\mathrel{\&}}
\newcommand{\GrTy}{\mathsf{Gr}}

\newcommand{\ctxwff}[1]{#1 \,\, \mathsf{ok}}
\newcommand{\ctxwffjdg}[2]{#1 \vdash #2 \,\, \mathsf{type}}
\newcommand{\linctxwff}[2]{#1 \vdash #2 \,\, \mathsf{ok}}
\newcommand{\linctxwffjdg}[2]{#1 \vdash #2 \,\, \mathsf{linear}}

\newif\ifdraft
\drafttrue
\newcommand{\steven}[1]{\ifdraft{\color{orange}[{\bf Steven}: #1]}\fi}
\renewcommand{\max}[1]{\ifdraft{\color{blue}[{\bf Max}: #1]}\fi}
\newcommand{\pedro}[1]{\ifdraft{\color{red}[{\bf Pedro}: #1]}\fi}
\newcommand{\pipe}{\,|\,}

%% Rights management information.  This information is sent to you
%% when you complete the rights form.  These commands have SAMPLE
%% values in them; it is your responsibility as an author to replace
%% the commands and values with those provided to you when you
%% complete the rights form.
\setcopyright{acmcopyright}
\copyrightyear{2018}
\acmYear{2018}
\acmDOI{XXXXXXX.XXXXXXX}

%% These commands are for a PROCEEDINGS abstract or paper.
\acmConference[Woodstock '18]{Woodstock '18: ACM Symposium on Neural
  Gaze Detection}{June 03--05, 2018}{Woodstock, NY}
\acmBooktitle{Woodstock '18: ACM Symposium on Neural Gaze Detection,
  June 03--05, 2018, Woodstock, NY}
\acmPrice{15.00}
\acmISBN{978-1-4503-XXXX-X/18/06}


\begin{document}

\title{Formal Grammars as Functors and Formal Grammars as Types in Non-commutative Linear-Non-Linear Type Theory}
\author{Steven Schaefer}
\affiliation{
  \department{Electrical Engineering and Computer Science}
  \institution{University of Michigan}
  \country{USA}
}
\email{stschaef@umich.edu}

\author{Max S. New}
\affiliation{
  \department{Electrical Engineering and Computer Science}
  \institution{University of Michigan}
  \country{USA}
}
\email{maxsnew@umich.edu}

\author{Pedro H. Azevedo de Amorim}
\affiliation{
  \department{Department of Computer Science}
  \institution{University of Oxford}
  \country{UK}
}
\email{pedro.azevedo.de.amorim@cs.ox.ac.uk}

\begin{abstract}
  We propose a semantic framework for the study of formal grammars. First, we
  provide a syntax-independent notion of formal grammar as a function from
  strings to sets of parse trees, generalizing the familiar notion of a language
  as a set of strings. This point-of-view naturally endows a grammar with the
  structure of a presheaf, which allows many language-theoretic constructs to be
  defined via universal properties.

  With these categorical constructions in mind, we propose a new syntactic
  formalism for formal grammars: a version of dependent linear-non-linear type
  theory where the tensor product is non-commutative. The linear types in this
  theory are interpreted as grammars --- and linear terms as parse transformers.
  The non-dependent fragment of this type theory is already expressive enough to
  capture regular expressions, context-free expressions, and finite automata.
  Moreover, the isomorphism between regular languages and finite automata is
  definable as a proof term in our formalism. The
  non-dependent fragment of this type theory is expressive enough to capture
  both regular and context-free expressions, as well as finite automata already
  regular expressions, context-free expressions, and finite automata. Moreover,
  the is definable as a proof term in this formalism. Further, through use of
  dependency of linear types on non-linear types, we can express richer grammar
  formalisms such as indexed grammars, pushdown automata, and Turing machines.

  Classically, many results in both language and automata theory are proved
  informally with respect to some ambient foundation --- e.g. Kleene algebra, grammars,
  parsers, semantic actions, etc. The type theory provided in
  this paper serves as an intuitive and general framework to syntactically unify
  these related ideas and internalize their
  theorems formally. We give a new form of \emph{logical characterization} of
  grammar classes based on a substructural logic rather than ordinary
  first-order logic. Further, the type theory can not only
  express the grammars as types, but equivalences between grammars and
  parsers as terms of the grammar.

  We give this type theory a semantics in the category of grammars and
  prove a related canonicity theorem. Moreover, we keep an eye toward the
  implementation of this type theory for correct-by-construction parsers and
  parser combinators.
\end{abstract}

\maketitle

\section{A Syntax-Independent Notion of Syntax}
\todoin{Rework intro from LICS draft with a focus on implementation}

\section{Kleene Categories}
\todoin{Tell a story winding through the use of KAs, action algebras, etc to
  motivate why we'd want categorical semantics for KCats}
\todoin{Incoporate Kleene Category definition}
\todoin{End with the punch that $\Set^{\Sigma^{*}}$ is a KCat}

\section{Automata}

\end{document}
